%% LyX 2.3.2 created this file.  For more info, see http://www.lyx.org/.
%% Do not edit unless you really know what you are doing.
\documentclass[12pt,twoside,english]{article}
\usepackage{mathptmx}
\usepackage{helvet}
\renewcommand{\ttdefault}{lmtt}
\usepackage[LGR,T1]{fontenc}
\usepackage[latin9]{inputenc}
\usepackage[letterpaper]{geometry}
\geometry{verbose,tmargin=1in,bmargin=1in,lmargin=1.2in,rmargin=1in}
\usepackage{fancyhdr}
\pagestyle{fancy}
\setcounter{tocdepth}{2}
\setlength{\parskip}{\medskipamount}
\setlength{\parindent}{0pt}
\usepackage{color}
\definecolor{page_backgroundcolor}{rgb}{1, 1, 1}
\pagecolor{page_backgroundcolor}
\usepackage{babel}
\usepackage{array}
\usepackage{verbatim}
\usepackage{longtable}
\usepackage{varioref}
\usepackage{prettyref}
\usepackage{fancybox}
\usepackage{calc}
\usepackage{textcomp}
\usepackage{url}
\usepackage{ifthen}
\usepackage{footnotehyper}
\usepackage{amsmath}
\usepackage{splitidx}
\makeindex
\usepackage{setspace}
\PassOptionsToPackage{normalem}{ulem}
\usepackage{ulem}
\newindex{idx}
\newindex{var}
\newindex{lis}
\newindex{con}
\usepackage[unicode=true,
 bookmarks=true,bookmarksnumbered=false,bookmarksopen=false,
 breaklinks=true,pdfborder={0 0 0},pdfborderstyle={},backref=false,colorlinks=true]
 {hyperref}
\hypersetup{pdftitle={Technical Note: Processing Algorithms},
 pdfauthor={RAF},
 pdfsubject={variables in data files and the algorithms used to produce them},
 pdfkeywords={algorithm, NCAR Research Aviation Facility, research aircraft, NCAR/EOL/RAF}}

\makeatletter

%%%%%%%%%%%%%%%%%%%%%%%%%%%%%% LyX specific LaTeX commands.

\makesavenoteenv{tabular}

\newcommand{\noun}[1]{\textsc{#1}}
\DeclareRobustCommand{\greektext}{%
  \fontencoding{LGR}\selectfont\def\encodingdefault{LGR}}
\DeclareRobustCommand{\textgreek}[1]{\leavevmode{\greektext #1}}
\ProvideTextCommand{\~}{LGR}[1]{\char126#1}

%% Because html converters don't know tabularnewline
\providecommand{\tabularnewline}{\\}

%%%%%%%%%%%%%%%%%%%%%%%%%%%%%% Textclass specific LaTeX commands.
\newlength{\lyxhang}
\IfFileExists{hanging.sty}{
  \usepackage{hanging}
  \newenvironment{hangparagraphs}
    {%
      \ifthenelse{\lengthtest{\parindent > 0pt}}%
        {\setlength{\lyxhang}{\parindent}}%
        {\setlength{\lyxhang}{2em}}%
      \par\begin{hangparas}{\lyxhang}{1}%
    }
    {\end{hangparas}}
}{%else
  \newenvironment{hangparagraphs}
    {%
      \ifthenelse{\lengthtest{\parindent > 0pt}}%
        {\setlength{\lyxhang}{\parindent}}%
        {\setlength{\lyxhang}{2em}}%
      \begin{hangparas}%
    }
    {\end{hangparas}}
  \newcommand{\hangpara}{\hangindent \lyxhang \hangafter 1 \noindent}
  \newenvironment{hangparas}{\setlength{\parindent}{\z@}
  \everypar={\hangpara}}{\par}
}
\newenvironment{lyxcode}
	{\par\begin{list}{}{
		\setlength{\rightmargin}{\leftmargin}
		\setlength{\listparindent}{0pt}% needed for AMS classes
		\raggedright
		\setlength{\itemsep}{0pt}
		\setlength{\parsep}{0pt}
		\normalfont\ttfamily}%
	 \item[]}
	{\end{list}}
\newenvironment{lyxlist}[1]
	{\begin{list}{}
		{\settowidth{\labelwidth}{#1}
		 \setlength{\leftmargin}{\labelwidth}
		 \addtolength{\leftmargin}{\labelsep}
		 \renewcommand{\makelabel}[1]{##1\hfil}}}
	{\end{list}}

\@ifundefined{date}{}{\date{}}
%%%%%%%%%%%%%%%%%%%%%%%%%%%%%% User specified LaTeX commands.
% macro for italic page numbers in the index
\newcommand{\IndexDef}[1]{\textit{#1}}
\newcommand{\IndexPrimary}[1]{\textbf{#1}}

% workaround for a makeindex bug,
% see sec. "Index Entry Order"
% only uncomment this when you are using makindex
\let\OrgIndex\index 
\renewcommand*{\index}[1]{\OrgIndex{#1}}
\usepackage{splitidx}
%\indexsetup{noclearpage}
\marginparsep 0.3 cm
\marginparwidth 2 cm 
\pagenumbering{roman}
% use this to make target one line higher
\makeatletter
 \newcommand{\nop}[1]{\Hy@raisedlink{\hypertarget{#1}{}}}
\makeatother

\AtBeginDocument{
  \def\labelitemii{\(\circ\)}
  \def\labelitemiii{\(\triangleright\)}
}

\makeatother

\begin{document}
\title{RAF Technical Note: Processing Algorithms}
\maketitle
\begin{center}
Algorithms Used to Produce Data Products from Research Aircraft;\\
also Definitions of Variables in Archived Data Files
\par\end{center}
\date{\centerline{version: February 2019}}

\vskip2in
\author{Research Aviation Facility, Earth Observing Laboratory\\
National Center for Atmospheric Research\\
PO Box 3000, Boulder CO ~80307-3000}

\vskip1inCurrent PDF version of this document: \href{https://github.com/NCAR/aircraft_ProcessingAlgorithms/blob/master/ProcessingAlgorithms.pdf}{Processing Algorithms}

\vfill\eject

\pagebreak{}

~~~~

\newpage{}

\renewcommand{\contentsname}{Table of Contents}
\phantomsection
\addcontentsline{toc}{section}{Table of Contents}
\tableofcontents{}
\cleardoublepage
\renewcommand{\abstractname}{Preface and Abstract}
%\phantomsection
%\begin{abstract}
%\addcontentsline{toc}{section}{Preface and Abstract}
%xxx
%\end{abstract}
\clearpage
\pagenumbering{arabic}

\vfill\eject


\section{INTRODUCTION\label{subsec:General-Comments}}

\subsection{Background Information}

This technical report defines the variables used in data sets that
are collected by the research aircraft operated by the Research Aviation
Facility (RAF) of the National Center for Atmospheric Research. Where
appropriate, it also documents the equations that are used by the
processing software (currently ``nimbus''\index{nimbus}) to calculate
the derived measurements that result from the use of one or more other
basic measurements (e.g., potential temperature). Since 1993, data
from research flights have been archived in NetCDF format\index{NetCDF!format}
(cf. \url{http://www.unidata.ucar.edu/software/netcdf/docs/}), and
the NetCDF header\index{NetCDF!header} for recent projects includes
detailed information on the measurements present in the file, how
they depend on other measurements, units, etc. The conventions that
the RAF uses for NetCDF data files are documented at \url{ http://www.eol.ucar.edu/raf/Software/netCDF.html }.

This document should change as changes in processing algorithms are
implemented, but the intent is also to provide a history of algorithms
that have been used, so there is an effort to document how past archives
were processed along with the descriptions of current algorithms.
Unlike some technical reports, this document is likely to change over
time and should provide a history extending back to \href{http://www.eol.ucar.edu/raf/Bulletins/bulletin9.html}{RAF Bulletin 9}\index{Bulletin 9},
which documented the processing algorithms as they existed in and
before about 2003. 

\index{data!acquisition}Currently, the data acquisition process on
the research aircraft of the Research Aviation Facility, Earth Observing
Laboratory, proceeds as \index{data!processing}follows:\label{DataAcquisitionDescription}
\begin{enumerate}
\item Analog or digital outputs from instruments are sampled at regular
intervals, typically 50 Hz\index{data!sample rate} when possible.
Analog outputs are converted to digital values via analog-to-digital
converters. The investigator's handbooks for each aircraft describe
this process in detail, including resolution of the sampling and handling
of the results. Often, signals from user-supplied instruments\index{instruments!user}
are also included in the measured values that are handled by the data
system.
\item The digital outputs are then recorded by the data system on the aircraft.
Currently, this is a task of the \emph{``}\href{http://www.eol.ucar.edu/data/software/nidas}{NIDAS}''\index{NIDAS}
system described below. That system also controls the sampling, time
stamps, and other aspects of data recording.
\item In flight, the data are processed by the \emph{``nimbus''} data
processing\index{data!processing} program, which makes them available
for display via ``\href{http://www.eol.ucar.edu/raf/Software/aeros_dnld.html}{aeros}''
\index{aeros}for real-time monitoring\index{data!display} of measurements.
\item Following the flight, \emph{nimbus\index{nimbus}} again processes
the data. At this stage, measurements can be re-sampled with averaging
and/or interpolation to produce various data rates, usually 1 Hz or
25 Hz, and known lags in measurements can be introduced to adjust
measurements to a common time basis. As part of this processing, \emph{nimbus}
applies calibration coefficients where appropriate to convert recorded
values (e.g., voltage) to engineering units (e.g., $^{\circ}$C).
Determining or checking these calibration coefficients is part of
the pre-flight and post-flight procedures for each project.
\item The output from \emph{nimbus} is the data file that is the permanent
archive from the experiment, often after merging in additional data
sets from users that are not recorded in the original data file produced
by \emph{NIDAS.} These files, in NetCDF format, have headers that
contain metadata on each measurement (such as the calibration coefficients,
the instrument that produced the measurement, etc.). Many of the variables
in these files are discussed in this technical note, but the files
may also include additional project-specific measurements for which
the NetCDF header and the project reports will be the only documentation.
\end{enumerate}
For assistance accessing data from RAF-supported projects, contact
the RAF data management group via \href{mailto:mailto:raf-dm@eol.ucar.edu}{this email address}.

The data system has changed several times over the history of RAF.
For a discussion of the history of the data systems, see \href{https://drive.google.com/open?id=0B1kIUH45ca5AekJYZmlsOV9sQ28}{this note},
written by Richard Friesen. The versions of data systems that produced
most of the data still available were, approximately, as given in
the following table:

\noindent\begin{minipage}[t]{1\columnwidth}%
\begin{center}
\begin{tabular}{|c|c|c|>{\centering}p{2.3in}|}
\hline 
\textbf{Data System} & \textbf{start} & \textbf{end} & \textbf{Aircraft}\tabularnewline
\hline 
\hline 
ADS I & 1984 & 1992 & King Air 200T, Sabreliner (1987), Electra (1991)\tabularnewline
\hline 
ADS II & 1992 & 2007 & C-130\tabularnewline
\hline 
ADS III (NIDAS)\footnote{ADS III is the name given to the full data system, which includes
these components: NIDAS (for data acquisition and recording); NIMBUS
(for data processing, both in flight and after the flight); AEROS
(for data display in flight); and the Mission Coordinator Station
and satellite communications system (for transmission of data to and
from the aircraft, display of such data for mission decisions, and
support for written ``chat'' communications among project participants
both on the aircraft and on the ground).} & 2005 &  & GV, C-130 (2007)\tabularnewline
\hline 
\end{tabular}
\par\end{center}%
\end{minipage}\\

Before 1993, data were processed by a different program, ``GENPRO,''\index{GENPRO}
and a different output format (also named GENPRO) was used for archived
datasets. Appendix E in \href{http://www.eol.ucar.edu/raf/Bulletins/bulletin9.html}{RAF Bulletin 9}\index{Bulletin 9},
the previous description of RAF data products that is now superseded
by this technical note, describes that format. Some variable names
in this document, esp.~in section \ref{sec:OBSOLETE-VARIABLES},
refer to obsolete variable names, some used with GENPRO and others
referring to instruments that are now retired. These names are included
here so that this report can be a reference for older archived data
as well as for current data files.

\subsection{Alphabetical List of Variables }

At the end of this document, there is a list of all the variable names\index{variable names}
that appear in standard data files along with links to the primary
discussion of those variables. The index to this technical report
also includes all variables described here, and also some variables
not discussed in detail in this document. Where possible, reference
to those variables and information on the project(s) where they were
used have been included also. In cases with multiple references, the
bold entry is the primary discussion of the variable. 

\index{variable names!conventions}In some cases redundant measurements
are present, often for key measurements like pressure or temperature.\index{variable names}
When these are used in subsequent calculation of derived variables
like potential temperature, some choice is usually made regarding
which measurement is considered most reliable for a particular project
or flight, and a single derived variable is calculated on the basis
of the chosen input variable(s). To record which measurements were
so designated, a reference measurement chosen from a group of redundant
measurements usually has a variable name ending with the letter(s)
X or XC.\footnote{Some that do not follow this convention are ATTACK and SSLIP; see
the individual descriptions that follow.} To see the variables in a particular netCDF data file, use the command
``ncdump -h file.nc''.


\subsection{Constants and Symbols}

\label{Punch1.1}The following table contains values used for some
constants in this document. For reference, the symbols used here and
elsewhere in this document are defined in the List of Symbols near
the end of the document (cf.~page \pageref{sec:Symbols}), and links
are provided to where they are used. Where references are to the ``NIST
Chemistry WebBook'', the associated URL is \url{http://webbook.nist.gov}.
References to the CODATA Internationally recommended values of the
Fundamental Physical Constants are available at \url{http://physics.nist.gov/.cuu/.Constants}.
The optimization involved in adjustment of these coefficients is documented
in Mohr et al., 2008a and 2008b, referenced at that URL. \footnote{P. J. Mohr, B. N. Taylor, and D. B. Newell, Rev. Mod. Phys 80(2),
633-730(2008); P. J. Mohr, B. N. Taylor, and D. B. Newell, J. Phys.
Chem. Ref. Data 37(3), 1187-1284(2008).} In this technical note, references to these symbols will often have
these symbols or definitions marked by the symbol $^{\dagger}$ to
indicate that the values used are the standard ones in the following
table.

\vfill\label{constants-table}

\noindent\fbox{\begin{minipage}[t]{1\textwidth - 2\fboxsep - 2\fboxrule}%
\label{ConstantsBox}\centerline{\bf\underbar{Table of Constants}}\index{constants!table}\index{constants|see{symbols}}

$g$ = \label{-constant-g}\sindex[lis]{g@$g$= acceleration of gravity}acceleration
of gravity\footnote{The International Standard Atmosphere specifies $g=9.80665$ ~m\,s$^{-2}$,
$M_{w}=28.9644$ , and $R_{0}$=8.31432\texttimes 10$^{3}$ J kmol$^{-1}$K$^{-1}$,
so these values are used to calculate pressure altitude. } at latitude $\lambda$ \sindex[lis]{lambda@$\lambda$=latitude}and
altitude\sindex[lis]{z@$z$ = height} $z$ above the WGS-84 \index{WGS-84 geoid}geoid,\footnote{cf. Moritz, H., 1988: Geodetic Reference System 1980, Bulletin Geodesique,
Vol. 62 , no. 3, and \href{http://earth-info.nga.mil/GandG/publications/tr8350.2/wgs84fin.pdf}{this link}.} \\
\begin{equation}
g(z,\lambda)=g_{e}\left(\frac{1+g_{1}\sin^{2}(\lambda)}{(1-g_{2}\sin^{2}\lambda)^{1/2}}\right)(1-(k_{1}-k_{2}\sin^{2}(\lambda))z+k_{3}z^{2})\label{eq:g_lambda}
\end{equation}
\hskip2em%
\parbox[t]{0.95\textwidth}{%
where $g_{e}=9.780327$\,m\,s$^{-2}$, $g_{1}=0.00193185$, $g_{2}=0.00669438$,
\\
~~~~~~~~~~$\{k_{1},k_{2},k_{3}\}=\{$3.15704$\times10^{-7}$m$^{-1}$,
2.10269$\times10^{-9}$m$^{-1}$, 7.37452$\times10^{-14}$m$^{-2}$\}%
} \\
$T_{0}$ = temperature in kelvin\sindex[lis]{T0@$T_{0}$=273.15\,K, temperature in kelvin corresponding to $0^{\circ}$C}
corresponding to $0^{\circ}$C = 273.15\,K\\
$T_{3}$ = temperature\sindex[lis]{T3@$T_{3}$= triple point temperature of water}
corresponding to the triple point of water\index{triple point of water}
= 273.16\,K\\
$M_{d}$ = \sindex[lis]{Md@$M_{d}$= molecular weight of dry air}molecular
weight of dry air$^{a}$\index{molecular weight!dry air}, 28.9637
kg\,kmol$^{-1}$~~~\footnote{Jones, F. E., 1978: J. Res. Natl. Bur. Stand., 83(5), 419, as quoted
by Lemmon, E. W., R. T. Jacobsen, S. G. Penoncello, and D. G., Friend,
J. Phys. Chem. Ref. Data, Vol. 29, No. 3, 2000, pp. 331-385. The quoted
values of mole fraction from Jones (1978) and the calculation of mean
molecular weight are tabulated below using values of molecular weights
taken from the NIST Standard Reference Database 69: NIST Chemistry
WebBook as of March 2011. With CO$_{2}$ about 0.00039 and others
decreased proportionately, the mean is 28.9637.
\begin{center}
\begin{tabular}{|c|c|c|c|}
\hline 
Gas & mole fraction $x$ & molecular weight $M$ & $x*M$\tabularnewline
\hline 
\hline 
N$_{2}$ & 0.78102 & 28.01340 & 21.87903\tabularnewline
\hline 
O$_{2}$ & 0.20946 & 31.99880 & 6.70247\tabularnewline
\hline 
Ar & 0.00916 & 39.94800 & 0.36592\tabularnewline
\hline 
CO$_{2}$ & 0.00033 & 44.00950 & 0.01452\tabularnewline
\hline 
Mean: &  &  & 28.96194\tabularnewline
\hline 
\end{tabular}
\par\end{center}}\\
$M_{w}$ = \sindex[lis]{Mw@$M_{w}$= molecular weight of water}molecular
weight of water\index{molecular weight!water}, 18.0153 kg\,kmol$^{-1}$~~~\footnote{NIST Standard Reference Database 69: NIST Chemistry WebBook as of
March 2011}\\
$R_{0}$ = \sindex[lis]{R@$R_{0}$= universal gas constant}universal
gas constant$^{a}$\index{universal gas constant}\index{gas constant!universal}
= 8.314472$\times10^{3}$ J\,kmol$^{-1}$K$^{-1}$~~~\footnote{\label{fn:2006-CODATA}2006 CODATA}\\
$N_{A}$ = Avogadro constant = 6.022141$\times10^{26}$~molecules
kmol$^{-1}$\index{Avogadro constant}\sindex[lis]{NA@$N_{A}$ = Avogadro constant, molecules per kmol}\\
$k=R_{0}/(\mathrm{N_{A})}=1.38065\times10^{-23}\mathrm{J}\,\mathrm{K}^{-1}$\sindex[lis]{k@$k$=Boltzmann Constant}\index{Boltzmann constant}\\
$R_{d}=(R_{0}/M_{d}$) = \sindex[lis]{Rd@$R_{d}=$gas constant for dry air}gas
constant for dry air\index{gas constant!dry air}\\
$R_{w}$ = ($R_{0}/M_{w})$ = \sindex[lis]{Rw@$R_{W}$= gas constant for water vapor}gas
constant for water vapor\index{gas constant!water vapor}\\
$R_{E}$ = \sindex[lis]{Re@$R_{E}$= radius of the Earth}radius of
the Earth\index{radius of the Earth} = 6.371229$\times$10$^{6}$
m ~~~\footnote{matching the value used by the inertial reference systems discussed
in \prettyref{sec:INS}} \\
$c_{p}$ = \sindex[lis]{cp@$c_{p}$ or $c_{pd}$ = specific heat of dry air at constant pressure}specific
heat of dry air at constant pressure\index{specific heat!dry air!constant pressure}
= $\frac{7}{2}R_{d}=$1.00473$\times10^{3}$ J$\,$kg$^{-1}$K$^{-1}$~~~\footnote{The specific heat of dry air at 1013 hPa and 250--280 K as given
by Lemmon et al. (2000) is 29.13 J/(mol-K), which translates to 1005.8$\pm0.3$
J/(kg-K). However, the uncertainty limit associated with values of
specific heat is quoted as 1\%, and the experimental data cited in
that paper show scatter that is at least comparable to several tenths
percent, so the ideal-gas value cited here is well within the range
of uncertainty. For this reason, and because this value is in widespread
use, the ideal-gas value is used throughout the algorithms described
here.}\\
$c_{v}$ = \sindex[lis]{cv@$c_{v}$ or $c_{vd}$ = specific heat of dry air at constant volume}specific
heat of dry air at constant volume\index{specific heat!dry air!constant volume}
= $\frac{5}{2}R_{d}=$0.71766$\times10^{3}$ J$\,$kg$^{-1}$K$^{-1}$\\
\noindent\parbox[t]{1\textwidth}{%
\hskip0.6cm(specific heat values are at 0$^{\circ}$C; small variations
with temperature are not included here)%
}\\
$\gamma$ = \sindex[lis]{gamma or gamma_{d} = ratio of specific heats of air, c_{p}/c_{v}@$\gamma$ or $\gamma_{d}$ = ratio of specific heats of air, $c_{p}/c_{v}$}ratio
of specific heats\index{specific heat ratio, dry air}, $c_{p}/c_{v}$,
taken to be 1.4 (dimensionless) for dry air\\
$\Omega$ = \sindex[lis]{Omega= angular rotation rate of the Earth@$\Omega$= angular rotation rate of the Earth}angular
rotation rate of the Earth\index{Earth, angular rotation rate} =
7.292115$\times10^{-5}$ radians/s\\
$\Omega_{Sch}$ = \sindex[lis]{Omega_{Sch}= angular frequency of the Schuler oscillation@$\Omega_{Sch}$= angular frequency of the Schuler oscillation}angular
frequency of the Schuler oscillation\index{Schuler oscillation} =
$\sqrt{\frac{g}{R_{E}}}$\\
$\sigma$ = \sindex[lis]{sigma= Stephan-Boltzmann constant@$\sigma$= Stephan-Boltzmann constant}Stephan-Boltzmann
Constant\index{Stephan-Boltzmann Constant} = 5.6704$\times10^{-8}$W\,m$^{-2}\mathrm{K}^{-4}$~~$^{e}$%
\end{minipage}}

\newpage{}

\section[GENERAL INFORMATION]{GENERAL INFORMATION ABOUT DATA FILES}

\subsection{System of Units\label{subsec:System-of-Units}}

This report uses the SI system of units\index{system of units}\index{units},
but with many exceptions. Among them are the following:
\begin{enumerate}
\item The millibar\index{millibar} (mb), equal to one hectopascal (hPa),
was used for pressure with some older variables.
\item Many variables are presented in the units most often used for that
variable, even when they involve CGS units or mixed CGS-MKS units\index{units, mixed CGS-MKS},
as for example {[}$\mathrm{g\,m}{}^{-3}]$ for liquid water content
or {[}$\mathrm{cm}{}^{-3}${]} for droplet concentration\index{concentration!droplet}.\index{units!exceptions to SI} 
\item Flow rates\index{flow rates} are often quoted in liters per minute
(LPM) or standard liters per minute (SLPM) because those terms are
linked to properties of commercially available instruments with flow
control. One liter is $10^{-3}\,\mathrm{m}^{3}$. Standard temperature
and pressure\index{STP} are respectively 273.15~K~and 1013.25~hPa.
However, there is considerable ambiguity in the definition of ``standard''
conditions (mostly regarding the choice of the reference temperature)
because some flow controllers and flowmeters specify a different ``standard''
temperature, so the particular usage will be documented when this
term is used. Mass flow meters\index{meter!mass flow} provide a measure
of the flow of mass but usually report the measurement in terms of
the volume flow that would be present under standard conditions\index{flow!SLPM}
(i.e., SLPM). Therefore, to convert to volumetric flow\index{flow!volumetric}
at other conditions, if the fluid density is $\rho$ and the mass
flow rate in units of mass per time is denoted by $\dot{m}^{\prime}$,
the volumetric flow $Q$ is $\dot{m}^{\prime}/\rho$. Then the mass
flow rate in units of standard volume per time is $\dot{m}=\dot{m}^{\prime}/\rho_{s}$
where $\rho_{s}$ is the density of the fluid under standard conditions.
To convert to volumetric flow under other conditions, $Q=\dot{m}^{\prime}/\rho=\dot{m}\rho_{s}/\rho$=$\dot{m}p_{s}T/(p\thinspace T_{s})$
where $p$ and $T$ are the pressure and absolute temperature for
the desired measurement and $p_{s}$ and $T_{s}$ are the corresponding
values for standard conditions.\index{flow!conversions}
\item \label{enu:ppmv}The International Bureau of Weights and Measures
recommends against use of units like percent or parts per million,
but these are in common use in atmospheric chemistry and elsewhere
so data files continue to use those units for relative humidity or
the concentration\index{concentration!chemical species} of chemical
species. Although a proper SI unit for a volumetric mixing ratio would
be $\mu$mol\,mol$^{-1}$, nmol\,mol$^{-1}$, or pmol\,mol$^{-1}$,
variables are instead often assigned the respective units of \index{units!ppmv}\index{units!ppmb}ppmv\index{ppmv},
ppbv\index{ppbv} or pptv\index{pptv} for parts per million, billion
or trillion by volume. Care must be taken to interpret ppbv especially,
because ``billion'' has different meaning in different languages
and different countries; herein, 1 ppbv means a volumetric ratio of
1:10$^{9}$. Many measurements produce native results in terms of
a mass ratio, often described as a mixing ratio $r_{m}$ in terms
of mass of the measured gas per unit mass of ``air'' (where the
mass of the ``air'' does not include the variable constituents,
usually only significant for water vapor). The perfect gas law relates
the density ratio of two gases ($\rho_{1}:\rho_{2})$ to the ratio
of their partial pressures ($p_{1}:p_{2}$) or number densities ($n_{1}:n_{2}$)\sindex[lis]{n@$n$=number density},
as follows:\\
\begin{equation}
r_{m}=\frac{\rho_{1}}{\rho_{2}}=\frac{p_{1}M_{1}}{p_{2}M_{2}}=\frac{n_{1}M_{1}}{n_{2}M_{2}}\label{eq:rm}
\end{equation}
where $M_{1}$ and $M_{2}$ are respective molecular weights for the
two gases.\sindex[lis]{rm@$r_{m}$=mixing ratio by mass}\index{mixing ratio!conversion}
The ratio of number densities or, equivalently, partial pressures,
denoted here as $r_{v}$ because it is also the volumetric mixing
ratio\sindex[lis]{rv@$r_{v}$=mixing ratio by volume}, is related
to the mass mixing ratio as follows:\\
\begin{equation}
r_{v}=\frac{n_{1}}{n_{2}}=\left(\frac{M_{2}}{M_{1}}\right)r_{m}\label{eq:rv}
\end{equation}
\\
When concentrations\index{concentration!chemical species} are recorded
with units of ``ppmv'', ``ppbv'' or ``pptv'', these units\index{ppmv}\index{ppbv}\index{pptv}
refer respectively to $10^{6}r_{v}$, $10^{9}r_{v}$, or $10^{12}r_{v}$
with $r_{v}$ given by the above equation.\\
\item The unit ``hertz''\index{units!hertz} (abbreviation Hz) is the
proper unit for a periodic sampling frequency and will be used here
in place of the more awkward ``samples per second.''\index{samples per second}
This usage is favored by the International Bureau of Weights and Measures
(cf.~\url{http://www.bipm.org/en/si/si_brochure/chapter2/2-2/table3.html#notes})
when the frequency represented refers to the rate of sampling. 
\item In some cases, particularly for older data files, speed has been recorded
in units of knot\index{knot}s (= 0.514444 m/s) and distance in nautical
mile\index{nautical mile}s $\equiv$ 1852 m).
\end{enumerate}
Near the end of this technical note, there is a list of symbols.\footnote{ Some symbols used only once and defined where they are used are omitted
from this list} The next table defines some abbreviations and additional symbols
used for units in this report, in addition to the standard abbreviations
for the mks system of units:\index{abbreviations!non-standard}\label{punch1.2}

\begin{center}
\noindent\begin{minipage}[t]{1\columnwidth}%
\begin{center}
\begin{tabular}{|c|l|}
\hline 
\textbf{abbreviation/symbol} & \textbf{definition}\footnote{where the symbol $\equiv$ is used, the relationship is exact by definition}\tabularnewline
\hline 
\hline 
� & degree, angle measurement $\equiv$ ($\pi$/180) radian\tabularnewline
\hline 
ft & foot\index{foot} $\equiv$ 0.3048 m\tabularnewline
\hline 
mb & millibar $\equiv$ 100 Pa $\equiv$ 1 hPa\tabularnewline
\hline 
ppmv & parts per million by volume (see subsection \ref{subsec:System-of-Units}
item \ref{enu:ppmv})\tabularnewline
\hline 
ppbv & parts per billion ($10^{9})$ by volume (see subsection \ref{subsec:System-of-Units}
item\ref{enu:ppmv})\tabularnewline
\hline 
pptv & parts per trillion ($10^{12})$ by volume (see subsection \ref{subsec:System-of-Units}
item\ref{enu:ppmv})\tabularnewline
\hline 
n~mi & nautical mile $\equiv$1852 m\tabularnewline
\hline 
kt & knot (n~mi/hour) $\equiv$ (1852/3600) m/s = 0.514444... m/s\tabularnewline
\hline 
\end{tabular}
\par\end{center}%
\end{minipage}
\par\end{center}

\subsection{Variables Used To Denote Time\index{time!variables}}

Although there are some exceptions in old archived data files, the
data in all modern output files are referenced to Coordinated Universal
Time (UTC). The time and date of the data acquisition system are synchronized
to time from the Global Positioning System (GPS) at the beginning
of each flight, and for data acquired by the present ADS-3 (NIDAS)\index{NIDAS}
data acquisition system time is synchronized continuously with the
GPS time. Time variables vary for older archived data files; some
of the following are obsolete, but are included here for reference
because they are important to those wanting to use those archives.\\

\begin{hangparagraphs}
\textbf{Time (s): }\textbf{\uline{Time}}\sindex[var]{Time}\index{time!variable Time}\label{han:Time-(s):-TimeThis-1}\nop{Time}{}\\
\emph{The reference-time }counter \emph{for the output data files,}
used by data system versions beginning with ADS-3. It is an integer
output at 1 Hz \emph{and has an initial value of zero at the start
of the flight}. Add this to the ``Time:units'' attribute found in
the NETCDF header section to obtain the UTC time.\index{time!example of usage}

\begin{minipage}[t]{0.9\textwidth}%
Example attribute: \texttt{\small{}}~\\
\texttt{\hspace*{1cm}}\texttt{\small{}Time:units = ``seconds since
2006-04-26 12:55:00 +0000'' };

For code examples that show how to use ``Time'' see:\\
\texttt{\hspace*{1cm}}\url{http://www.eol.ucar.edu/raf/Software/TimeExamp.html}%
\end{minipage}

~

\textbf{Reference Start Time (s): }\textbf{\uline{base\_time}}\index{base time}\index{time!base_time@base\_time}
(\emph{Obsolete; versions before ADS-3 only})\\
\emph{The reference time for the netCDF output data files for data
system versions before ADS-3. }It represents the time of the first
data record. Its format is Unix time (elapsed seconds after midnight
1 January 1970). Add time\_offset (below) to obtain the time for each
data record. (Note: base\_time is a single scalar, not a ``record''
variable, so it occurs just once in the output file.) 

\textbf{Time Offset from Reference Start Time (s): }\textbf{\uline{time\_offset}}\index{time!offset}
(\emph{Obsolete)} \\
\emph{The time offset from base\_time of each data record used for
the NETCDF output files produced by data system versions before ADS-3.}
It starts at zero (0) and increments each second, so it can also be
thought of as a record counter. Use this measurement and add base\_time
to obtain the time for each data record.

\textbf{Raw Tape Time (hour, minute, s): }\textbf{\uline{HOUR}}\textbf{\index{HOUR},
}\textbf{\uline{MINUTE}}\textbf{\index{MINUTE}, }\textbf{\uline{SECOND}}\textbf{\index{SECOND}
}(\emph{Obsolete})\\
These three time variables are recorded directly from the aircraft's
data system. Since ADS-3, this information is replaced by the ``Time''
variable and the ``Time:units'' attribute of that variable.

\textbf{Date (m, d, y): }\textbf{\uline{MONTH}}\textbf{\index{MONTH},
}\textbf{\uline{DAY}}\textbf{\index{DAY}, }\textbf{\uline{YEAR}}\index{YEAR}
(\emph{Obsolete})\\
These three variables represent the date when the aircraft's data
system began recording data. They are repeated as 1 Hz variables but
are NOT incremented if the time rolls over to the next day. Use base\_time
and time\_offset for reference timing. Since ADS-3, this information
is replaced by the ``Time'' variable and the ``Time:units'' attribute
of that variable.
\end{hangparagraphs}


\subsection{Synchronization of Measurements\label{subsec:Synchronization-of-Measurements}}

Measurements sampled under control of the ``NIDAS''\index{NIDAS}
sampling system\index{synchronization} are acquired at 50 Hz.\index{sampling rates}
However, the standard archive files are produced at a rate of 1 Hz,
and each sample is the average of 50 samples. Therefore, the time
associated with measurements reported at 1~Hz is actually an average
over the specified second, so the reference time for the averaged
measurement is actually 0.5~s past the reported time.\index{offset@sample time}
Analogous offsets apply to variables reported at other rates different
from 50~Hz. Where it applies, electronic filters\index{filter!electronic}
with cutoff frequency of 25 Hz are used with analog measurements.
Higher-rate files are sometimes produced, standardized to 25 Hz but
sometimes at other frequencies. 

There are time shifts\index{time!shifts} inherent in many of the
measurements, and in some cases (e.g., those produced by inertial
reference units) these time shifts arise because the information is
transmitted from the measuring system at a time later than when it
was sampled. In these cases, shifts (``lags'') are applied to the
measurements. The lags\index{lags in sampling} may be either static
or dynamic. Static lags\index{lags!static} are specified in a configuration
file, saved for each project; dynamic lags\index{lagsdynamic} provided
as part of data sampling by specific instruments are recorded by NIDAS
for use in processing. Dynamic lags are usually a difference in time
from a gridded time value to the time it was actually acquired. e.g.
for a 5hz parameter the expected or gridded millisecond offset into
each second would be 0, 200, 400, 600, and 800. If the data actually
were sampled or acquired at 50, 250, 450, 650, and 850 ms then the
dynamic lag for this particular second would be -50 ms. Corrections
for time lags\index{time!lags} are applied to measurements before
conversion to one of the standard data rates. 

Where data rates\index{data rates} for particular measurements do
not match the basic 50 Hz sampling rate, linear interpolation\index{interpolation}\index{time!interpolation}
is used to obtain higher-rate values. For 1 Hz data files, measurements
are then averaged within each second. For 25 Hz files, 50 Hz measurements
are digitally filtered using a finite impulse response (FIR) filter,\index{filter!FIR}
while data acquired at less than 25 Hz are linearly interpolated to
25 Hz and then FIR-filtered for smoothing.
\begin{hangparagraphs}
\label{punch1.3}
\end{hangparagraphs}


\subsection{Other Comments On Terminology}

\subsubsection{Variable Names In Equations}

This report often uses variable names\index{nomenclature!equations}
in equations, and sometimes there is potential for confusion because
the variable names consist of multiple characters. In most cases,
to denote that the variable name is the variable in the equation (as
opposed to each of the letters in the variable name representing quantities
to be multiplied together), the variable name has been enclosed in
brackets, as in \{TASX\}.\index{variable!names in brackets} In addition,
variable names are displayed with upright Roman character sets, while
other symbols in equations are shown using slanted (script) character
sets as is conventional for mathematical equations. In cases where
code\index{nomenclature!code} segments (usually expressed in C code)
are included to document how calculations are performed, typewriter
character sets indicate that the segment is a representation of how
the processing could be coded. Such a code segment is not always a
direct copy of the code in use, but such code is sometimes the most
convenient way to express the algorithm in use.

\subsubsection{Distinction Between Original Measurements and Derived Variables}

Many of the variables in the data files and in this report are derived
from combinations of measurements. The terms ``raw'' or ``original''
measurement\index{measurement!raw}\index{measurement!original}\index{measurement!derived}
are sometimes used for a minimally processed output received directly
from a sensor or instrument. Such measurements may be converted to
engineering units via calibration\index{calibration!coefficients}
coefficients, but otherwise they are a direct representation of the
output from a sensor.\footnote{Calibration coefficients,\index{calibration!coefficients} e.g. those
used to convert from voltage output from an analog sensor to a measured
quantity with physical units like $\text{�}$C), are not included
or discussed in this report. They are normally included in project
reports and, in recent years, many are included in the header of the
NETCDF file.} In contrast, derived variables\index{variable!derived} (e.g., potential
temperature) depend on one or more ``raw'' measurements and are
not direct results of output from an instrument. For most derived
measurements, a box that follows an introductory comment is used in
this report to document the processing algorithm. The box has a line
dividing top from bottom; in the top are definitions used and explanations
regarding variables that enter the calculation, while the bottom portion
contains the equation, algorithm, or code segment that documents how
the variable is calculated.\label{punch1.4}

\subsubsection{Dimensions in Equations}

An effort has been made to avoid dimensions\index{equations!dimensionless}\index{dimensions in equations}
in equations\index{nomenclature!equations} except where it would
be awkward otherwise. \index{equations!scale factors}Some scale factors
are introduced for only this purpose (e.g., to avoid dimensions in
arguments to logarithmic or exponential functions), and some effort
was made to isolate dimensions to defined constants rather than requiring
that variables in equations be used with specific units. However,
some exceptions remain to be consistent with historical usage.


\include{Section3}

\newpage\vfill\eject\cleardoublepage{}


\section{THE STATE OF THE ATMOSPHERE\label{sec:State Variables}}

\subsection{Information on Instruments and Calibrations}

The instruments used to collect the measurements that lead to the
variables in this section are described on the EOL web site, in the
``State Parameters'' section at \href{http://www.eol.ucar.edu/aircraft-instrumentation}{this URL}.\index{instrument descriptions!web site}
The data acquisition and processing for these variables and the calibration
coefficients\index{coefficients!calibration}\index{calibration}
used where applicable are described \vpageref{DataAcquisitionDescription}.

\subsection{Variable Names\index{variable names}}

Measurements of some meteorological state variables like pressure,
temperature, and water vapor pressure may originate from multiple
sensors mounted at various locations on an aircraft. To distinguish
among similar measurements, many variable names\index{names!variable}
incorporate an indication of where the measurement was made. In this
document, locations in variable names\index{names!variable!location in}
are represented by ``x'', where ``x'' may be one of the following:

\begin{center}
\begin{tabular}{|c|c|}
\hline 
Character & Location\tabularnewline
\hline 
\hline 
B & bottom (or bottom-most)\tabularnewline
\hline 
B & (obsolete) boom\tabularnewline
\hline 
F & fuselage\tabularnewline
\hline 
G & (obsolete) gust probe\tabularnewline
\hline 
R & radome\tabularnewline
\hline 
T & top (or top-most)\tabularnewline
\hline 
W & wing\tabularnewline
\hline 
\end{tabular} 
\par\end{center}

In addition, a true letter 'X' (not replaced by the above letters)
may be appended to a measurement to indicate that it is the preferred
choice\index{variable!preferred choice} among similar measurements
and is therefore used to calculate derived variables that depend on
the measured quantity. Other suffixes\index{names!variable!suffixes}
sometimes used to distinguish among measurements are these: 'D' for
a digital sensor; 'H' for a heated (usually, anti-iced) sensor, 'L'
for port-side sensors, and 'R' for starboard-side sensors.

\subsection{\label{subsec:PTq}Pressure\index{pressure}}
\begin{hangparagraphs}
\textbf{Static Pressure (hPa): }\textbf{\uline{PSx}}\textbf{,\sindex[var]{PSx@\textbf{PSx}}\index{PSx}
}\textbf{\uline{PSxC}}\sindex[var]{PSxC}\index{PSxC}, \textbf{\uline{PS\_A}}\sindex[var]{PS_A@PS\_A}\index{PS_A@PS\_A}\textbf{,
}\textbf{\uline{PSXC}}\index{PSXC}\sindex[var]{PSXC}\textbf{.
}\textbf{\uline{PSFD}}\textbf{\sindex[var]{PSFD}}\index{PSFD}\textbf{,
}\textbf{\uline{PSFRD}}\textbf{\sindex[var]{PSFRD}}\index{PSFRD}\\
\emph{The atmospheric pressure at the flight level of the aircraft,
measured by a calibrated absolute (barometric) transducer at location
x.} PSx is the measured static or ambient pressure\index{pressure!ambient}\index{transducer!barometric}
before correction, and it may be affected by local flow-field distortion\index{flow distortion}.
PS\_A is the pressure measurement taken from the avionics\index{avionics}
system on the aircraft, processed via unknown algorithms in the avionics
system that may smooth, correct, and perhaps delay the result. PSxC
is PSx corrected for local flow-field distortion. (See \href{http://www.eol.ucar.edu/raf/Bulletins/bulletin21.html}{RAF Bulletin \#{}21}\index{Bulletin 21}
and the discussion in \href{https://drive.google.com/open?id=0B1kIUH45ca5AaW02ZUt1X2kyX2s}{this memo}),
and PSXC is the preferred\index{measurement!preferred} corrected
measurement used for derived calculations. These measurements have
been made using various sensors, so it is best to consult the project
documentation for the transducer used. Recent measurements from both
the C-130 and the GV have been made using a Paroscientific\index{transducer!Paroscientific}
Model 1000 Digiquartz Transducer. \\
\\
Corrections to the pressures\index{PCORS|see {pressure corrections}}\index{pressure!corrections}
have been determined by reference to some standard, including a ``trailing
cone''\index{trailing cone} sensor, the pressure PS\_A from the
cockpit avionics system, or (since 2012) the Laser Air Motion Sensing
System\index{LAMS} (LAMS). The latter correction is discussed in
the memo \href{https://drive.google.com/open?id=0B1kIUH45ca5AaW02ZUt1X2kyX2s}{referenced above},
where corrections used prior to 2011 are also discussed. Beginning
in 2012, the deduced corrections $\Delta p$ \index{correction!pressure}\sindex[lis]{Deltap@$\Delta p$=correction to pressure}to
the measured pressures\index{defect!static} as functions of dynamic
pressure\index{pressure!dynamic}\sindex[lis]{q@$q$= dynamic pressure}
$q,$ angle of attack $\alpha,$\footnote{A weakness is this form for the pressure correction is that occasionally
the radome ports become plugged with ice and the measurement of angle
of attack is not available. When the variable ATTACK representing
angle of attack is invalid, the angle of attack is instead calculated
from PITCH$-$VSPD/TASX, which approximates the angle of attack if
the vertical wind is zero.} and the Mach number $M$ are described by the following equations
and coefficients:\\
\\
\begin{minipage}[t]{1\columnwidth}%
\begin{quote}
\textbf{For the C-130,}\footnote{For C-130 measurements prior to 2012 but after September 2003, the
correction applied to PSF was $\Delta p=p+\max((3.29+\{\mathrm{QCX}\}*0.0273),$4.7915)
using units of hPa. Prior to Sept 2003, the correction was $\Delta p=\max((4.66+11.4405\Delta p_{\alpha}/\Delta q_{r}$),
1.113). For both PSFD and PSFRD, the correction prior to (2012?) was
$\Delta p=p+\max((3.29+\{\mathrm{QCX}\}*0.0273),$4.7915). For GV
measurements Aug 2006 to 2012, $\Delta p=$ (-1.02 + 0.1565{*}q) +
q1{*}(0.008 + q1{*}(7.1979e-09{*}q1 - 1.4072e-05). Before Aug 2006:
$\Delta p=$(3.08 - 0.0894{*}\{PSF\}) + \{QCF\}{*}(-0.007474 + \{QCF\}{*}4.0161e-06).}\\
\begin{equation}
\frac{\Delta p}{p}=d_{0}+d_{1}\frac{\alpha}{a_{r}}+d_{2}\,M\label{eq:PCORC130}
\end{equation}
where, for $p$ = PSFD \sindex[lis]{alpha@$\alpha$=angle of attack},
$\alpha=\mathrm{ATTACK}$ and $a_{r}=1^{\circ}$ (included to keep
the equation and coefficients\index{coefficients!sensitivity} dimensionless),
\{$d_{0},d_{1},d_{2}$\}=\{$-$0.00637, 0.001366, 0.0149\}. For PSFRD,
the coefficients are \{$d_{0}^{\prime},\,d_{1}^{\prime},\,d_{2}^{\prime}$\}\sindex[lis]{di@$d_{0-2}$=coefficients, pressure correction, C-130}
= \{$-$0.00754, 0.000497, 0.0368\}. The latter coefficients are
significantly different from the coefficients for PSFD, but the static
ports where PSFRD is measured are at a different location on the fuselage
so different flow-distortion effects are expected.
\end{quote}
%
\end{minipage}\\
\\
\begin{minipage}[t]{1\columnwidth}%
\begin{quote}
\textbf{For the GV,}\footnote{See \href{https://drive.google.com/a/ucar.edu/file/d/0B1kIUH45ca5AWlFWYXBDRlI1VnM}{this memo}
for details regarding implementation of this representation of $\Delta p$
for the GV: }\\
\begin{equation}
\frac{\Delta p}{p}=a_{0}+a_{1}\frac{q}{p}+a_{2}M^{3}+a_{3}\frac{\alpha}{a_{r}}\label{eq:PCORforGV}
\end{equation}
where, for $p$ = PSF, $q$ = QCF, $\alpha=\mathrm{ATTACK}$, and
$a_{r}=1^{\circ}$ (included to keep the equation and coefficients
dimensionless) \{$a_{0},a_{1},a_{2},a_{3}$\}\sindex[lis]{ai@$a_{0,1,2,3}$=coefficients, pressure correction, GV}
= \{$-0.012255$, $0.075372$, $-0.087508$, $0.002148$\}\}.\label{punch:4-10}
\end{quote}
%
\end{minipage}\\
\\
In equations (\ref{eq:PCORC130}) and (\ref{eq:PCORforGV}) the Mach
number\index{Mach number!uncorrected} is calculated from the uncorrected
measurements of $p$ and $q$, via\\
\begin{equation}
M=\left\{ \left(\frac{2c_{v}}{R}\right)\left[\left(\frac{p+q}{p}\right)^{R/c_{p}}-1\right]\right\} ^{1/2}\,\,\,.\label{eq:MachEquation}
\end{equation}
\\

\textbf{Dynamic Pressure (hPa): QCx, QCxC, QCXC}\index{QCXC}\index{QCx}\index{QCxC}\sindex[var]{QCx}\sindex[var]{QCxC}\sindex[var]{QCXC}\\
\emph{The pressure excess caused by bringing the airflow to rest relative
to the aircraft.} \index{pressure!dynamic}These quantities represent
the difference between the total pressure $p_{t}$\sindex[lis]{pt@$p_{t}$=total pressure}
as measured at the inlet of a pitot tube or other forward-pointing
port and the ambient pressure that would be present in the absence
of motion through the air.\emph{ }The variables ending in ``C''
have been corrected\index{pressure!dynamic!corrected} for flow-distortion
effects, mostly arising from errors in the measurement of static pressure.
Since 2012, the corrections are based on measurements from the LAMS\index{LAMS}
system as described for PSxC, and they have the same functional form
as in (\ref{eq:PCORC130}) and (\ref{eq:PCORforGV}) except that the
correction\index{correction!dynamic pressure} applied to $q$ is
$-\Delta p$ with reversed sign because $q=p_{t}-p_{a}$ and the error
arises primarily from the error in $p_{a}$. The same correction is
applied to QCR\index{QCR!correction} because it is also measured
relative to the static pressure ports so errors in the pressure sensed
at those ports affect QCR in the same way that QCF is affected. \label{punch4.3}See
the notes referenced in the preceding section, and also \href{http://www.eol.ucar.edu/raf/Bulletins/bulletin21.html}{RAF Bulletin 21}\index{Bulletin 21}
for the corrections applied to earlier data files.\footnote{\textbf{}%
\begin{minipage}[t]{1\columnwidth}%
\textbf{C-130}, prior to 2012: 
\begin{itemize}
\item For QCFC: subtract $\max((4.66+11.4405*\mathrm{{ADIFR\}/\{QCR}}$),
1.113
\item For QCFRC prior to Sept 2003: same as for QCFC
\item ~~~~~~~~~after/including Sept 2003, subtract $\max((3.29+\{\mathrm{QCX}\}*0.0273),$4.7915)
\item For QCRC: subtract $\max((3.29+\{\mathrm{QCX}\}*0.0273),$4.7915)
\end{itemize}
\textbf{GV, }Aug 2006 to 2012:
\begin{itemize}
\item For QCF, subtract (1.02+\{PSF\}{*}(0.215 - 0.04{*}\{QCF\}/1000.) +
\{QCF\}{*}(-0.003266 + \{QCF\}{*}1.613e-06))\label{punch4.2}
\end{itemize}
%
\end{minipage}}\index{pressure!dynamic}\index{pressure!total}\index{pressure!pitot|see {pressure, total}}
\index{pressure!ambient}\index{pressure!static|see {pressure ambient}}\index{Bulletin 21}
A Rosemount Model 1221 differential pressure transducer\index{pressure!transducer}
is used for current measurements of dynamic pressure on the C-130,
and a Honeywell PPT transducer is used on the GV.\label{punch4.1}
This measurement enters the calculation of true airspeed\index{true airspeed|see {airspeed}}\index{airspeed}
and Mach number\index{Mach number} and so is needed to calculate
many derived variables. 

\textbf{D-Value (m): DVALUE\index{DVALUE}}\sindex[var]{DVALUE}\\
The difference\index{d-value} between geopotential altitude\index{altitude!geopotential}
and pressure altitude\index{altitude!pressure} (m). This variable
is calculated from \{GGALT\}$-$\{PALT\} and, for appropriate flight
segments, can be used to measure height gradients on a constant-pressure
surface.

\textbf{Special Pressure Measurements (hPa): }\textbf{\uline{PSDPx}}\textbf{,
}\textbf{\uline{CAVP\_x}}\textbf{, }\textbf{\uline{PCAB}}\textbf{,
}\textbf{\uline{PSURF}}\index{PCAB}\index{PSDPx}\index{CAVP_x@CAVP\_x}\index{PSURF}\sindex[var]{PSDPx}\sindex[var]{CAVP_x@CAVP\_x}\sindex[var]{PCAB}\sindex[var]{PSURF}\index{pressure!cabin}\index{pressure!dew-point cavity}\index{pressure!surface}\\
\emph{PSDPx and CAVP\_x are measurements of the pressure in the housing
of the dew-point sensors, }as discussed in connection with DPxC.\emph{
PCAB is a measurement of the pressure in the cabin of the aircraft.
PSURF is the estimated surface pressure }calculated from HGME\index{HGME}
(a radar-altimeter measurement of height), TVIR\index{TVIR}, PSXC\index{PSXC}\label{punch:4-11},
and MR\index{MR} using the thickness equation\index{equation!thickness}.
TVIR and MR are described later in this section (cf.~pages \pageref{TVIR}
and \pageref{Mixing-Ratio-(g/kg):}, respectively), and HGME was described
on page \pageref{HGME} in Section \ref{sec:INS}. The average temperature
for the layer is obtained by using HGME and assuming a dry-adiabatic
lapse rate from the flight level to the surface. Because of this assumption,
the result is only valid for flight in a well-mixed surface layer
or in other conditions in which the temperature lapse rate matches
the dry-adiabatic lapse rate.\footnote{The symbol $^{\dagger}$ indicates that values are included in the
table of constants, p.~\pageref{ConstantsBox}.}\\
\fbox{\begin{minipage}[t]{0.95\columnwidth}%
PSXC\index{PSXC} = ambient pressure (hPa)\\
HGME\index{HGME} = (radar) altitude above the surface (m)\\
TVIR\index{TVIR} = virtual temperature ($^{\circ}\mathrm{C}$)\\
PSURF\index{PSURF} = estimated surface pressure (hPa)\\
$g$ = acceleration of gravity$^{\dagger}$\\
$R_{d}$ = gas constant for dry air$^{\dagger}$\\
$c_{pd}$ = specific heat of dry air at constant pressure$^{\dagger}$\\
\\
\rule[0.5ex]{1\columnwidth}{1pt}

\[
T_{m}=(\mathrm{\{TVIR\}}+T_{0})+0.5\mathrm{\{HGM\}}\frac{g}{c_{pd}}
\]
\begin{equation}
\mathrm{PSURF}=\mathrm{\{PSXC\}}\,\exp\left\{ \frac{g\,\{\mathrm{HGM}\}}{R_{d}T_{m}}\right\} \label{eq:PSURF-1-1}
\end{equation}
%
\end{minipage}}
\end{hangparagraphs}


\subsection{Temperature\index{temperature}}
\begin{hangparagraphs}
\textbf{\label{RTx discussion}Recovery Temperature ($\text{�}$C):
}\textbf{\uline{RTx}}\textbf{\sindex[var]{RTx}\index{RTx}, }\textbf{\uline{RTxH}}\sindex[var]{RTxH}\index{RTxH}\textbf{,
RTHRx\sindex[var]{RTHRx}}\index{RTHRx}\\
\emph{The recovery temperature\index{temperature!recovery} is the
temperature sensed by a temperature probe that is exposed to the atmosphere.}
In flight, the temperature is heated above the ambient temperature\index{temperature!ambient}
because it senses the temperature of air near the sensor that has
been heated adiabatically during compression as it is brought near
the airspeed of the aircraft. These variables are the measurements
of that recovery temperature from calibrated temperature sensors at
location x.\index{TTx}\sindex[var]{TTx}\footnote{Prior to 2012, these variables were called ``total temperature''
and symbols starting with 'TT' instead of 'RT' were used. That name
was misleading because these values are not true total-temperature\index{temperature!total}
measurements, for which the air would be at the same speed as the
aircraft, but instead recovery-temperature measurements. The name
has been changed to correct this mis-labeling, although this was a
long-standing convention in past datasets.} For Rosemount temperature probes, the recovery temperature\index{temperature!recovery vs.~total}
is near the total temperature, but all probes must be corrected to
obtain either true total temperature or true ambient temperature.
In the standard output, the variable name also conveys the sensor
type: RTx is a measurement from a Rosemount Model 102 non-deiced temperature
sensor,\index{sensor!temperature} RTxH is the measurement from a
Rosemount Model 102 anti-iced (heated) temperature\index{sensor!temperature!anti-iced}
sensor, and RTHRx is the measurement from a HARCO heated sensor.\index{sensor!temperature!heated}
Some past experiments also used a reverse-flow temperature\index{temperature!reverse-flow}
housing and a fast-response ``K'' housing;\index{sensor!temperature!K-probe}
the associated variable names for these probes were TTRF\index{TTRF}\sindex[var]{TTRF}
and TTKP.\index{TTKP}\sindex[var]{TTKP}\footnote{See the related obsolete variables\index{variables!obsolete!TTx}
TTx,\sindex[var]{TTx} which are previously used names for these variables.
The names were changed to clarify that the quantity represented is
the recovery temperature, not the total temperature.}

\textbf{Ambient Temperature ($\lyxmathsym{�}$C): }\textbf{\uline{ATx}}\textbf{\sindex[var]{ATx}\index{ATx},
}\textbf{\uline{ATxH}}\sindex[var]{ATxH}\index{ATxH}\textbf{,
}\textbf{\uline{ATxD}}\textbf{\sindex[var]{ATxD}}\index{ATxD}\\
\emph{The temperature of the atmosphere at the location of the aircraft,
as it would be measured by a sensor at rest relative to the air. }The
'x' in the name of the variable used for ambient temperature, ATx,
conveys the same information regarding sensor type and location as
the variable name used with total (recovery) temperature. See the
discussion above regarding RTx. The ambient temperature\index{temperature!ambient}
(also known as the static air temperature\index{temperature!static air})
is calculated from the measured recovery temperature, which is increased
above the ambient temperature by dynamic heating caused by the airspeed
of the aircraft. The calculated temperature\index{temperature!calculation}
therefore depends on the recovery temperature RTx\index{RTx} as well
as the dynamic\index{pressure!dynamic} and ambient pressure\index{pressure!ambient},
usually respectively QCXC\index{QCXC} and PSXC\index{PSXC}. The
ambient and dynamic pressures are first corrected from the raw measurements\index{measurement!raw}
QCX\index{QCX} and PSX\index{PSX} to obtain variables that account
for deviations caused by airflow around the aircraft and/or position-dependent
systematic errors, as discussed in the section describing PSxC. The
following basic equations are developed on the basis of conservation
of energy for a perfect gas\index{perfect gas} undergoing an adiabatic
compression.\index{compression!adiabatic}\\
\\
This section combines discussion of the calculations of temperature
and airspeed\index{airspeed}, to reflect the linkage\index{linkage. temperature and airspeed}
between these derived measurements. To provide accuracy in the equations,
this discussion considers effects of the humidity\index{correction!moist air}
of the air on characteristics like the gas constant\index{gas constant!moist air}
and the specific heats.\index{specific heat!moist air} Most archived
data before 2012 used values for dry air, although a special variable
TASHC\index{TASHC} has been used to represent the true airspeed in
cases where the correction was significant. That variable is based
on a good approximation to the results from the following equations;
see the discussion of TASHC later in this section. TASHC is now considered
an obsolete variable. New variables ATxD and TASxD have been introduced
that neglect the humidity corrections and perform all calculations
as if the humidity is negligible.\\
\\
\label{ambient temperature and TAS calculation}As discussed above,
temperature\index{temperature!sensor} sensors on aircraft that are
exposed to the airflow do not measure the total temperature\index{temperature!total}
but rather the temperature of the air immediately in contact with
the sensing element. This air will not have undergone an adiabatic
deceleration completely to zero velocity and hence will have a temperature
$T_{r}$ somewhat less than the total temperature $T_{t}$ that would
require the air to reach zero velocity. $T_{r}$ is the measured or
``recovery'' temperature\index{temperature!recovery}.\sindex[lis]{Ta@$T_{a}$= ambient air temperature in absolute units; sometimes,
$T_{K}$}\sindex[lis]{Tr@$T_{r}$= recovery temperature}\sindex[lis]{Tt@$T_{t}$= total air temperature},
The ratio of the actual temperature difference attained to the temperature
difference relative to the total temperature is defined to be the
``recovery factor''\index{recovery factor} $\alpha:$\sindex[lis]{ar@$\alpha_{r}$= recovery factor, temperature probe}\\
\begin{equation}
\alpha_{r}=\frac{T_{r}-T_{a}}{T_{t}-T_{a}}\label{eq:8.2-1}
\end{equation}
where $T_{a}$ is the ambient air temperature. From conservation of
energy\index{conservation of energy}:\\
\begin{equation}
\frac{U_{a}^{2}}{2}+c_{p}^{\prime}T_{a}=\frac{U_{r}^{2}}{2}+c_{p}^{\prime}T_{r}=\frac{U_{t}^{2}}{2}+c_{p}^{\prime}T_{t}\label{eq:8.1-1}
\end{equation}
\\
where primes on quantities like $c_{p}^{\prime}$, or (below) $c_{v}^{\prime}$
and $R^{\prime}$ denote properties\index{properties of moist air}
of moist air, respectively the specific heat\index{specific heat}
at constant pressure, specific heat at constant volume, and gas constant.\index{gas constant}
\\
\\
\fbox{\begin{minipage}[t]{1\columnwidth - 2\fboxsep - 2\fboxrule}%
\textbf{Moist-air considerations:}\\
\\
Primes on the symbols denote that these values should be moist-air
values, appropriately weighted averages of the dry-air and water-vapor
contributions. The practice prior to 2014 was to use the dry-air values
for specific heats and the gas constant, except as described in connection
with TASHC below. Since 2014, calculations use the appropriate values
for moist air, except that to avoid errors introduced by unrealistically
high measurements of humidity the humidity correction was limited
to be less than or equal to the equilibrium value at the measured
temperature. The formulas used for the specific heats and gas constant
of moist air in terms of the water vapor pressure $e$, the specific
heats for dry air ($c_{pd}=\frac{7}{2}R_{0},\,c_{vd}=\frac{5}{2}R_{0}$)
and water vapor ($c_{pw}=4R_{0},\,c_{vw}=3R_{0}$), and the ratio
of molecular weights ($\epsilon=M_{W}/M_{d}$) are those of Khelif
et al.~1999:\sindex[lis]{epsilon@$\epsilon=M_{W}/M_{d}$}\index{gas constant!moist air}\index{specific heat!moist air}\sindex[lis]{R@$R^{\prime}=$gas constant for moist air}\sindex[lis]{cpprime@$c_{p}^{\prime}$= specific heat at constant pressure for
moist air}\sindex[lis]{cvprime@$c_{v}^{\prime}=$specific heat at constant volume for moist
air}
\begin{equation}
R^{\prime}=R_{d}/[1+(\epsilon-1)\frac{e}{p}]\label{eq:moistR}
\end{equation}
\begin{equation}
c_{v}^{\prime}=\frac{(p-e)R^{\prime}}{pR_{d}}\frac{5R_{0}}{2M_{d}}+\frac{eR^{\prime}}{pR_{w}}\frac{3R_{0}}{M_{w}}=c_{vd}\frac{R^{\prime}}{R_{d}}\left(1+\frac{1}{5}\frac{e}{p}\right)\label{eq:moistcv}
\end{equation}

\begin{equation}
c_{p}^{\prime}=c_{pd}\frac{R^{\prime}}{R_{d}}\left(1+\frac{1}{7}\frac{e}{p}\right)\label{eq:moistcp}
\end{equation}

\begin{equation}
\gamma\,^{\prime}=\gamma_{d}\frac{1+\frac{1}{7}\frac{e}{p}}{1+\frac{1}{5}\frac{e}{p}}\label{eq:moistgamma}
\end{equation}
See also the discussion of TASHC\index{TASHC} in section \ref{sec:WIND}
and the reference there for Khelif et al.~1999.%
\end{minipage}}\\
\\
\\
\label{DiscussionOfMoistAirVariables}\index{moist-air properties}In
(\ref{eq:8.1-1}), $\{U_{a},\,U_{r},\,U_{t}\}$ \sindex[lis]{Ua@$U_{a}$= true airspeed (sometimes $U$)}are
respectively the aircraft true airspeed\index{airspeed}, the airspeed
relative to the aircraft of the air in thermal contact with the sensor,
and the airspeed of air relative to the aircraft when fully brought
to the motion of the sensor (i.e., zero). Then, from (\ref{eq:8.1-1})\\
\begin{equation}
T_{a}=T_{r}-\alpha_{r}\frac{U_{a}^{2}}{2c_{p}^{\prime}}\label{eq:8.3-1}
\end{equation}
The temperature sensors used on RAF aircraft are designed to decelerate
the air adiabatically to near zero velocity. Recovery factors\index{recovery factor!Rosemount sensors}
determined from wind tunnel testing for the Rosemount sensors are
approximately 0.97 (unheated model) and 0.98 (heated models).\footnote{The recovery factor determined for the now-obsolete NCAR reverse-flow
sensor was 0.6. The recovery factor for the now retired NCAR fast-response
(K-probe) temperature sensor was 0.8. } These values\index{recovery factor} have also been confirmed from
flight maneuvers, often from ``speed runs'' where the aircraft is
flown level through its speed range and the variation of recovery
temperature with airspeed is used with (\ref{eq:8.3-1}), with the
assumption that $T_{a}$ remains constant, to determine the recovery
factor. Data files and project reports normally document what recovery
factor was used for calculating the true airspeed and ambient temperature
for a particular project.\\
\\
Because the values used in processing have varied, the project reports
should be consulted to find what was used for particular projects.
The Goodrich Technical Report 5755\label{punch:4-4} documents wind-tunnel
testing of the probes formerly made by Rosemount. Their plot showed
that, for heated sensors, there is a significant variation with Mach
number\index{Mach number} ($M$); cf Eq.~\pageref{eq:MachEquation}).
The dependence in their plot is represented well by the following
equations, where $\alpha_{r}^{[h]}$ refers to heated probes and $\alpha_{r}^{[u]}$
to unheated probes:
\begin{equation}
\alpha_{r}^{[h]}=0.988+0.053(\log_{10}M)+0.090(\log_{10}M)^{2}+0.091(\log_{10}M)^{3}\label{eq:RecFactorMachDep}
\end{equation}
\begin{equation}
\alpha_{r}^{[u]}=0.9959+0.0283(\log_{10}M)+0.0374(\log_{10}M)^{2}+0.0762(\log_{10}M)^{3}\label{eq:RecFactorUnheated}
\end{equation}
Some studies of the recovery factor are discussed further in \href{https://drive.google.com/open?id=0B1kIUH45ca5AOWlIbGxVcC13SHM}{this memo}.\\
\\
The true airspeed $U_{a}$ is used in (\ref{eq:8.3-1}) to calculate
the ambient temperature $T_{a}$. However, the ambient temperature
is also needed to calculate the true airspeed. Therefore the constraints
imposed on ambient temperature and true airspeed by the measurements
of recovery temperature, total pressure\index{pressure!total} (the
pressure measured by a pitot tube pointed into the airstream and assumed
to be that obtained when the incoming air is brought to rest relative
to the aircraft), and ambient pressure must be used to solve simultaneously
for the two unknowns, temperature and airspeed. \\
\\
The relationship is conveniently derived by first calculating the
dimensionless Mach number\index{Mach number}\sindex[lis]{M@$M$= Mach number, ratio of airspeed to the speed of sound}
($M$), which is the ratio of the airspeed to the speed of sound\index{speed of sound}
($U_{s}=\sqrt{\gamma^{\prime}R^{\prime}T_{a}}$ \sindex[lis]{Us@$U_{s}$= speed of sound}where
$\gamma^{\prime}$ is the ratio\sindex[lis]{gamma@$\gamma^{\prime}=$$c_{p}^{\prime}/c_{v}^{\prime}$}
of specific heats of (moist) air, $c_{p}^{\prime}/c_{v}^{\prime}$).
The Mach number is a function of air temperature only and can be determined
as follows: \\
a). Express energy conservation, as in (\ref{eq:8.1-1}), in the form\\
\begin{equation}
d\left(\frac{U^{2}}{2}\right)+c_{p}^{\prime}dT=0\,\,\,\,.\label{eq:8.1a-1}
\end{equation}
where the total derivatives apply along a streamline as $U$ changes
from $U_{a}$ to $U_{t}=0$ and $T$ changes from $T_{a}$ to $T_{t}$.\\
b). Use the perfect gas law to replace $dT$ with $\frac{pV}{nR}(\frac{dV}{V}+\frac{dp}{p})$
where $V$\sindex[lis]{V@$V$= volume} and $p$\sindex[lis]{p@$p$= pressure}
are the volume and pressure of a parcel of air. Then use the expression
for adiabatic compression\index{adiabatic compression} in the form
$pV^{\gamma}=constant$ to replace the derivative $\frac{dV}{V}$
with $-\frac{1}{\gamma}\frac{dp}{p}$, leading to $dT=\frac{R^{\prime}T}{c_{p}^{\prime}}\frac{dp}{p}$
or, after integration, $T(p)=T_{a}\left(\frac{p}{p_{a}}\right)^{R^{\prime}/c_{p}^{\prime}}.$
Using this expression for $T$ in the formula for $dT$ and then integrating
both total derivatives in (\ref{eq:8.1a-1}) along the streamline
leads to \\
\begin{equation}
\frac{U_{a}^{2}}{2}+c_{p}^{\prime}T_{a}=c_{p}^{\prime}T_{a}\left(\frac{p_{t}}{p_{a}}\right)^{\frac{R^{\prime}}{c_{p}^{\prime}}}\label{eq:8.4-1}
\end{equation}
where $p_{t}$ is the total pressure\index{pressure!total} (i.e.,
PSXC\index{PSXC}+QCXC\index{QCXC}) and $p_{a}$ the ambient pressure\index{pressure!ambient}\sindex[lis]{pa@$p_{a}$= ambient air pressure}
(PSXC\index{PSXC}).\\
\\
c). Use the above definition of the Mach number\index{Mach number}
$M$ ($M=U_{a}/U_{s}$) in the form $U_{a}^{2}=\gamma^{\prime}M^{2}R^{\prime}T_{a}$
to obtain:\\
\begin{equation}
M^{2}=\left(\frac{2c_{v}^{\prime}}{R^{\prime}}\right)\left[\left(\frac{p_{t}}{p_{a}}\right)^{\frac{R^{\prime}}{c_{p}^{\prime}}}-1\right]\label{eq:8.5-1}
\end{equation}
which is the same as (\ref{eq:MachEquation}). This equation shows
that $M$ can be found from $p_{t}$ and $p_{a}$ alone, except for
the moist-air corrections. \sindex[lis]{ptotal@$p_{t}=$total pressure (ambient + dynamic)}\\
\\
d). Use the expression for ambient temperature in terms of recovery
temperature and airspeed, (\ref{eq:8.3-1}), to obtain the temperature
in terms of the Mach number and the recovery temperature:\sindex[lis]{gamma@$\gamma^{\prime}=$$c_{p}^{\prime}/c_{v}^{\prime}$}\\
\begin{eqnarray}
T_{a} & = & T_{r}-\alpha_{r}\frac{U_{a}^{2}}{2c_{p}^{\prime}}=T_{r}-\alpha_{r}\frac{M^{2}\gamma^{\prime}R^{\prime}T_{a}}{2c_{p}^{\prime}}\nonumber \\
 & = & \frac{T_{r}}{1+\dfrac{\alpha_{r}M^{2}R^{\prime}}{2c_{v}^{\prime}}}\label{eq:8.6-1}
\end{eqnarray}
\\
e). Express the true airspeed\index{airspeed} ($U_{a}$) as\\
\begin{equation}
U_{a}=M\sqrt{\gamma\,^{\prime}R^{\prime}T_{a}}\label{eq:8.7-1}
\end{equation}
\\
\label{ATX discussion}Then the temperature is found as described
in the following box:\footnote{A problem sometimes arises from use of the measured humidity, because
that measurement might be obviously in error. For example, following
descents the dew point determined from chilled-mirror hygrometers
sometimes overshoots the correct value significantly, producing dew-point
measurements well above the measured temperature. If such measurements
are used, the result can produce a significant error in derived variables
based on the humidity-corrected gas constant and specific heats. If
the measurements are flagged as bad, there will be gaps in derived
variables. To avoid these two errors, the corrections applied to the
gas constant and specific heats are treated as follows:
\begin{itemize}
\item The humidity correction is limited to not more than that given by
the water-equilibrium humidity at the temperature ATXD, calculated
using dry-air specific heats and gas constant. 
\item If the humidity from the primary sensor is flagged as a missing measurement
(e.g., from a dew-point sensor), a secondary measurement is used (e.g.,
the VCSEL)\index{hygrometer!VCSEL} in cases when the secondary sensor
is almost always present in an experiment.
\item As a backup, the variables TASxD and ATxD are always calculated omitting
the humidity correction to the gas constant and the specific heats.
These variables usually provide continuous measurements, although
they will be offset from the humidity-corrected values. The offset
indicates the magnitude of the correction when both are present, and
one of the variables TASxD (ATxD) may be selected as TASX (ATX) in
cases where missing values might cause a problem for derived variables. 
\end{itemize}
}\\
\\
\fbox{\begin{minipage}[t]{0.9\textwidth}%
RTX\index{RTX} = recovery temperature ($T_{r})$\\
QCxC\index{QCxC} = dynamic pressure, corrected ($q_{a}$)\\
PSXC\index{PSXC} = ambient pressure, after airflow/location correction
($p_{a}$)\\
MACHx\index{MACHx}\sindex[var]{MACH = Mach number} = Mach number
based on QCxC and PSXC; cf.~(\ref{eq:8.5-1})\\
MACHX = best Mach number, based on QCXC and PSXC\\
$\alpha_{r}$ = recovery factor for the particular temperature sensor\\
$R^{\prime}$, $c_{v}^{\prime}$ and $c_{p}^{\prime}$ as defined
above and in the list of symbols\\
\\
\rule[0.5ex]{1\columnwidth}{1pt}

From (\ref{eq:8.5-1}),

\begin{equation}
\mathrm{MACHx}=\left\{ \left(\frac{2c_{v}^{\prime}}{R^{\prime}}\right)\left[\left(\frac{\mathrm{\{PSXC\}+\{QCxC\}}}{\mathrm{\{PSXC\}}}\right)^{\frac{R^{\prime}}{c_{p}^{\prime}}}-1\right]\right\} ^{1/2}\label{eq:8.8-1}
\end{equation}
\\
From (\ref{eq:8.6-1})

\begin{equation}
\mathrm{ATx}=\frac{\mathrm{\left(\{RTx\}+T_{0}\right)}}{\left(1+\dfrac{\alpha_{r}\mathrm{(\{MACHX\})}^{2}R^{\prime}}{2c_{v}^{\prime}}\right)}-T_{0}\label{eq:8.9-1}
\end{equation}

%
\end{minipage}}

\textbf{In-cloud Air Temperature, Radiometric ($^{\circ}$C):} \textbf{\uline{AT\_ITR}}\index{AT_ITR@AT\_ITR}\sindex[var]{AT_ITR@AT\_ITR}
\\
\emph{The radiometric ambient air temperature measured by the In-cloud
Air Temperature Radiometer,} which measures the radiometric temperature\index{temperature!radiometric}\index{temperature!in-cloud}
in the 4.3 $\mu$m CO$_{2}$ band.\label{AT_ITR} Its primary use
is in water cloud when the standard thermometers are affected by wetting.\index{wetting!of thermometers}
In clear air the temperature is an average over an integrating range
of up to 100s of meters away from the aircraft, whereas in clouds
the integrating range is as little as 10 meters because of water droplets.
The calibration is by a polynomial fit of the internal reference temperature
and measured radiance to the ATX temperature.\label{punch:4-5}

\textbf{Ophir Air Temperature ($^{\circ}$C): }\textbf{\uline{OAT}}\index{OAT}\sindex[var]{OAT}\\
\emph{The radiometric temperature reported by the Ophir III radiometer,}
which operates on the same principles as the ITR,\label{OAT} with
the same limitations. For more information on this instrument, see
this \href{http://opensky.library.ucar.edu/collections/TECH-NOTE-000-000-000-822}{Technical Note}.
The in-cloud air temperature radiometer is a later, improved version,
but the Ophir III radiometer remains in use.
\end{hangparagraphs}


\subsection{Humidity\index{humidity}}
\begin{hangparagraphs}
\textbf{Dew/Frost Point ($\text{�}$C): }\textbf{\uline{DPx}}\sindex[var]{DPx}\index{DPx}\textbf{,
}\textbf{\uline{DP\_x}}\index{DP_x@DP\_x}\sindex[var]{DP_x@DP\_x}\textbf{,
}\textbf{\uline{MIRRTMP\_DPx}}\textbf{\sindex[var]{DPx}}\index{MIRRTMP_DPx@MIRRTMP\_DPx}\\
\emph{The mirror temperature measured directly by a dew-point sensor,
without correction. }The dew point\index{dew point} or frost point\index{frost point}
is measured by either an EG\&G Model 137, a General Eastern Model
1011B or a Buck Model 1011C dew-point hygrometer\index{hygrometer!dew point}.
Below 0\textbf{$^{\circ}$C} the instrument is assumed to be responding
to the frost point, although occasionally in climbs there is a short
transition near the freezing level before the condensate on the mirror
of the instrument freezes and there may be a measurement error before
the condensate freezes. The measurements are usually made within a
housing where the pressure ($p_{h})$ may differ from the ambient
pressure, so the pressure in the housing affects the measured dew
point or frost point. The housing pressure is often adjusted to be
near the ambient pressure by appropriate orientation of inlets, and
recently the pressure in the housing is measured and a correction
is applied, as discussed in the next paragraph. 

\textbf{Corrected Dew Point (C): DPxC}\sindex[var]{DPxC}\index{DPxC}\textbf{}\footnote{See also DP\_VXL and DP\_CR2C below}\textbf{}\\
\index{dew point!corrected}\emph{\index{frost point}The dew point
obtained from the original measurement after correction for the housing
pressure, the enhancement of the equilibrium vapor pressure arising
from the total pressure (discussed below), and conversion from frost
point if appropriate,} The result is the temperature at which the
equilibrium vapor pressure\index{pressure!water vapor!equilibrium}
over a plane water surface in the absence of other gases would match
the actual water-vapor pressure. Dew/frost-point hygrometers\index{hygrometer}
measure the equilibrium point in the presence of air\index{enhancement factor},
and the presence of air affects the measurement in a minor way that
is represented by a small correction here named the ``enhancement
factor.'' In the case where the dew-point or frost-point sensor is
exposed to ambient air directly, the enhancement factor is defined
so that the ambient vapor pressure\sindex[lis]{ea@$e_{a}$= ambient water vapor pressure}
$e_{a}$ is related to $T_{p}$, the \emph{measured }dew or frost
point \emph{in the presence of air} having total pressure $p$, by
$e_{a}=f(p,T_{P})\,e_{s}(T_{p})$ \sindex[lis]{fpT@$f(p,T_{p})$=water vapor pressure enhancement factor}where
$e_{s}(T_{p})$ is the vapor pressure in equilibrium with ice or water
at the dew or frost point $T_{p}$ \emph{in the absence of air.} Calculation
of DPxC removes this dependence, so the vapor pressure obtained from
$e_{s}(\{\mathrm{DPxC\}})$ will be that vapor pressure corresponding
to equilibrium \emph{in the absence of air}. In addition, if the measurement
is below 0$^{\circ}$C, it is assumed to be a measurement of frost
point and a corresponding dew point is calculated from the measurement
(also with correction for the influence of the total pressure on the
measurement). Some changes were made to these calculations in 2011;
for more information, see \href{https://drive.google.com/open?id=0B1kIUH45ca5Ab1NTRVo0bjdac0U}{this memo}.\\
\\
An additional correction is needed in those cases where the pressure\index{pressure!dew point housing}
in the housing of the instrument (measured as PSDPx or CAVP\_x) differs
from the ambient pressure, because the changed pressure affects the
partial pressure\index{pressure!partial, water vapor} of water vapor
in proportion to the change in total pressure and so changes the measured
dew point from the desired quantity (that in the ambient air) to that
in the housing\index{hygrometer!housing}. This is especially important
in the case of the GV because the potential effect increases with
airspeed. If the pressure in the housing is measured or otherwise
known (e.g., from correlations with other measurements), then this
correction can be introduced into the processing algorithm at the
same time that the correction for the presence of dry air is introduced,
and the enhancement factor should be evaluated at the pressure in
the housing. \\
\\
The relationship between water-vapor pressure and dew- or frost-point
temperature is based on the Murphy and Koop\footnote{Q. J. R. Meteorol. Soc. (2005), 131, pp. 1539\textendash 1565}\index{Murphy and Koop, 2005}\index{pressure!water vapor!equilibrium}
(2005) equations.\footnote{Prior to 2010, the vapor pressure relationship used was the Goff-Gratch
formula as given in the Smithsonian Tables (List, 1980).} They express the equilibrium vapor pressure as a function of frost
point or dew point \emph{and at a total air pressure $p$} via equations
that are equivalent to the following:\\
\begin{eqnarray}
e_{s,i}(T_{FP})= & b_{0}^{\prime}\exp(b_{1}\frac{(T_{0}-T_{FP})}{T_{0}T_{FP}}+b_{2}\ln(\frac{T_{FP}}{T_{0}})+b_{3}(T_{FP}-T_{0}))\label{eq:MK1-1}
\end{eqnarray}
\begin{equation}
e_{s,w}(T_{DP})=c_{0}\exp\left((\alpha-1)c_{6}+d_{2}(\frac{T_{0}-T_{DP}}{T_{DP}T_{0}})\right)+d_{3}\ln(\frac{T_{DP}}{T_{0}})+d_{4}(T_{DP}-T_{0})\label{eq:proposedNewWater}
\end{equation}
\begin{equation}
f(p,T_{P})=1+p(f_{1}+f_{2}T_{p}+f_{3}T_{P}^{2})\label{eq:EnhancementFactor}
\end{equation}
where $e$\sindex[lis]{e@$e$= water vapor pressure} is the water
vapor pressure, $T_{FP}$\sindex[lis]{Tdp@$T_{DP}$= temperature at the dew point}\sindex[lis]{Tfp@$T_{FP}$= temperature at the frost point}
or $T_{DP}$ is the frost or dew point, respectively, expressed in
kelvin, $T_{0}$=273.15\,K, $e_{s,i}(T_{FP})$\sindex[lis]{esi@$e_{s,i}$= equilibrium vapor pressure over a plane ice surface}
is the equilibrium vapor pressure over a plane ice surface at the
temperature $T_{FP}$, $e_{s,w}(T_{DP})$\sindex[lis]{esl@$e_{s,l}$= equilibrium vapor pressure over a plane water surface}
is the equilibrium vapor pressure over a plane water surface at the
temperature $T_{DP}$ (above or below $T_{0}$), and $f(p,T_{P})$\sindex[lis]{fpT@$f(p,T_{p})$=water vapor pressure enhancement factor}is
the enhancement factor at total air pressure $p$ and temperature
$T_{p}$\sindex[lis]{Tp@$T_{p}=$dew point temperature if above 0$^{\circ}C$, frost point
temperature otherwise}, with $T_{P}$ equal to $T_{DP}-T_{0}$ when above $T_{0}$ and $T_{FP}-T_{0}$
when below 0$\text{�}$C . \\
\\
The coefficients used in the above formulas are given in the following
tables, with the additional definitions that \sindex[lis]{alphaT@$\alpha_{T}$= tanh($e_{s}(T-T_{x})$, Murphy/Koop equations}$\alpha_{T}=\tanh(c_{5}(T-T_{x}))$,
\sindex[lis]{Tx@$T_{x}$= 218.8~K, Murphy/Koop equations}$T_{X}$
= 218.8~K, and $d_{i}=c_{i}+\alpha_{T}c_{i+5}$ for i = \{2,3,4\}:\sindex[lis]{c19@$c_{0-9}$=coefficients, vapor pressure equation}\sindex[lis]{b03@$b_{0-3}$=coefficients, vapor pressure equation}\sindex[lis]{f13@$f_{1-3}$=coefficients, vapor pressure equation}\\
\fbox{\begin{minipage}[t]{0.95\textwidth}%
\textbf{~~~~~}%
\begin{tabular}{|c|c|}
\hline 
\textbf{Coefficient} & \textbf{Value}\tabularnewline
\hline 
\hline 
$b_{0}^{\prime}$ & 6.11536\,hPa\tabularnewline
\hline 
$b_{1}$ & $-5723.265\,K,$\tabularnewline
\hline 
$b_{2}$ & 3.53068\tabularnewline
\hline 
$b_{3}$ & -0.00728332\,K$^{-1}$\tabularnewline
\hline 
$f_{1}$ & 4.923$\times10^{-5}$ hPa$^{-1}$\tabularnewline
\hline 
$f_{2}$ & -3.25$\times10^{-7}$hPa$^{-1}$K$^{-1}$\tabularnewline
\hline 
$f_{3}$ & 5.84$\times10^{-10}$hPa$^{-1}$K$^{-2}$\tabularnewline
\hline 
\end{tabular}\textbf{~~~~~}%
\begin{tabular}{|c|c|}
\hline 
\textbf{coefficient} & \textbf{value}\tabularnewline
\hline 
\hline 
$c_{0}$ & 6.091886 hPa\tabularnewline
\hline 
$c_{1}$ & 6.564725\tabularnewline
\hline 
$c_{2}$ & -6763.22\,K\tabularnewline
\hline 
$c_{3}$ & -4.210\tabularnewline
\hline 
$c_{4}$ & 0.000367\,K$^{-1}$\tabularnewline
\hline 
$c_{5}$ & 0.0415\,K$^{-1}$\tabularnewline
\hline 
$c_{6}$ & -0.1525967\tabularnewline
\hline 
$c_{7}$ & -1331.22\,K\tabularnewline
\hline 
$c_{8}$ & -9.44523\tabularnewline
\hline 
$c_{9}$ & 0.014025\,K$^{-1}$\tabularnewline
\hline 
\end{tabular}%
\end{minipage}}\\
\\
The vapor pressure in the instrument housing, \sindex[lis]{eh@$e_{h}$= water vapor pressure in an instrument housing}$e_{h}$,
is related to the sensed dew or frost point according to equation
(\ref{eq:MK1-1}) or (\ref{eq:proposedNewWater}), but further corrections
must also be made for the enhancement factor and to account for possible
difference between the pressure in the sensor housing\sindex[lis]{ph@$p_{h}$= pressure in a sensor housing}
$p_{h}$ and the ambient pressure $p_{a}$: \\
\begin{equation}
e_{a}=f(p_{a},T_{p})e_{h}\frac{p_{a}}{p_{h}}\label{eq:HousingPressureCorrection}
\end{equation}
 \\
\label{punch:4.6}Because processing to obtain the corrected dew
point DPxC\index{DPxC} from the ambient vapor pressure\index{pressure!water vapor}
$e_{a}$ would require difficult inversion of the above formulas,
interpolation is used instead. A table constructed from (\ref{eq:MK1-1})
and another constructed from (\ref{eq:proposedNewWater}), giving
water vapor pressure as a function of frost point or dew point temperature
in 1$^{\circ}$C increments from -100 to +50$\text{�}$C, is then
used with three-point Lagrange interpolation (via a function described
below as $F_{D}(e)$)\sindex[lis]{Fd@$F_{d}$= interpolation formula for dew point}
to find the dew point temperature from the vapor pressure.\footnote{prior to 2011 the conversion was made using the formula $\mathrm{DPxC=0.009109+DPx(1.134055+0.001038DPx)}$.
For instruments producing measurements of vapor density (RHO), the
previous Bulletin 9 section incorrectly gave the conversion formula
as $DPxC=273.0Z/(22.51-Z)$, a conversion that would apply to frost
point, not dew point. However, the code in use shows that the conversion
was instead $237.3Z/(17.27-Z)$, where Z in both cases is $Z=\ln((\mathrm{ATX}+273.15)\mathrm{RHO/1322.3)}$. }
\end{hangparagraphs}

~~
\begin{hangparagraphs}
~~~\\
Tests of these interpolation formulas against high-accuracy numerical
inversion of formulas (\ref{eq:MK1-1}) and (\ref{eq:proposedNewWater})
showed that the maximum error introduced by the interpolation formula
was about 0.004$^{\circ}$C and the standard error about 0.001$^{\circ}$C.
This inversion then provides a corrected dew point\index{dew point!corrected}
that incorporates the effects of the enhancement factor as well as
differences between the ambient pressure and that in the housing.
The algorithm is documented in the box below.\\
\\
For other instruments that measure vapor density, such as a Lyman-alpha\index{Lyman-alpha hygrometer}
or tunable diode laser hygrometers\index{hygrometer!tunable diode laser}
(including the Vertical Cavity Surface Emitting Laser (VCSEL) hygrometer),\index{hygrometer!VCSEL}
a similar conversion is made from vapor density to dew point, as documented
below:\\
\\
\\
\\
\doublebox{\begin{minipage}[t]{1\columnwidth - 2\fboxsep - 7.5\fboxrule - 1pt}%
\begin{center}
{[}See next page{]}
\par\end{center}%
\end{minipage}}\\
\fbox{\begin{minipage}[t]{0.95\columnwidth}%
$T_{p}$ = DPx\index{DPx} = mirror-temperature measurement\sindex[lis]{Tp@$T_{p}$=mirror temperature}
from instrument x {[}$^{\circ}$C{]}, or alternately\\
RHO = water vapor density\index{water vapor!density} measurement
{[}$\mathrm{g\,}\mathrm{m}^{-3}${]}; only one is used in any calculation\\
ATX\index{ATX} = reference ambient temperature\index{temperature!ambient}
{[}$^{\circ}C${]}\\
$T_{K}$=ATX+$T_{0}$~$^{\dagger}$\sindex[lis]{Tk@$T_{K}$= absolute temperature in kelvin}
= ambient temperature {[}K{]} \\
$p$ = PSXC\index{PSXC} = reference ambient pressure {[}hPa{]}\\
$p_{h}$ = CAVP\_x (e.g.) = pressure in instrument housing {[}hPa{]}\\
$e_{t}$ = intermediate vapor pressure used for calculation only\\
$e$ = EWx = water vapor pressure from source x {[}hPa{]}\\
$M_{w}$ = molecular weight of water$^{\dagger}$\\
$R_{0}$ = universal gas constant$^{\dagger}$\\
$f(p_{h},T_{p})$ = enhancement factor (cf.~(\ref{eq:EnhancementFactor}))\\
$F_{d}(e)$ = interpolation formula\sindex[lis]{Fd@$F_{d}(e)$=interpolation formula for dew point}
giving dew point temperature from water vapor pressure

\rule[0.5ex]{1\linewidth}{1pt}

for dew/frost point hygrometers, producing the measurement DPx:~~~~if
DPx < 0$\text{�}$C:

~~~~~~~~obtain $e_{t}$ from (\ref{eq:MK1-1}) using $T_{FP}$=DPx
+ $T_{0}$

~~~~else (i.e., DPx $\geq$ 0$\text{�}$C):

~~~~~~~~obtain $e_{t}$ from (\ref{eq:proposedNewWater})
using $T_{DP}=\mathrm{DPx}+T_{0}$

~~~~correct $e_{t}$ for enhancement factor and internal pressure,
to get ambient vapor pressure $e$:

~~~~~~~~$e=f(p_{h},T_{P})\,e_{t}\,(p/p_{h})$

~~~~obtain DPxC by finding the dew point corresponding to the
vapor pressure $e$:

~~~~~~~~DPxC = $F_{d}(e)$

\textemdash - \textemdash - \textemdash - \textemdash - \textemdash -
\textemdash - \textemdash - \textemdash - \textemdash - \textemdash -
\textemdash - \textemdash - \textemdash - \textemdash - \textemdash -

for other instruments producing measurements of vapor density (RHO
{[}g~m$^{-3}${]}:\footnote{prior to 2011 the following formula was used: 
\[
Z=\frac{\ln((\mathrm{ATX}+273.15)\,\mathrm{RHO}}{1322.3}
\]

\[
\mathrm{DPxC}=\frac{273.0\,Z}{(22.51-Z)}
\]
}

~~~~find the water vapor pressure in units of hPa:

~~~~~~~~$e=$ (\{RHO\}~$R_{0}\,T_{K}$\,/\,$M_{w}$)$\times10^{-5}$

~~~~find the equivalent dew point:

~~~~~~~~DPxC = $F_{d}(e)$%
\end{minipage}}\\
\begin{comment}
this is a description of the old Bulletin-9 section, saved here for
reference. It did not correspond to the code in use prior to the 2011
change, however; only the >0 form of $f$ was used, and there was
a small error in coefficients, as described in a 2010 Note that documented
the change made in 2011 and the reasons for it.%
\begin{minipage}[t]{1\columnwidth}%
DPX = measured dew point ($\geq0^{\circ}$) or frost point ($<0^{\circ})$
in $^{\circ}C$\\
$D_{K}$ = DPX + 273.15\,K = measured dew point in kelvin\\
$f$= enhancement factor (Appendix C)\\
\\
\rule[[0.5ex]]{1.0\linewidth}{1pt}if DPX$\geq0^{\circ}C:$\\
\[
A=23.832241-5.02808\,\log\left(D_{K}\right)-1.3816\times10^{-7}(10^{11.334-0.0303998(D_{K}})
\]
\[
f=1.0007+(3.46\times10^{-6}\mathrm{PSXC})
\]
if DPX $<0^{\circ}C$:
\[
A=3.56654\log_{10}(D_{K})-0.0032098(D_{K})-\frac{2484.956}{D_{K}}+2.0702294
\]
\[
f=1.0003+(4.18\times10^{-6}\mathrm{PSXC})
\]
vapor pressure:

\begin{equation}
\mathrm{EDPC}=f\,10^{A}\label{eq:8.11GoffG-1}
\end{equation}
%
\end{minipage}
\end{comment}

\textbf{Dew Point Determined from the VCSEL Hygrometer ($^{\circ}$C):
}\textbf{\uline{DP\_VXL}}\index{DP_VXL@DP\_VXL}\sindex[var]{DP_VXL@DP\_VXL}\\
\emph{The dew point temperature determined from the measured water
vapor density from the VCSEL hygrometer. }\index{hygrometer!VCSEL}The
calculation is as described at the bottom of the box immediately above
this paragraph. The water vapor density converted from a molecular
density {[}molecules~cm$^{-3}${]} to a mass density {[}g~m$^{-3}${]}
via\footnote{The conversion factor is given by this formula:\\
\[
C^{\prime}=\frac{10^{6}\mathrm{cm}^{3}}{\mathrm{m}^{3}}\times\frac{M_{W}^{\dagger}}{N_{A}^{\dagger}}
\]
where $N_{A}$ is the Avogadro constant, 6.022147$\times10^{26}$
molecules~kmol$^{-1}$\index{Avogadro constant}\sindex[lis]{NA@$N_{A}$ = Avogadro constant, molecules per kmol}.} \{CONCV\_VXL\}{*}2.9915$\times10^{-17}$ is used for \{RHO\}. DP\_VXL
is given by DPxC on the last line of that algorithm box. See CONCV\_VXL
below.

\textbf{Frost Point Temperature from the CR2 Cryogenic Hygrometer
($^{\circ}$C): }\textbf{\uline{FP\_CR2\sindex[var]{FP_CR2@FP\_CR2}\index{FP_CR2@FP\_CR2}}}\textbf{,}\\
\textbf{\uline{MIRRORT\_CR2}}\index{MIRRORT_CR2@MIRRORT\_CR2}\sindex[var]{MIRRORT_CR2@MIRRORT\_CR2}\\
\emph{The mirror temperature in the CR2 cryogenic hygrometer, }which
is normally the frost point inside the measuring chamber of the instrument\emph{.
}\label{punch:4-7}The measurement is often suspect when the value
is above about -15$^{\circ}$C; the measurement is intended for use
below this value. The CR2 is a cabin-mounted instrument, so the measured
pressure (P\_CR2\sindex[var]{P_CR2@P\_CR2}) in the instrument must
be used with the ambient pressure (PSXC) to convert the measurement
to ambient humidity measures like DP\_CR2 and EW\_CR2. 

\textbf{Corrected Dew Point Temperature from the CR2 Cryogenic Hygrometer
($^{\circ}$C): }\textbf{\uline{DP\_CR2C}}\index{DP_CR2C@DP\_CR2C}\sindex[var]{DP_CR2C@DP\_CR2C}\\
\emph{The dew point temperature corresponding to equilibrium at the
ambient humidity, }as determined by the CR2 hygrometer.\index{hygrometer!CR2}
The measurement of the mirror temperature inside the CR2, FP\_CR2,
is converted to a vapor pressure assuming equilibrium water vapor
pressure relative to a plane ice surface at that temperature, and
the resulting vapor pressure is converted to an ambient value via
the assumption that the ratio of vapor pressure internal to the instrument
to ambient vapor pressure is the same as the corresponding total pressure
ratio. The resulting ambient vapor pressure (EW\_CR2) is then converted
to an equivalent ambient dew point. The steps are the same as those
in the algorithm box above, with these substitutions: FP\_CR2 is used
for DPx and P\_CR2 for $p_{h}$.

\textbf{Uncorrected Water Vapor Number Density from the VCSEL Hygrometer
(molecules cm$^{-3}$): }\textbf{\uline{RAWCONC\_VXL}}\index{RAWCONC_VXL@RAWCONC\_VXL}\sindex[var]{RAWCONC_VXL@RAWCONC\_VXL}
\emph{}\\
\emph{The uncorrected water vapor number density reported by the VCSEL
hygrometer.} \index{water vapor!density}This is determined by comparing
the measured absorption peak height against a reference spectrum generated
using the HITRAN spectral parameters, the ambient temperature and
the ambient pressure.\footnote{For details see Zondlo, M. A., M. E. Paige, S. M. Massick, and J.
A. Silver, 2010: Vertical cavity laser hygrometer for the National
Science Foundation Gulfstream-V aircraft. \emph{J. Geophys. Res.,
}\textbf{115,} D20309, doi:10.1029/2010JD014445.}

\textbf{Corrected Water Vapor Concentration from the VCSEL Hygrometer
(molecules cm$^{-3}$): }\textbf{\uline{CONCV\_VXL}}\uline{:}\index{CONCV_VXL@CONCV\_VXL}\sindex[var]{CONCV_VXL@CONCV\_VXL}\\
\emph{The corrected water vapor number density produced by the VCSEL
hygrometer,} after minor corrections for ambient temperature, pressure,
laser intensity and water vapor concentration.\index{concentration!water vapor}\label{punch:4-8}
For more information on calibration and data processing for this instrument,
see the \href{https://www.eol.ucar.edu/instruments/vertical-cavity-surface-emitting-laser-vcsel-hygrometer}{instrument web page}
and additional documentation there.

\textbf{Voltage Output from the UV Hygrometer (V): }\textbf{\uline{XSIGV\_UVH}}\textbf{\index{hygrometer!UV}\index{XSIGV_UVH@XSIGV\_UVH}}\sindex[var]{XSIG_UVH@XSIG\_UVH}\\
\emph{The voltage from a modern (as of 2012) version of the Lyman-alpha
hygrometer,} which provides a signal that represents water vapor density.
The instrument also provides measurements of pressure and temperature
inside the sensing cavity; they are, respectively, XCELLPRES\_UVH
and XCELLTEMP\_UVH. See the discussion of EW\_UVH below for the data-processing
algorithm that uses this variable.

\textbf{Water Vapor Number Density from the UV Hygrometer (molecules
cm$^{-3}$): }\textbf{\uline{CONCH\_UVH}}\index{CONCH_UVH@CONCH\_UVH}\sindex[var]{CONCH_UVH@CONCH\_UVH}\\
\emph{Water vapor number density (or concentration of molecules) measured
by the UV Hygrometer.} This is the direct measurement from the instrument.
Its calculation relies on a bench calibration that fits the water
vapor number density to the Beers-Lambert absorption law and corrects
for output offsets and the effect of UV absorption by atmospheric
constituents other than water vapor. See also the discussion of EW\_UVH
in the paragraph that immediately follows.

\textbf{Water Vapor Pressure (hPa): }\textbf{\uline{EWx}}\textbf{\sindex[var]{EWx}}\index{EWx}\textbf{,
}\textbf{\uline{EWX}}\textbf{\sindex[var]{EWX}}\index{EWX}\textbf{,}
\textbf{\uline{EW\_UVH,\index{EW_UVH@EW\_UVH}}}\sindex[var]{EW_UVH@EW\_UVH}\textbf{
}\textbf{\uline{EDPC}}\sindex[var]{EDPC}\index{EDPC}\textbf{
}(obsolete)\textbf{ }\\
\emph{The ambient vapor pressure of water, }also used in the calculation
of several derived variables. It is often obtained from an instrument
measuring dew point or water vapor density. In the case where it is
derived from a measurement of dew point (DPx\index{DPXC}), a correction
is applied for the enhancement factor\index{enhancement factor} that
influences dew point or frost point measurements.\footnote{prior to 2011, this variable was calculated using the Goff-Gratch
formula. See the discussion of DPXC for more information on previous
calculations.} The formula for obtaining the ambient water vapor pressure as a function
of dew point is given in the discussion of DPxC above, Eqs.~(\ref{eq:proposedNewWater})
and (\ref{eq:EnhancementFactor}), where the calculation of the variables
EWx and EWX are also discussed. EWX (or previously EDPC) is the preferred
variable that is selected from among the possibilities \{EWx\} for
subsequent calculation of derived variables.\\
\\
For the case where water vapor pressure is determined by the VCSEL
hygrometer, EW\_VXL is determined from CONCV\_VXL: EW\_VXL=$C$$k$\{CONCV\_VXL\}\{ATX+273.15)
where $k$ is the Boltzmann constant and $C=10^{-4}$(cm/m)$^{3}$(hPa/Pa)
converts units to hPa.\\
\\
In the case where the water vapor pressure is determined from the
UV Hygrometer data, this variable is calculated using one of two methods:
(a) using the ideal gas law to convert the water vapor number density
from the UV Hygrometer to water vapor pressure, using XCELLTEMP\_UVH
and XCELLPRES\_UVH, the measured temperature and pressure in the absorption
cell, via the equation\\
\[
\mathrm{EW\_UVH=C\,\{CONC\_UVH\}\,\frac{k\,(\mathrm{\{XCELLTEMP\_UVH\}+273.15)\,\mathrm{\{PSX\}}}}{\mathrm{{\{XCELLPRES\_UVH\}}}}}\,\,\,\,;
\]
or (b) through use of a polynomial fit of the form\\
\[
\mathrm{EWX}=c_{0}+c_{1}{\mathrm{\{XSIGV\_UVH\}}}+c_{2}{\mathrm{\{XSIGV\_UVH\}}}^{2}
\]
\\
where EWX is a reference water vapor pressure provided by another
instrument. This preserves the fast-response characteristics of the
UV hygrometer while linking the absolute values to a baseline provided
by a more stable instrument. This can be done on a flight-by-flight
basis and largely eliminates drift.\footnote{For more details see Beaton, S. P. and M. Spowart, 2012: UV Absorption
Hygrometer for Fast-Response Airborne Water Vapor Measurements. \emph{J.
Atmos. Oceanic Technol., }\textbf{\emph{29. }}DOI: 10.1175/JTECH-D-11-00141.1} See the project reports to determine which method was used for a
particular project.

\textbf{Relative Humidity (per cent or Pa/hPa): }\textbf{\uline{RHUM}}\sindex[var]{RHUM}\index{RHUM}\\
\index{humidity!relative}\index{relative humidity|see {humidity, relative}}\emph{The
ratio of the water vapor pressure to the water vapor pressure in equilibrium
over a plane }liquid\emph{-water surface, }scaled to express the
result in units of per cent or Pa/hPa:\\
 %
\fbox{\begin{minipage}[t]{0.95\columnwidth}%
EWX\index{EWX} = atmospheric water vapor pressure (hPa)\\
ATX\index{ATX} = ambient air temperature ($^{\circ}C$)\\
$T_{0}=$273.15 K\\
$e_{s.w}(\mathrm{ATX+T_{0}})$ = equilibrium water vapor pressure\index{pressure!water vapor!equilibrium}
at \emph{dewpoint} ATX (hPa)\\
~~~~~~~~~~(see eq. \ref{eq:proposedNewWater} for the formula
used.)\\
\\
\\
\rule[0.5ex]{1\columnwidth}{1pt}

\begin{equation}
\mathrm{RHUM}=100\%\,\times\,\frac{\mathrm{\{EWX\}}}{e_{s,w}(\mathrm{\{ATX\}+T_{0}})}\label{eq:8.16RHUM-1}
\end{equation}
 %
\end{minipage}}\\
To follow normal conventions, the change in equilibrium vapor pressure
that arises from the enhancement factor\index{enhancement factor}
is not included in the calculated relative humidity, even though the
true relative humidity should include the enhancement factor as specified
in (\ref{eq:EnhancementFactor}) in the denominator of (\ref{eq:8.16RHUM-1}).
\\

\textbf{Relative Humidity with respect to Ice (per cent or Pa/hPa):
}\textbf{\uline{RHUMI}}\sindex[var]{RHUMI}\index{RHUMI}\\
\index{humidity!relative to ice}\index{relative humidity wrt ice|see {humidity, relative to ice}}\emph{The
ratio of the water vapor pressure to the water vapor pressure in equilibrium
over a plane }ice\emph{ surface, }scaled to express the result in
units of per cent or Pa/hPa:\\
 %
\fbox{\begin{minipage}[t]{0.95\columnwidth}%
EWX\index{EWX} = atmospheric water vapor pressure (hPa)\\
ATX\index{ATX} = ambient air temperature ($^{\circ}C$)\\
$T_{0}=$273.15 K\\
$e_{s,i}(\mathrm{ATX+T_{0}})$ = equilibrium water vapor pressure\index{pressure!water vapor!equilibrium}
at \emph{frostpoint} ATX (hPa)\\
~~~~~~~~~~(see eq. \ref{eq:MK1-1} for the formula used.)\\
\\
\\
\rule[0.5ex]{1\columnwidth}{1pt}

\begin{equation}
\mathrm{RHUMI}=100\%\,\times\,\frac{\mathrm{\{EWX\}}}{e_{s,i}(\mathrm{\{ATX\}+T_{0}})}\label{eq:RHUMI}
\end{equation}
 %
\end{minipage}}\\
To follow normal conventions, the change in equilibrium vapor pressure
that arises from the enhancement factor\index{enhancement factor}
is not included in the calculated relative humidity, even though the
true relative humidity should include the enhancement factor as specified
in (\ref{eq:EnhancementFactor}) in the denominator of (\ref{eq:RHUMI}).
\\

\textbf{Absolute Humidity, Water Vapor Density (g/m$^{3}$): }\textbf{\uline{RHOx}}\sindex[var]{RHOx}\index{RHOx}\\
\emph{\index{humidity!absolute}The water vapor density\index{density!water vapor}\index{water vapor!density}
computed from various measurements of humidity as indicated by the
'x' suffix, }and conventionally expressed in units of g\,kg$^{-1}$
or per mille. The calculation proceeds in different ways for different
sensors. For sensors that measure a \index{chilled-mirror}\index{hygrometer!chilled-mirror}chilled-mirror
temperature, the calculation is based on the equation of state for
a perfect gas and uses the water vapor pressure determined by the
instrument, as in the following box.\label{punch:4-9}\\
\\
\fbox{\begin{minipage}[t]{0.95\textwidth}%
ATX\index{ATX} = ambient temperature ($^{\circ}C$)\\
EWX\index{EWX} = water vapor pressure, hPa\\
$C_{mb2Pa}$\sindex[con]{Cmb2@$C_{mb2Pa}$= conversion factor, hPa to Pa}
= 100 Pa\,hPa$^{-1}$ (conversion factor to MKS units) \\
$C_{kg2g}=$\sindex[con]{Ckg2@conversion factor, kg to g}$10^{3}$\,g\,kg$^{-1}$
\sindex[lis]{Cx2y@$C_{x2y}$=conversion factor from x to y}(conversion
factor to give final units of g\,m$^{-3}$)\\
$T_{0}$ = 273.15\,K

\rule[0.5ex]{1\columnwidth}{1pt}
\begin{eqnarray}
\mathrm{RHOx} & = & C_{kg2g}\frac{C_{mb2Pa}\mathrm{\{EWX\}}}{R_{w}\mathrm{(\{ATX\}+\mbox{\ensuremath{T_{0}}})}}\label{eq:8.17RHO-1}
\end{eqnarray}
%
\end{minipage}}\\
For instruments measuring the vapor pressure density (including the
Lyman-alpha\index{hygrometer!Lyman-alpha} probes and the newer version
called the UV hygrometer\index{hygrometer!UV}), the basic measurement
from the instrument is the water vapor density, \textbf{\uline{RHOUV}}\sindex[var]{RHOUV}\index{RHOUV}
or\textbf{ }\textbf{\uline{RHOLA}}\textbf{\sindex[var]{RHOLA}\index{RHOLA}},
determined by applying calibration coefficients to the measured signals
(XUVI\sindex[var]{XUVI} or VLA\sindex[var]{VLA}). In addition, a
slow update to a dew-point measurement is used to compensate for drift
in the calibration. The algorithm for the UV Hygrometer is as described
in the following box; the processing used for early projects with
the Lyman-alpha instruments is similar but more involved and won't
be documented here because the instruments are obsolete. See \href{http://www.eol.ucar.edu/raf/Bulletins/bulletin9.html}{RAF Bulletin 9}
for the processing previously used for archived measurements from
the Lyman-alpha hygrometers.\\

\textbf{Specific Humidity (g/kg): }\textbf{\uline{SPHUM}}\sindex[var]{SPHUM}\index{SPHUM}\\
\index{humidity!specific}\emph{The mass of water vapor per unit mass
of (moist) air, conventionally measured in units of g/kg or per mille.
}\\
\\
\fbox{\begin{minipage}[t]{0.95\textwidth}%
PSXC\index{PSXC} = ambient pressure. hPa \\
EWX\index{EWX} = ambient water vapor pressure, hPa\\
$C_{kg2g}=$$10^{3}$\,g\,kg$^{-1}$ (conversion factor to give
final units of g\,kg$^{-1}$)\\
$M_{w}=$molecular weight of water$^{\dagger}$\\
$M_{d}=$molecular weight of dry air$^{\dagger}$\\
\\
\rule[0.5ex]{1\columnwidth}{1pt}
\begin{eqnarray}
\mathrm{SPHUM} & = & C_{kg2g}\frac{M_{w}}{M_{d}}(\mathrm{\frac{\{EWX\}}{\mathrm{\{PSXC\}-(1-\frac{M_{w}}{M_{d}})\{\mathrm{EWX}\}}}})\label{eq:8.18SPHUM-1}
\end{eqnarray}
%
\end{minipage}}\\
\\

\textbf{Mixing Ratio (g/kg): }\textbf{\uline{\label{Mixing-Ratio-(g/kg):}MR}}\sindex[var]{MR}\textbf{\index{MR},
}\textbf{\uline{MRCR}}\textbf{\sindex[var]{MRCR}\index{MRCR},
}\textbf{\uline{MRLA}}\textbf{\sindex[var]{MRLA}\index{MRLA},
}\textbf{\uline{MRLA1}}\textbf{\index{MRLA1}, }\textbf{\uline{MRLH}}\sindex[var]{MRLH}\index{MRLH}\textbf{,
}\textbf{\uline{MRVXL}}\index{MRVXL}\sindex[var]{MRVXL}\\
\index{water vapor!mixing ratio}\emph{The ratio of the mass of water
to the mass of dry air in the same volume of air, }conventionally
expressed in units of g/kg or per mille. Mixing ratios may be calculated
for the various instruments measuring humidity on the aircraft, and
the variable names reflect the source: MR from the dewpoint \index{hygrometer!dew point}hygrometers,
MRCR from the cryogenic hygrometer\index{hygrometer!cryogenic}, MRLA
from the Lyman-alpha\index{hygrometer!Lyman-alpha} sensor, MRLA1
if there is a second Lyman-alpha sensor, MRLH from a tunable-diode
laser hygrometer\index{hygrometer!tunable diode laser}, and MRVXL
is from the VCSEL\index{hygrometer!VCSEL} hygrometer (also a laser
hygrometer). The example in the box below is for the case of the dewpoint
hygrometers; others are analogous.\\
\\
\fbox{\begin{minipage}[t]{0.95\textwidth}%
EWX\index{EWX} = water vapor pressure, hPa\\
PSXC\index{PSXC} = ambient total pressure, hPa\\
$C_{2kg2g}=$$10^{3}$\,g\,kg$^{-1}$ (conversion factor to give
final units of g\,kg$^{-1}$)\\
$M_{w}=$molecular weight of water$^{\dagger}$\\
$M_{d}=$molecular weight of dry air$^{\dagger}$\\
\\
\rule[0.5ex]{1\columnwidth}{1pt}
\begin{equation}
\mathrm{MR=C_{kg2g}\frac{M_{w}}{M_{d}}\frac{\mathrm{\{EWX\}}}{(\mathrm{\{PSXC\}-\{EWX\})}}}\label{eq:8.19MR-1}
\end{equation}
%
\end{minipage}}\\

\end{hangparagraphs}


\subsection{Derived Thermodynamic Variables\index{derived variables}}
\begin{hangparagraphs}
\textbf{Potential Temperature (K): }\textbf{\uline{THETA}}\sindex[var]{THETA}\index{THETA}\\
\index{temperature!potential}\index{potential temperature|see {temperature, potential}}\emph{The
absolute temperature reached if a dry parcel at the measured pressure
and temperature were to be compressed or expanded adiabatically to
a pressure of 1000 hPa}. It does not take into account the difference
in specific heats caused by the presence of water vapor, and water
vapor can change the exponent in the formula below enough to produce
errors of 1\,K or more.\\
\fbox{\begin{minipage}[t]{0.95\columnwidth}%
ATX\index{ATX} = ambient temperature, $^{\circ}$C\\
PSXC\index{PSXC} = ambient pressure (hPa)\\
$p_{0}$\sindex[lis]{p0@$p_{0}$= reference pressure equal to 1000 hPa}
= reference pressure = 1000 hPa\\
$R_{d}$ = gas constant\index{gas constant!dry air} for dry air$^{\dagger}$\\
$c_{pd}$ = specific heat\index{specific heat!constant pressure}
at constant pressure for dry air$^{\dagger}$\\
\\
\rule[0.5ex]{1\columnwidth}{1pt}
\begin{equation}
\mathrm{THETA}=\left(\mathrm{\{ATX\}}+T_{0}\right)\left(\frac{p_{0}}{\mathrm{\{PSXC\}}}\right)^{R_{d}/c_{pd}}\label{eq:8.13THETA-1}
\end{equation}
 %
\end{minipage}}\\
\\

\textbf{Pseudo-Adiabatic Equivalent Potential Temperature (K): }\textbf{\uline{THETAP}}\sindex[var]{THETAP}\sindex[var]{THETAE}\index{THETAE}\index{THETAE}\\
\emph{The absolute temperature reached if a parcel of air were to
be expanded pseudo-adiabatically (i.e., with immediate removal of
all condensate) to a level where no water vapor remains, after which
the dry parcel would be compressed to 1000 hPa. }Beginning in 2011,
pseudo-adiabatic equivalent potential temperature\index{temperature!pseudo-adiabatic equivalent potential}\sindex[lis]{thetaP@$\Theta_{P}$=temperature, pseudo-adiabatic equivalent potential}
is calculated using the method developed by Davies-Jones\index{Davies-Jones}
(2009).\footnote{Davies-Jones, R., 2009: On formulas for equivalent potential temperature.
\emph{Mon. Wea. Review, }\textbf{137, }3137\textendash 3148.} This is discussed in the memo available at \href{https://drive.google.com/open?id=0B1kIUH45ca5ATVl1MjV6Q0E0YXc}{this URL}.
The following summarizes that study. The Davies-Jones formula is\\
\begin{equation}
\Theta_{P}=\Theta_{DL}\exp\{\frac{r(L_{0}^{*}-L_{1}^{*}(T_{L}-T_{0})+K_{2}r)}{c_{pd}T_{L}}\}\label{eq:DaviesJonesThetaP-1}
\end{equation}
and\\
\begin{equation}
\Theta_{DL}=T_{K}(\frac{p_{0}}{p_{d}})^{0.2854}\,(\frac{T_{k}}{T_{L}})^{0.28\times10^{-3}r}\label{eq:ThetaDL-1}
\end{equation}
\\
where $T_{K}$ is the absolute temperature (in kelvin) at the measurement
level, $p_{d}$\sindex[lis]{pd@$p_{d}$= partial pressure of dry air}
is the partial pressure of dry air at that level, $p_{0}$ is the
reference pressure (conventionally 1000 hPa), $r$\sindex[lis]{r@$r$=water-vapor mixing ratio, dimensionless}
is the (dimensionless) water vapor mixing ratio, $c_{pd}$ the specific
heat of dry air, $T_{L}$\sindex[lis]{temperaturelifted@$T_{L}$= temperature, lifted condensation level}
the temperature at the lifted condensation level\index{lifted condensation level}
(in kelvin), and $T_{0}=273.15\,$K. The coefficients in this formula
are: $L_{0}^{*}=2.56313\times10^{6}\mathrm{{J\,kg^{-1}}}$, $L_{1}^{*}=1754$
J\,kg$^{-1}\,\mathrm{{K}^{-1}}$, and $K_{2}=1.137\times10^{6}$\,J\,kg$^{-1}$.
The asterisks on $L_{0}^{*}$ and $L_{1}^{*}$ indicate that these
coefficients depart from the best estimate of the coefficients that
give the latent heat of vaporization\index{latent heat of vaporization}
of water, but they have been adjusted to optimize the fit to values
obtained by exact integration. Note that, unlike the formula discussed
below that was used prior to 2011, the mixing ratio must be used in
dimensionless form (i.e., kg/kg), \emph{not} with units of g/kg. The
following empirical formula, developed by Bolton\index{Bolton} (1980),\footnote{Bolton, D., 1980: The computation of equivalent potential temperature.
\emph{Mon. Wea. Rev.,} \textbf{108,} 1046\textendash 1053. } is used to determine $T_{L}$:\\
\begin{equation}
T_{L}=\frac{\beta_{1}}{3.5\ln(T_{K}/\beta_{3})-\ln(\mathrm{e/\beta_{4}})+\beta_{5}}+\beta_{2}\label{eq:TLCL-1}
\end{equation}
where $e$ is the water vapor pressure, $\beta_{1}=2840\,K$, $\beta_{2}=55\,$K,
$\beta_{3}=1$\,K, $\beta_{4}=1$\,hPa, $\beta_{5}=-4.805$. (Coefficients
$\beta_{3}$ and $\beta_{4}$ have been introduced into (\ref{eq:TLCL-1})
only to ensure that arguments to logarithms are dimensionless and
to specify the units that must be used to achieve that.)\\
\\
Prior to 2011, the variable called the equivalent potential temperature\footnote{The AMS glossary defines equivalent potential temperature as applying
to the adiabatic process, not the pseudo-adiabatic process; the name
of this variable has therefore been changed.}\index{temperature!equivalent potential} and named THETAE in the
output data files was that obtained using the method of Bolton (1980),
which used the same formula to obtain the temperature at the lifted
condensation level ($T_{L}$) and then used that temperature to find
the value of potential temperature\index{temperature!potential} of
dry air that would result if the parcel were lifted from that point
until all water vapor condensed and was removed from the air parcel.
 The formulas used were as follows:\\
\\
\fbox{\begin{minipage}[t]{0.95\textwidth}%
$T_{L}$= temperature at the lifted condensation level, K\\
ATX\index{ATX} = ambient temperature ($^{\circ}C$)\\
EDPC\index{EDPC} = water vapor pressure (hPa) \textendash{} now superceded
by EWX\\
MR\index{MR} = mixing ratio (g/kg)\\
THETA\index{THETA} = potential temperature (K)\\
\\
\rule[0.5ex]{1\columnwidth}{1pt}
\[
T_{L}=\frac{2840.}{3.5\ln(\mathrm{\{ATX\}+T_{0}})-\ln(\mathrm{\{EDPC\}})-4.805}+55.
\]
\begin{equation}
\mathrm{THETAE}=\mathrm{\{THETA\}\left(\frac{3.376}{T_{L}}-0.00254\right)(\{MR\})(1+0.00081(\{MR\}))}\label{eq:8.13THETAE-1}
\end{equation}
%
\end{minipage}}\textbf{}\\
Differences vs the new formula are usually minor but can be of order
0.5\,K.\textbf{}\\

\textbf{Virtual Temperature ($\text{�}$C): }\textbf{\uline{\label{TVIR}TVIR}}\sindex[var]{TVIR}\index{TVIR}\\
\index{temperature!virtual}\emph{The temperature of dry air having
the same pressure and density as the air being sampled.} The virtual
temperature thus adjusts for the buoyancy added by water vapor. \\
\fbox{\begin{minipage}[t]{0.95\columnwidth}%
ATX\index{ATX} = ambient temperature, $^{\circ}C$\\
$r$ = mixing ratio, dimensionless {[}kg/kg{]} = \{MR\}/(1000 g/kg)\\
$T_{0}=273.15$\,K\\
\\
\\
\rule[0.5ex]{1\columnwidth}{1pt}

\begin{equation}
\mathrm{TVIR}=(\mathrm{\{ATX\}}+T_{0})\left(\frac{1+\frac{M_{d}}{M_{w}}r}{1+r}\right)-T_{0}\label{eq:814TVIR-1}
\end{equation}
 %
\end{minipage}}\\
\\

\textbf{Virtual Potential Temperature (K): }\textbf{\uline{THETAV}}\sindex[var]{THETAV}\index{THETAV}\\
\index{temperature!virtual potential}\emph{A potential temperature
analogous to the conventional potential temperature except that it
is based on virtual temperature instead of ambient temperature.} Dry-adiabatic
expansion or compression to the reference level (1000 hPa) is assumed.
As for THETA, use of dry-air values for the gas constant and specific
heat at constant pressure can lead to significant errors in humid
conditions. For further information, see \href{https://drive.google.com/open?id=0B1kIUH45ca5AZXZZbWJSbEctdmM}{this note}.\\
\fbox{\begin{minipage}[t]{0.95\columnwidth}%
TVIR\index{TVIR} = virtual temperature, $^{\circ}C$\\
PSXC\index{PSXC} = ambient pressure, hPa\\
$R_{d}=$gas constant for dry air$^{\dagger}$\\
$c_{pd}=$specific heat at constant pressure for dry air$^{\dagger}$\\
$T_{0}=273.15$\,K\\
$p_{0}$ = reference pressure, conventionally 1000 hPa

\rule[0.5ex]{1\columnwidth}{1pt}
\begin{equation}
\mathrm{THETAV}=\left(\mathrm{\{TVIR\}}+T_{0}\right)\left(\frac{p_{o}}{\mathrm{\{PSXC\}}}\right)^{R_{d}/c_{pd}}\label{eq:8.15THETAV-1}
\end{equation}
 %
\end{minipage}}\href{file:https://drive.google.com/open?id=0B1kIUH45ca5AZXZZbWJSbEctdmM}{this note}\\

\textbf{Wet-Equivalent Potential Temperature (K): }\textbf{\uline{THETAQ}}\index{THETAQ}\sindex[var]{THETAQ}\\
\emph{The absolute temperature reached if a parcel of air were to
be expanded adiabatically (i.e., retaining the condensed water in
the liquid phase and accounting for the specific heat of that condensate)
to a level where no water vapor remains, after which the condensate
would be removed and the resulting dry parcel compressed to 1000 hPa.
}This variable was not included in data archives prior to 2012. Emanuel
(1994) gives the following formula (his Eq. 4.5.11):\sindex[lis]{temperatureweteq@$\Theta_{q}=$temperature, wet-equivalent potential}\emph{
}\\
\begin{equation}
\Theta_{q}=T(\frac{p_{0}}{p_{d}})^{\frac{R_{d}}{c_{pt}}}\exp\left\{ \frac{L_{v}r}{c_{pt}T}\right\} \left(\frac{e}{e_{s,w}(T)}\right)^{-rR_{w}/c_{pt}}\label{eq:ThetaQEquation}
\end{equation}
where $\Theta_{q}$ is the wet-equivalent potential temperature, $L_{v}$
the latent heat of vaporization\sindex[lis]{Lv@$L_{v}$=latent heat of vaporization of water}\index{latent heat of vaporization},
$r$ the (dimensionless) water-vapor mixing ratio\sindex[lis]{r@$r$=water-vapor mixing ratio, dimensionless},
$c_{pt}=c_{pd}+r_{t}c_{w}$ where $r_{t}$ is the total-water mixing
ratio including vapor and condensate, $c_{w}$ is the specific heat
of liquid water, and other symbols are as used previously. See \href{https://drive.google.com/open?id=0B1kIUH45ca5ATVl1MjV6Q0E0YXc}{this memo}
for additional discussion of this variable, for values to use for
the latent heat and specific heat, and in particular for analysis
indicating that $\Theta_{q}$ evaluated with this formula can be expected
to vary from the true adiabatic value by a few tenths kelvin (in a
worst case, by about 1 K) because of variation in (and uncertainty
in) the specific heat of supercooled water at low temperature. The
details of the calculation are described in the following box. Note
that this algorithm only uses the liquid water content as measured
by a King probe, PLWCC; other similar calculations could be based
on other measures of liquid water such as that from a cloud-droplet
spectrometer.\\
\fbox{\begin{minipage}[t]{1\columnwidth - 2\fboxsep - 2\fboxrule}%
$e=$\{EDPC\}{*}100 = water vapor pressure (Pa)\\
ATX\index{ATX} = ambient temperature ($^{\circ}C$)\\
$r=$\{MR\}/1000.\index{MR} = mixing ratio (dimensionless)\\
$p_{d}=$(\{PSXC\}$-$\{EDPC\}){*}100 = ambient dry-air pressure (Pa)
\\
$p_{0}=$reference pressure for potential temperature, 10$^{5}$Pa
\\
$\chi=$\{PLWCC\}/1000.=cloud liquid water content\sindex[lis]{chi@$\chi$=liquid water content}
(kg\,m$^{-3}$)\\
$R_{d}=$gas constant for dry air$^{\dagger}$\\
$\rho_{d}=$density of dry air = $\frac{p_{d}}{R_{d}(\{ATX\}+T_{0})}$\\
$c_{pd}=$specific heat of dry air$^{\dagger}$\\
$c_{w}=$specific heat of liquid water$^{\dagger}$\\
$L_{V}=L_{0}+L_{1}\mathrm{\{ATX\}}$ where $L_{0}=2.501\times10^{6}\mathrm{J}\,\mathrm{kg^{-1}}$
and $L_{1}=-2370\,\mathrm{J\,\mathrm{kg^{-1}\,\mathrm{K^{-1}}}}$\\
\\
\rule[0.5ex]{1\columnwidth}{1pt}\\
\[
r_{t}=r+(\chi/\rho_{d})
\]
\[
c_{pt}=c_{pd}+r_{t}c_{w}
\]

If outside cloud or below 100\% relative humidity, define

\[
F_{1}=\left(\frac{e}{e_{s,w}(T)}\right)^{-\frac{rR_{w}}{c_{pt}}}\,\,\,\,\,\,,
\]

otherwise set $F_{1}=1$. 

\[
T_{1}=\mathrm{(\{ATX\}}+T_{0})\left\{ \frac{p_{0}}{(\mathrm{\{PSXC\}}-\mathrm{\{EDPC\})}}\right\} ^{\frac{R_{d}}{c_{pt}}}
\]
\[
\mathrm{THETAQ}=T_{1}F_{1}\exp\left\{ \frac{L_{v}r}{c_{pt}(\{\mathrm{ATX\}}+T_{0})}\right\} 
\]
%
\end{minipage}}

\end{hangparagraphs}





\subsection{Wind\label{sec:WIND}}

\href{http://www.eol.ucar.edu/raf/Bulletins/bulletin23.html}{RAF Bulletin 23}\index{Bulletin 23}
documents the calculation of wind components, both with respect to
the earth (UI, VI, WI, WS and WD) and with respect to the aircraft
(UX and VY). In data processing, a separate function (GUSTO in GENPRO,
gust.c in NIMBUS) is used to derive these wind components. That function
uses the measurements from an Inertial Navigation System (INS\index{INS})
as well as aircraft true airspeed, aircraft angle of attack, and aircraft
sideslip angle. The wind components calculated in GUSTO/gust.c are
used to derive the wind direction (WD) and wind speed (WS). Additional
variables UIC, VIC, WSC, WDC, UXC, and VYC are also calculated based
on the variables VNSC, VEWC discussed in section \ref{subsec:IRS/GPS},
which combine INS and GPS information to obtain improved measurements
of the aircraft motion. Those are usually the highest-quality measurements
of wind because the merged INS/GPS variables combine the high-frequency
response of the INS with the long-term accuracy of the GPS.

There is an extensive discussion of the wind-sensing system and the
uncertainties associated with measurements of wind in this \href{http://dx.doi.org/10.5065/D60G3HJ8}{Technical Note}.\index{uncertainty!wind!Technical Note}
The details contained therein and in Bulletin 23 will not be repeated
here, so those documents should be consulted for additional information.
There are two exceptions that are discussed in more detail here:
\begin{enumerate}
\item The calculation of vertical wind is described in more detail below
for the variables WI and WIC. 
\item Because measurements obtained by a GPS receiver are often used, the
motion of the GPS receiving antenna\index{motion!GPS antenna} relative
to the IRU must be considered. Standard processing corrects for the
motion of the gust system relative to the IRU arising from aircraft
rotation, but a similar correction is needed because the GPS antenna
is displaced from the IRU. The displacement is almost entirely along
the longitudinal axis of the aircraft, so GPS-measured velocities
like GGVNS, GGVEW, and GGVSPD (denoted here $v_{n}$, $v_{e}$, and
$v_{u}$) are corrected as follows to give measurements that apply
at the location of the IRU. Then these variables can be used in place
of or to complement similar measurements from the IRU in the processing
algorithms. The equations are:\\
\begin{equation}
\delta v_{u}{\rm =-L_{G}\dot{\theta}}\label{eq:pitchr}
\end{equation}
\begin{eqnarray}
\delta v_{e} & = & -L_{G}\dot{\psi}\,{\rm \cos\psi}\label{eq:GGVEW}\\
\delta v_{n} & = & L_{G}\dot{\psi}{\rm \,\sin\psi}\label{eq:GGVNS}
\end{eqnarray}
where $L_{G}$ is the distance\sindex[lis]{Lg@$L_{G}$=distance from IRU to GPS antenna}
forward along the longitudinal axis from the IRU to the GPS antenna
($-4.30$\,m for the GV, where the GPS receiver is behind the IRU)
and where $\theta$ and $\psi$ respectively represent the pitch and
heading angle. The dots over the attitude-angle symbols represent
time derivatives, so for example $\dot{\theta}$ is the rate of change
of the pitch angle and all angles are expressed in radians. The correction
terms are added to the GPS-measured velocity components so that they
represent the motion of the IRU relative to the Earth.
\end{enumerate}
The variables pertaining to the relative wind are described in the
next subsection, and the variables characterizing the wind are then
described briefly in the last subsection. Some additional detail is
included in cases where procedures are not documented in that earlier
bulletin.

\subsubsection{Relative Wind}

Wind\index{wind!relative} is measured by adding two vectors: (1)
the measured air motion relative to the aircraft (called the relative
wind), and (2) the motion of the aircraft relative to the Earth. The
following are the measurements used to determine the relative wind.
The motion of the aircraft relative to the ground was discussed in
Section \ref{sec:INS}, and the combination of these two vectors to
measure the wind is described in \href{http://www.eol.ucar.edu/raf/Bulletins/bulletin23.html}{RAF Bulletin 23}. 

RAF uses the radome\label{radome gust-sensing system} gust-sensing\index{radome gust probe}\index{gust probe!radome}
technique\footnote{Brown, E.~N, C.~A.~Friehe, and D.~H.~Lenschow, 1983:\emph{ Journal
of Climate and Applied Meteorology,} \textbf{22, }171\textendash 180} to measure incidence angles of the relative wind (i.e., angles of
attack and sideslip). The pressure difference between sensing ports
above and below the center line of the radome is used, along with
the dynamic pressure\index{dynamic pressure} measured at a pitot
tube and referenced to the static pressure source, to determine the
angle of attack. The sideslip angle is determined similarly using
the pressure ports on the starboard and port sides of the radome.
A Rosemount Model 858AJ gust probe\index{gust probe!858AJ} has occasionally
been used for specialized measurements. The radome measurements are
made by differential pressure sensors located in the nose area of
the aircraft and connected to the radome by semi-rigid tubing.
\begin{hangparagraphs}
\textbf{Mach Number (dimensionless):}\nop{MACHx}{}\nop{MACHX}{}\textbf{
}\textbf{\uline{MACHx}}\textbf{, }\textbf{\uline{MACHX}}\textbf{\sindex[var]{MACHx}}\index{MACHx}\index{MACHX}\sindex[var]{MACHX}\\
\emph{The Mach Number that characterizes the flight speed.} The Mach
number is defined as the ratio of the flight speed (or the magnitude
of the relative wind) to the speed of sound. See Eq.~(\ref{eq:8.8-1})
in Section \ref{sec:State Variables} for the equation used. Many
archived data files have a variable XMACH2\index{XMACH2}\sindex[var]{XMACH2},
which is the square of MACHx.

\textbf{Aircraft True Airspeed (m/s):}\nop{TASx}{}\nop{TASxD}{}\nop{TASX}{}\textbf{
}\textbf{\uline{TASx}}\index{TASx}\textbf{, }\textbf{\uline{TASxD}}\textbf{\sindex[var]{TASxD}}\index{TASxD}\\
\emph{The flight speed of the aircraft relative to the atmosphere.}
This derived measurement of the flight speed\index{flight speed|see {true airspeed}}
of the aircraft relative to the atmosphere is based on the Mach number
calculated from both the dynamic pressure at location x and the static
pressure\index{pressure!ambient}. See the derivation for ATx \vpageref{ambient temperature and TAS calculation}.
The different variables for TASx (TASF, TASR, etc) use different measurements
of QCxC in the calculation of Mach number. The variable TASxD is the
result of calculations for which the Mach number, air temperature,
and true airspeed are determined for dry instead of humid air. See
the discussion of ATX on page \ref{ATX discussion} for an explanation
of how humidity is handled in the calculation of true airspeed.\\
\\
\fbox{\begin{minipage}[t]{0.95\textwidth}%
(see box for ATx\index{ATx} and MACHx\index{MACHx})\\
Note dependence of MACHx on choices for QCxC\index{QCxC} and PSXC\index{PSXC}\\
TASx\index{TASx} depends on QCxC, PSXC\index{PSXC}, ATX\index{ATX}\\
~~~~~where PSXC and ATX are the preferred choices\\
$\gamma^{\prime}$, $R^{\prime}$, and $T_{0}$: See the List of Symbols\\
\rule[0.5ex]{1\columnwidth}{1pt}

\begin{equation}
\mathrm{TASx}=\mathrm{\{MACHx\}}\sqrt{\gamma^{\prime}R^{\prime}\mathrm{\,(\{ATX\}}+T_{0})}\label{eq:8.10-1}
\end{equation}
%
\end{minipage}}\\
\\

\textbf{Aircraft True Airspeed (Humidity Corrected) (m/s):}\nop{TASHC}{}\textbf{
}\textbf{\uline{TASHC}}\sindex[var]{TASHC}\index{TASHC} \textendash{}
\emph{obsolete}\\
 This derived measurement of \index{airspeed}true airspeed accounted
for deviations of specific heats of moist air from those of dry air.
See List, 1971, pp 295, 331-339, and Khelif, et al., 1999. The equation
used for this variable, given by Khelif et al.~1999,\footnote{Khelif, D., S.P. Burns, and C.A. Friehe, 1999: Improved wind measurements
on research aircraft. \emph{Journal of Atmospheric and Oceanic Technology,}
\textbf{16,} 860\textendash 875.} added a moisture correction to the true airspeed derived for dry
air, as follows:\\
\\
 %
\fbox{\begin{minipage}[t]{0.95\columnwidth}%
$q$ = specific humidity (dimensionless) = SPHUM/1000.\index{SPHUM}
\index{humidity!specific} for SPHUM expressed in g/kg\\
$c=0.000304\,\mathrm{kg\,g^{-1}}=0.304$ (dimensionless)\\
\rule[0.5ex]{1\columnwidth}{1pt}
\begin{lyxcode}
TASHC~=~TASX~{*}~(1.0~+~$c$~{*}~$q$)
\end{lyxcode}
%
\end{minipage}}\\

\textbf{Attack Angle Differential Pressure (mb):}\nop{ADIFR}{}\textbf{
}\textbf{\uline{ADIFR}}\sindex[var]{ADIFR}\index{ADIFR}\\
\emph{The pressure difference between the top and bottom pressure
ports of a radome gust-sensing system. }\index{attack, angle of}This
measurement is used to determine the angle of attack; see AKRD below.\index{radome gust probe}\index{gust probe!radome}\textbf{
}Obsolete variable \uline{ADIF}\index{ADIF} is a similar variable
used for old gust-boom systems or for Rosemount Model 858AJ flow-angle
sensors.\index{gust probe!Rosemount 858AJ}

\textbf{Sideslip Angle Differential Pressure (mb):}\nop{BDIFR}{}\textbf{
}\textbf{\uline{BDIFR}}\sindex[var]{BDIFR}\index{BDIFR}\\
\emph{The pressure difference between starboard and port pressure
inlets of a radome gust-sensing system. }\index{sideslip, angle of}\index{radome gust probe}
This measurement is used to determine the sideslip angle; see SSRD
below. Obsolete variable \uline{BDIF}\index{BDIF} is a similar
variable used for old gust-boom systems or for Rosemount Model 858AJ
flow-angle sensors.

\textbf{Attack Angle, Radome (}$\text{�}$\textbf{):}\nop{AKRD}{}\textbf{
}\textbf{\uline{AKRD}}\sindex[var]{AKRD}\index{AKRD}\\
\emph{The angle of attack of the aircraft. }This derived measurement
represents the angle between the relative wind vector and the plane
determined by the longitudinal and lateral axes of the aircraft. Positive
values indicate flow moving upward relative to that plane.\index{attack, angle of}
The calculation is based on ADIFR\index{ADIFR} and a measurement
of dynamic pressure, and so is the measurement produced by a radome
gust-sensing system. Empirical sensitivity coefficients for each aircraft,
determined from special flight maneuvers, are used; see \href{http://www.eol.ucar.edu/raf/Bulletins/bulletin23.html}{RAF Bulletin 23}\index{Bulletin 23}
and this \href{http://dx.doi.org/10.5065/D60G3HJ8}{Technical Note}
for more information.\index{sensitivity coefficient!ADIFR} The sensitivity
coefficients listed below have changed when the radomes were changed
or refurbished, so the project documentation should be consulted for
the values used in a particular project. For more information on the
latest C-130 calibration, see \href{https://drive.google.com/open?id=0B1kIUH45ca5AN2VYdFF2V0tHcVU}{this note}.\\
\fbox{\begin{minipage}[t]{0.95\textwidth}%
ADIFR\index{ADIFR} = attack differential pressure, radome (hPa)\\
QCXC\index{QCXC} = reference dynamic pressure (hPa)\\
MACHX\index{MACHX} = reference Mach number\\
$e_{0},\,e{}_{1},\,e_{2}$ = \sindex[lis]{e02@$e_{0-2}$=sensitivity coefficients, angle of attack}sensitivity
coefficients determined empirically; typically:\\
~~~~~~~~~~%
\begin{minipage}[t]{0.8\textwidth}%
\begin{quote}
\{5.776$\,[^{\circ}]$, 15.031$\,[^{\circ}]$, 0\} for the C-130\\
\footnote{prior to Jan 2012, when the radome was changed: \{5.516, 19.07, 2.08\}}
\{4.605$\,[^{\circ}]$, 18.44$\,[^{\circ}]$, 6.75$\,[^{\circ}]$\}
for the GV\\
\end{quote}
%
\end{minipage}

\rule[0.5ex]{1\linewidth}{1pt}

\begin{equation}
\mathrm{AKRD}=e_{0}+\frac{\{\mathrm{ADIFR}\}}{\{\mathrm{QCF}\}}\left(e_{1}+e{}_{2}\mathrm{\{MACHX\}}\right)\label{eq:AKRDeq}
\end{equation}
%
\end{minipage}}

\textbf{}%
\begin{comment}
Old Bulletin-9 code: omitted now

\textbf{Attack Angle, 858 (}$\text{�}$\textbf{): }\textbf{\uline{AKDF}}\index{AKDF}\\
This is a derived output of the aircraft's angle of attack obtained
from ADIF (the vertical differential pressure measured by a Rosemount
858AJ flow-angle sensor) and a dynamic pressure (Qc) using an empirically
determined sensitivity function. For the Rosemount 858AJ, the sensitivity
function (GR) is a constant 0.079 for Mach numbers less than 0.515.
For larger Mach numbers the sensitivity decreases as a function of
Mach number.\\
\fbox{\begin{minipage}[t]{0.9\textwidth}%
MACH = Aircraft Mach Number\\
GR = sensitivity function (given below)\\
Qc = dynamic pressure, mb

\begin{eqnarray*}
\mathrm{for\,MACH} & \leq & 0.515:\\
 &  & \mathrm{GR}=0.079\\
\mathrm{otherwise}:\\
 & \mathrm{GR} & =0.086577797-0.03560256\,MACH\\
 &  & +0.00006143\,MACH^{2}
\end{eqnarray*}

\rule[[0.5ex]]{1.0\linewidth}{1pt}

\[
\mathrm{AKDF=\frac{ADIF}{GR\,Qc}}
\]
%
\end{minipage}}
\end{comment}

\textbf{Reference Attack Angle (}$\text{�}$\textbf{): }\textbf{\uline{ATTACK}}\sindex[var]{ATTACK}\index{ATTACK}\label{punch:4-12}\\
\emph{The reference angle of attack used to calculate derived variables.
}This variable is the reference selected from other measurements of
angle of attack in the data set. In most projects, it is equal to
AKRD\index{AKRD}. It is used where attack angle is needed for other
derived calculations (e.g., wind measurements).

\textbf{Sideslip Angle (Differential Pressure) (}$\text{�}$\textbf{):
}\textbf{\uline{SSRD}}\sindex[var]{SSRD}\index{SSRD}\\
\emph{The angle of sideslip of the aircraft. }This derived measurement
represents the angle between the longitudinal axis of the aircraft
and the projection of the relative wind onto the plane determined
by the longitudinal and lateral axes. Positive values indicate airflow
from the starboard side. This variable is derived from BDIFR\index{BDIFR}
and a dynamic pressure\index{pressure!dynamic} using a sensitivity
function that has been determined empirically for each aircraft. \\
\fbox{\begin{minipage}[t]{0.9\textwidth}%
BDIFR = differential pressure between sideslip pressure ports, radome
(mb)\\
QCXC = dynamic pressure (mb)\\
$s_{0},\,s{}_{1}$ = \sindex[lis]{s0@$s_{0,1}$=sensitivity coefficients, sideslip}empirical
coefficients dependent on the aircraft and radome configuration\\
\hspace*{1cm} ~ = \{-0.000983, (1/0.08189) $\text{�}$ \} for the
C-130\\
\hspace*{1cm} ~ = \{-0.0025, (1/0.04727) $\text{�}$\} for the GV

\rule[0.5ex]{1\linewidth}{1pt}

\[
\mathrm{SSRD=}s_{1}(\frac{\{BDIFR\}}{\{\mathrm{QCXC}\}}+s_{0})
\]
%
\end{minipage}}

\textbf{}%
\begin{comment}
The following is from old Bulletin 9:

\textbf{Sideslip Angle (Differential Pressure) (}$\text{�}$\textbf{):
}\textbf{\uline{SSDF}}\index{SSDF}\\
This variable is derived from BDIF and a dynamic pressure using an
empirically determined sensitivity function, as expressed in the following:\\
\fbox{\begin{minipage}[t]{0.9\textwidth}%
$M$ = Aircraft Mach Number\\
BDIF = differential pressure between sideslip pressure ports, 858
probe (mb)\\
QC = dynamic pressure from the 858 probe (mb)\\
$g_{r}$ = dimensionless sensitivity function (given below)
\begin{eqnarray*}
\mathrm{for\,M} & \leq & 0.515:\\
 & g_{r} & =0.079\\
\mathrm{otherwise}:\\
 & g_{r} & =0.086577797-0.03560256\,M\\
 &  & +0.00006143\,M^{2}
\end{eqnarray*}

\rule[0.5ex]{1\linewidth}{1pt}

\[
\mathrm{SSDF=}\frac{\mathrm{\{BDIF\}}}{g_{r}\mathrm{\{QC\}}}
\]
%
\end{minipage}}
\end{comment}

\textbf{Reference Sideslip Angle (}$\text{�}$\textbf{):}\nop{SSLIP}{}\textbf{
}\textbf{\uline{SSLIP}}\sindex[var]{SSLIP}\index{SSLIP}\\
\emph{The reference sideslip angle used to calculate derived variables}.
This variable is the reference selected from other measurements of
sideslip angle\index{sideslip, angle of} in the data set. In most
projects, it is equal to SSRD\index{SSRD}. It is used where sideslip
angle is needed for other derived calculations (e.g., wind measurements). 

\end{hangparagraphs}


\subsubsection{Wind Components and the Wind Vector}
\begin{hangparagraphs}
\textbf{}%
\noindent\begin{minipage}[t]{1\columnwidth}%
\begin{hangparagraphs}
\textbf{Wind Vector East Component (m/s):}\nop{UI}{}\textbf{ }\textbf{\uline{UI\sindex[var]{UI}\index{UI}}}

\textbf{Wind Vector North Component (m/s):}\nop{VI}{}\textbf{ }\textbf{\uline{VI\sindex[var]{VI}\index{VI}}}

\textbf{Wind Vector Vertical Component (m/s):}\nop{WI}{}\textbf{
}\textbf{\uline{WI\sindex[var]{WI}\index{WI}}}
\end{hangparagraphs}

%
\end{minipage}\textbf{\uline{}}\\
\textbf{\uline{}}\\
\emph{The three-dimensional wind\index{wind}\index{wind!vector}\index{wind!components}
vector with respect to the earth, }as determined from the inertial
reference systems. UI is the east-west component with positive values
\uline{toward} the east, VI is the north-south component with positive
values \uline{toward} the north,\index{wind!sign convention} and
WI is the vertical component with positive values toward the zenith.\\
\\
The calculation of WI differs from the description in Bulletin 23
because the output from the inertial reference system is different
for the modern units now in use. The vertical wind is the sum of the
vertical gust component (represented approximately by TASX\,sin(ATTACK-PITCH))
and the motion of the aircraft as measured by VSPD (discussed in Section
\mbox{\pageref{sec:INS}}). Bulletin 23 describes the historical calculation
of the vertical motion of the aircraft via a barometric-inertial feedback
loop, but equivalent calculations (including pressure damping to the
pressure altitude) are incorporated into current IRS units so VSPD
already is the product of such a calculation. To calculate WI, VSPD
is therefore used in place of the obsolete variable WP3 that was discussed
in Bulletin 23.\\
\\
WIC should usually be used instead of WI because VSPD, entering WI,
is updated to the pressure altitude and so can have false variations
in baroclinic conditions. \\

\textbf{}%
\noindent\begin{minipage}[t]{1\columnwidth}%
\begin{hangparagraphs}
\textbf{Wind Speed (m/s):}\nop{WS}{}\textbf{ }\textbf{\uline{WS}}\textbf{\sindex[var]{WS}\index{WS}}

\textbf{Wind Direction (}$\text{�}$\textbf{):}\nop{WD}{}\textbf{
}\textbf{\uline{WD}}\sindex[var]{WD}\index{WD}
\end{hangparagraphs}

%
\end{minipage}\\
\\
\emph{The magnitude and direction of the horizontal wind. }These variables\index{wind!speed}\index{wind!direction}
are obtained in a straightforward manner from UI and VI. The resulting
wind direction is relative to true north and represents the direction
\uline{from which} the wind blows. That is the reason that 180$^{\circ}$
appears in the following algorithm.\\
\\
\fbox{\begin{minipage}[t]{0.95\textwidth}%
UI = easterly component of the horizontal wind\\
VI = northerly component of the horizontal wind\\
atan2 = 4-quadrant arc-tangent function producing output in radians
from -$\pi$ to $\pi$\\
$C_{rd}$\sindex[con]{Crd@$C_{rd}$= conversion factor, radians to degrees (180/$\pi$)}
= conversion factor, radians to degrees, = 180/$\pi$ {[}units: $^{\circ}/radian${]}\\
\rule[0.5ex]{1\linewidth}{1pt} 
\begin{eqnarray}
\mathrm{WS} & = & \sqrt{\mathrm{\{UI\}}^{2}+\{\mathrm{VI\}}^{2}}\nonumber \\
\mathrm{WD} & = & C_{rd}\mathrm{\,atan2(\{UI\},\,\{VI\})}+180^{\circ}\label{eq:9.1}
\end{eqnarray}
%
\end{minipage}}\\

\textbf{}%
\noindent\begin{minipage}[t]{1\columnwidth}%
\begin{hangparagraphs}
\textbf{Wind Vector Longitudinal Component (m/s):}\nop{UX}{}\textbf{
}\textbf{\uline{UX\sindex[var]{UX}\index{UX}}}

\textbf{Wind Vector Lateral Component (m/s):}\nop{VY}{}\textbf{ }\textbf{\uline{VY}}\sindex[var]{VY}\index{VY}
\end{hangparagraphs}

%
\end{minipage}\textbf{}\\
\textbf{}\\
\emph{The horizontal wind}\index{wind!vector}\emph{ vector relative
to the frame of reference attached to the aircraft.} UX is parallel
to the longitudinal axis and positive toward the nose.\index{wind!longitudinal}\index{wind!lateral}
VY is along the lateral axis and normal to the longitudinal axis;
positive is toward the port (or left) wing.\\

\textbf{}%
\noindent\begin{minipage}[t]{1\columnwidth}%
\begin{hangparagraphs}
\textbf{GPS-Corrected Wind Vector, East Component (m/s):}\nop{UIC}{}\textbf{
}\textbf{\uline{UIC\sindex[var]{UIC}\index{UIC}}}

\textbf{GPS-Corrected Wind Vector, North Component (m/s):}\nop{VIC}{}\textbf{
}\textbf{\uline{VIC}}\sindex[var]{VIC}\index{VIC}
\end{hangparagraphs}

%
\end{minipage}\\
\\
\index{wind!GPS-corrected} \emph{The horizontal wind}\index{wind!vector}
c\emph{omponents respectively }\emph{\uline{toward}}\emph{ the
east and }\emph{\uline{toward}}\emph{ the north.} They are derived
from measurements from an inertial reference unit (IRU) and a Global
Positioning System (GPS), as described in the discussion of VEW and
VNS above. They are calculated just as for UX and VY except that the
GPS-corrected values for the aircraft groundspeed are used in place
of the IRU-based values. They are considered ``corrected'' from
the original measurements from the IRU or GPS, as described in section
\ref{subsec:IRS/GPS}.\\

\textbf{Wind Vector, Vertical Component (m/s):}\nop{WIC}{}\textbf{
}\textbf{\uline{WIC}}\sindex[var]{WIC}\index{WIC}\\
\emph{The component of the wind in the vertical direction. }This is
the standard calculation of vertical wind\index{wind!vertical}, obtained
from the difference between the measured vertical component of the
relative wind and the vertical motion of the aircraft (usually GGVSPD
in recent projects).\label{punch:4-13}\textbf{} This should be used
in preference to WI if the latter is present; see the discussion of
WP3 in section \ref{sec:INS}. Positive values are toward the zenith.\\

\textbf{}%
\noindent\begin{minipage}[t]{1\columnwidth}%
\begin{hangparagraphs}
\textbf{GPS-Corrected Wind Direction (}$^{\circ}$\textbf{):}\nop{WDC}{}\textbf{
}\textbf{\uline{WDC}}\sindex[var]{WDC}\index{WDC}

\textbf{GPS-Corrected Wind Speed (m/s):}\nop{WSC}{}\textbf{ }\textbf{\uline{WSC\sindex[var]{WSC}}}\index{WSC}
\end{hangparagraphs}

%
\end{minipage}\\
\\
\emph{The direction and magnitude of the wind vector, }obtained by
combining measurements from GPS and IRU units. These variables\index{wind!GPS-corrected}
are obtained in a straightforward manner from UIC and VIC, using equations
analogous to (\ref{eq:9.1}) but with UIC and VIC as input measurements.
They are expected to be the preferred measurements of wind because
they combine the best features of the IRU and GPS measurements.\\

\textbf{}%
\noindent\begin{minipage}[t]{1\columnwidth}%
\begin{hangparagraphs}
\textbf{GPS-Corrected Wind Vector, Longitudinal Component (m/s):}\nop{UXC}{}\textbf{
}\textbf{\uline{UXC}}\sindex[var]{UXC}\textbf{\uline{\index{UXC}}}

\textbf{GPS-Corrected Wind Vector, Lateral Component (m/s):}\nop{VYC}{}\textbf{
}\textbf{\uline{VYC}}\sindex[var]{VYC}\index{VYC}
\end{hangparagraphs}

%
\end{minipage}\\
\\
\emph{The longitudinal and lateral components of the three-dimensional
wind}\index{wind!lateral component}\index{wind!longitudinal component}\emph{,
similar to UX and VY, but corrected by the complementary-filter algorithm
that combines IRU and GPS measurements}. See the discussion in Section
\ref{subsec:IRS/GPS}. The components UXC and VYC are toward the front
of the aircraft and toward the port (left) wing, respectively. 
\end{hangparagraphs}


\subsection{Special-Use Remote Sensors}

The above variables are normally included in the archived netCDF files
from projects, but there are a few remote sensors that provide additional
state-parameter measurements in some projects. These include:\label{subsec:MTP}

\noindent\fbox{\begin{minipage}[t]{1\columnwidth - 2\fboxsep - 2\fboxrule}%
\begin{itemize}
\item Microwave Temperature Profiler (\href{http://www.eol.ucar.edu/instruments/microwave-temperature-profiler}{MTP})
\textendash{} remotely sensed temperature profiles\index{MTP}
\item Dropsonde System (\href{http://www.eol.ucar.edu/instruments/avaps}{AVAPS})
\textendash{} profiles of temperature, humidity, and wind vs pressure.\index{dropsonde}
\item GPS-Occultation Sensor (\href{http://www.eol.ucar.edu/instruments/gnss-instrument-system-multi-static-and-occultation-sensing}{GISMOS})
\textendash{} atmospheric soundings of refractivity via GPS occultation.\index{GISMOS}
\end{itemize}
%
\end{minipage}} 

The links provided connect to descriptions of these instruments on
the EOL web site, and each provides a summary of how data are acquired
and processed. These measurements are not normally part of the archived
netCDF project files. Those interested in using these measurements
should contact EOL data management (\url{mailto:raf-dm@eol.ucar.edu})
for access to the measurements and for information on how the measurements
are processed.



\section{CLOUD PHYSICS VARIABLES}

\subsection{Measurements of Liquid Water Content}
\begin{hangparagraphs}
\textbf{Raw Output PMS/CSIRO (King) Liquid Water Content} (W):\nop{PLWC}\nop{PLWC1}{}\textbf{
}\textbf{\uline{PLWC}}\sindex[var]{PLWC}\textbf{\index{PLWC},
}\textbf{\uline{PLWC1}}\index{PLWC1}\sindex[var]{PLWC1}\label{punch:5-3}\\
\emph{The power\index{King probe} dissipated by the sensor of a PMS/CSIRO
(King) liquid water\index{liquid water content} probe (in watts).
}PLWC is the power required to maintain constant temperature in a
heated element as that element is cooled by convection and evaporation
of impinging liquid water. The convective heat losses are determined
by calibration in dry air over a range of airspeeds and temperatures,
so that the remaining power can be related to the liquid water content.
The instrument is described in \href{http://www.eol.ucar.edu/raf/Bulletins/bulletin24.html}{RAF Bulletin 24}\index{Bulletin 24}
and at \href{http://www.eol.ucar.edu/instruments/king-csiro-liquid-water-sensor}{this URL}.
See \nameref{PLWCC} (which follows) for processing.\\

\textbf{\label{PLWCC}PMS/CSIRO (King) Liquid Water Content} (g/m$^{3}$):\nop{PLWCC}{}\nop{PLWCC1}{}\textbf{
}\textbf{\uline{PLWCC}}\sindex[var]{PLWCC}\textbf{\index{PLWCC},
}\textbf{\uline{PLWCC1}}\index{PLWCC1}\sindex[var]{PLWC1}\\
\emph{The liquid water content}\index{liquid water content!King probe}
\emph{measured by a King probe.} This is calculated by relating the
power consumption required to maintain a constant temperature to the
liquid water content, taking into account the effect of convective
heat losses. The instrument and processing are described by King et
al. (1978)\footnote{King, W. D., D. A. Parkin and R. J. Handsworth, 1978 A hot-wire liquid
water device having fully calculable response characteristics. J.
Appl. Meteorol., 17, 1809--1813. See also Bradley, S. G., and W.
D. King, 1979 Frequency response of the CSIRO Liquid Water Probe.
J. Appl. Meteorol., 18, 361--366.} and in a note available at \href{https://www.eol.ucar.edu/system/files/PLWCCrev140214.pdf}{this URL}.
Because the temperature of the sensing wire is typically well above
the boiling point of water\index{water!boiling point}, the assumption
made in processing is that the water collected on the sensing wire
is vaporized at the boiling point $T_{b}$. The boiling point is represented
as a function of pressure as described below. \\
\\
\\
\\
{[}see the algorithmn box on the next page{]}\\
\vfill

~~\\
\fbox{\begin{minipage}[t]{0.95\textwidth}%
PLWC = total power dissipated by the probe (W)\\
$P_{D}$ = power dissipated\index{King probe!power dissipated}\sindex[lis]{Power@$P$=power}
by the cooling effect of dry air alone\\
$P_{W}$ = power needed to heat and vaporize the liquid water that
hits the probe element\\
$L$ = length\sindex[lis]{L@$L$ =length (of a King-probe element)}
of the probe sensitive element\index{King probe!element dimensions},
typically 0.021\,m\\
$d$= diameter\sindex[lis]{d@$d$=diameter} of the probe sensitive
element, typically 1.805$\times10^{-3}$m\\
$T_{s}$= sensor temperature\index{King probe!sensor temperature}\sindex[lis]{Ts@$T_{s}$=temperature of a sensor}
($^{\circ}$C)\\
$T_{a}$= ambient temperature ($^{\circ}$C) = ATX\\
$T_{b}$\sindex[lis]{Tb@$T_{b}$= boiling temperature of water} =
boiling temperature of water (dependent on pressure): 

~~~~~~~~~with $x=\log_{10}(p/(1$hPa)), $B=1^{\circ}C$,
and \{$b_{0},$ $b_{1}$, $b_{2}$, $b_{3}$\} = \{0.03366503, 1.34236135,
-0.33479451, 0.0351934\}: $T_{b}=B\times10^{(b_{0}+b_{1}x+b_{2}x^{2}+b_{3}x^{3})}$\\
$T_{m}=(T_{a}+T_{s})/2$ = mean temperature for air properties\\
$L_{v}(T_{b})$ = latent heat of vaporization of water\sindex[lis]{Lv@$L_{v}$=latent heat of vaporization of water}\index{latent heat of vaporization}
= (2.501-0.00237$T_{b}$)$\times10^{6}$J\,kg$^{-1}$\\
$c_{w}$\sindex[lis]{cw@$c_{w}$= specific heat of liquid water} =
specific heat of water = 4190~J\,kg$^{-1}$K$^{-1}$ (mean value,
0--90$^{\circ}$C)\\
$U_{a}$ = true airspeed (m/s) = TASX\\
$\lambda_{c}$\sindex[lis]{lambdac@$\lambda_{c}$= thermal conductivity, dry air}
= thermal conductivity\index{thermal conductivity|see {conductivity, thermal}}\index{conductivity!thermal}
of dry air (2.38+0.0071$T_{m}$)$\times10^{-2}$J\,m$^{-1}$s$^{-1}$K$^{-1}$\\
$\mu$\sindex[lis]{mu_{a}= dynamic viscosity of air@$\mu_{a}$= dynamic viscosity of air}
=viscosity of air =\index{viscosity} (1.718+0.0049$T_{m})\times10^{-5}$
kg\,~m$^{-1}$s$^{-1}$\\
$\rho_{a}$\sindex[lis]{rhoa@$\rho_{a}$= density of air} = density
of air = $p/(R_{d}(T_{a}+T_{0}))$\\
Re\sindex[lis]{Re= Reynolds number} = Reynolds number = $\rho_{a}U_{a}d/\mu_{a}$\\
Nu\sindex[lis]{Nu= Nusselt number} = Nusselt Number relating conduction
heat loss to the total heat loss for dry air: 

~~~~~~~~~~%
\fbox{\begin{minipage}[t]{0.9\textwidth}%
typically Nu=$a_{0}\mathrm{Re}^{a_{1}}$ where, for the GV, $\{a_{0},\,a_{1}\}=1.868,\,0.343$
for Re<7244 and $\{0.135,\,0.638\}$ otherwise, except when TASX <
150 m/s; then use $\{0.133,\,0.382\}$. For the C-130 $\{a_{0},\,a_{1}\}=\{0.118,\,0.675\}$.%
\end{minipage}}, \\
$C_{kg2g}=1000$\sindex[lis]{Ckg2g@$C_{kg2g}$= conversion factor, g to kg}
= grams per kilogram 

~~~~~~~~~~(unit conversion to conventional units for liquid
water content)\\
$\chi$ \sindex[lis]{chi@$\chi$=liquid water content}= liquid water
content (g/m$^{3}$) = PLWCC

\rule[0.5ex]{1\linewidth}{1pt}
\[
\mathrm{PLWC}=P_{D}+P_{W}
\]
where
\[
P_{D}=\pi\mathrm{Nu}\,L\lambda_{c}(T_{s}-T_{a})
\]
\[
P_{W}=L\,d[L_{v}(T_{b})+c_{w}(T_{b}-T_{a})]\,U_{a}\chi
\]
Result:
\[
\mathrm{PLWCC}=\chi=\frac{C_{kg2g}(\mathrm{\{PLWC\}}-P_{D})}{L\,d\,U_{a}[L_{v}(T_{b})+c_{w}(T_{b}-T_{a})]}
\]
%
\end{minipage}}\\
\\
In addition, a processing step is used to remove drift by calculating
the offset required to zero measurements obtained outside cloud. This
is done by adjusting the coefficient $a_{0}$ by nudging toward the
value required to give zero liquid water content outside cloud (as
indicated by another instrument, often a CDP showing droplet concentration
of <1 cm$^{-3}$). Specifically, when out-of-cloud, Nu$^{\prime}$
is calculated from Nu$^{\prime}=\mathrm{\{PLWC\}}/(\pi L\lambda_{c}(T_{s}-T_{a}))$.
Then the value of $a_{0}$ is updated via $a_{0}$ += (Nu$^{\prime}/Re^{a_{1}}-a_{0})/\tau$
(using, for the GV, separate coefficients for each of the three branches).
In this formula, $\tau$ should be the number of updates in a fixed
period, e.g., for a 100 s time constant and for 25-Hz processing,
$\tau=100\times25$. In addition, to avoid jumps when switching among
the branches, the linear coefficients \{$a_{0}$\} are adjusted with
each transition between branches to provide a continuous estimate
of the zero value.

\textbf{PVM-100 Liquid Water Content} (g/m$^{-3}$):\nop{PLWCG}{}\textbf{
}\textbf{\uline{PLWCG}}\textbf{\sindex[var]{PLWCG}}\index{PLWCG}\\
\emph{Cloud liquid water content for cloud droplets in the approximate
size range from 3--50 $\mu$m. }The PVM produces a measure of the
liquid water content directly, but a baseline value is sometimes subtracted
by reference to another cloud droplet instrument such as an FSSP or
CDP, such that when the other instrument measures a very low droplet
concentration the baseline value for the PVM-100 is updated at the
corresponding time and that average is then subtracted from the measurements
directly produced by the PVM-100.\label{punch:5-1}

\textbf{Rosemount Icing Detector Signal} (V):\nop{RICE}{}\textbf{
}\textbf{\uline{RICE}}\sindex[var]{RICE}\index{RICE}\\
\emph{The voltage related to loading on the element of a Rosemount
871F\index{Rosemount 871F icing probe} ice-accretion probe.} This
instrument (see \href{http://www.eol.ucar.edu/instruments/rosemount-icing-probe}{this URL})
consists of a rod set in vibration by a piezoelectric crystal. The
oscillation frequency of the probe changes with ice loading, so in
supercooled cloud ice accumulates on the sensor and the change in
oscillation frequency is transmitted as a DC voltage. When the rod
loads to a trigger point, the probe heats the rod to remove the ice.
The rate of voltage change can be converted to an estimate of the
supercooled liquid water content\index{liquid water content!supercooled},
as described in connection with the obsolete variable SCLWC. This
calculation is no longer provided routinely but can be duplicated
by a user on the basis of the SCLWC algorithm (see page \pageref{SCLWC})
or one of several other published algorithms.
\end{hangparagraphs}


\subsection{Sensors of Individual Particles (1-D Probes)\label{subsec:1DProbes}}

The RAF operates a set of hydrometeor detectors\index{hydrometeor detector}
that provide single-dimension measurements (i.e., not images) of individual
particle sizes. \href{http://www.eol.ucar.edu/raf/Bulletins/bulletin24.html}{RAF Bulletin 24}
contains extensive information on the operating principles and characteristics
of some of the older instruments. Here the focus will be on the meanings
of the variables in the archived data files.\index{Bulletin 24}\label{punch:5-2}

\label{VariableNames1DProbes}Four- and five-character variable names\index{variable names!hydrometeor probes}
shown in this section are generic. The actual names appearing in NIMBUS-generated
production output data sets have appended to them an underscore (\_)
and three or four more characters that indicate a probe's specific
aircraft mounting location. For example, AFSSP\_RPI refers to a variable
from an FSSP-100 probe mounted on the inboard, right-side pod. The
codes presently in use are given in the following table. For the GV,
there are 12 locations available, characterized by three letters.
The first is the wing (\{L,R\} for \{port,starboard\}), the second
is the pylon (\{I,M,O\} for inboard, middle, outboard\}), the third
is which of the two possible canister locations at the pylon is used
(\{I,O\} for \{inboard, outboard\}).

\begin{center}
\begin{tabular}{|c|c|c|}
\hline 
\textbf{Code} & \textbf{Location} & \textbf{Aircraft}\tabularnewline
\hline 
\hline 
OBL  & Outboard Left  & C-130Q \tabularnewline
\hline 
IBL  & Inboard Left  & C-130Q \tabularnewline
\hline 
OBR  & Outboard Right  & C-130Q \tabularnewline
\hline 
IBR  & Inboard Right  & C-130Q \tabularnewline
\hline 
LPO  & Left Pod Outboard  & C-130Q \tabularnewline
\hline 
LPI  & Left Pod Inboard  & C-130Q \tabularnewline
\hline 
LPC  & Left Pod Center  & C-130Q \tabularnewline
\hline 
RPO  & Right Pod Outboard  & C-130Q \tabularnewline
\hline 
RPI  & Right Pod Inboard  & C-130Q \tabularnewline
\hline 
RPC  & Right Pod Center  & C-130Q \tabularnewline
\hline 
OBL  & Left Wing  & Electra \tabularnewline
\hline 
IBL  & Left Pylon  & Electra \tabularnewline
\hline 
WDL  & Window Left  & Electra \tabularnewline
\hline 
OBR  & Right Wing  & Electra \tabularnewline
\hline 
IBR  & Right Pylon  & Electra \tabularnewline
\hline 
WDR  & Window Right  & Electra \tabularnewline
\hline 
\{L,R\}W\{I,M,O\}\{I,O\} & see discussion above & GV\tabularnewline
\hline 
\end{tabular}
\par\end{center}

The probe type also is coded into each variable's name, sometimes
using four characters, sometimes only one: FSSP-100 (FSSP or F), FSSP-300
(F300 or 3), CDP (CDP or D), UHSAS (UHSAS or U), PCASP (PCAS or P),
OAP-200X (200X or X), OAP-260X (260X or 6) and OAP-200Y (200Y or Y).
Prefix letters are used to identify the type of measurement (A=accumulated
particle counts per time interval per channel, C = concentration per
channel, CONC = Concentration from all channels, DBAR = mean diameter,
DISP = dispersion, PLWC =liquid water content, DBZ = radar reflectivity
factor).

Some of the probes discussed in this section are primarily aerosol
spectrometers but are described here, rather than in Section \ref{sec:AEROSOL-PARTICLE-MEASUREMENTS},
because they are similar to the hydrometeor spectrometers and so are
most economically discussed here. However, see Sect.~\ref{sec:AEROSOL-PARTICLE-MEASUREMENTS}
for the processing algorithms that lead to concentrations from the
UHSAS and PCASP/SPP-200. The following table and discussion includes
some obsolete variables (for the 200X and 200Y) for the same reason.
The table also includes some variables derived from imaging spectrometers
(the 2DC and 2DP probes) to highlight that the primary variables are
similar to those discussed in this sub-section. Those variables are
discussed in the next sub-section. In two cases, the FSSP and PCASP,
two versions are listed, an obsolete version and a current version
with a revised processing package (SPP-100 for the FSSP, SPP-200 for
the PCASP). Both are included for historical completeness, but algorithms
in this document discuss the current versions.

The archived data files sometimes have ``housekeeping'' variables\index{housekeeping variables}
included that provide information on the operating state and data
quality from the probes. For example, the CDP provides information
on the average transit time, the voltage from the nominal 5-V source,
the control board temperature, the laser block temperature, the laser
current, the laser power monitor, the qualifier bandwidth, the qualifier
baseline, the qualifier threshold, the sizer baseline, the wing-board
temperature, an A-to-D overflow flag, and a count of particles rejected
as being outside the depth of field. The netCDF variables and attributes
should be consulted for this housekeeping information. The large number
of housekeeping variables will not be included in this document, so
appropriate manuals and the netCDF files should be consulted when
interpreting this information.

\index{hydrometeor probes!table of}

\begin{center}
\noindent\begin{minipage}[t]{1\columnwidth}%
\begin{center}
\textbf{Probes that produce size distributions of particles (with
links to descriptions):\label{TableOfProbes}}
\par\end{center}
\begin{center}
\begin{tabular}{|c|c|c|c|c|c|c|}
\hline 
\textbf{\small{}Generic Name} &  & \textbf{\small{}Probe} & \textbf{\small{}Channels} & \textbf{\small{}Usable}\footnote{Channels may be unusable because the first channel is a historical
carry-over and should be ignored, or because in the case of 2D probes
the entire-in sizing technique reduces the number of bins where particles
can be sized. Also, when some channels have been considered unreliable
the netCDF header may specify that the usable bins are smaller than
indicated here.} & \textbf{\small{}Diameter Range} & \textbf{\small{}Bin Width}\tabularnewline
\hline 
\hline 
{\small{}FSSP-100/original} & {\small{}F} & {\small{}\href{http://www.eol.ucar.edu/raf/Bulletins/B24/fssp100.html}{FSSP-100}}\footnote{Now obsolete but present in many archived data sets} & {\small{}0--15} & {\small{}1-{}-15} & \multicolumn{2}{c|}{{\small{}(See FRNG below)}}\tabularnewline
\hline 
FSSP/SPP-100 & F & \href{http://www.eol.ucar.edu/instruments/forward-scattering-spectrometer-probe-model-100}{SPP-100} & 0--30 & 1-{}-30 & \multicolumn{1}{c||}{3--45 $\mu$m } & 3 $\mu$m (typ.)\tabularnewline
\hline 
UHSAS & U & \href{https://www.eol.ucar.edu/instruments/ultra-high-sensitivity-aerosol-spectrometer}{UHSAS} & 0--99 & 1-{}-99 & \multicolumn{1}{c||}{0.06--1.0 $\mu$m} & variable\tabularnewline
\hline 
CDP & D & \href{https://www.eol.ucar.edu/instruments/cloud-droplet-probe}{CDP} & 0--30 & 1---30 & \multicolumn{1}{c||}{2.0--50} & variable\tabularnewline
\hline 
{\small{}F300} & {\small{}3} & {\small{}\href{https://www.eol.ucar.edu/instruments/forward-scattering-spectrometer-probe-model-300}{FSSP-300}$^{b}$} & {\small{}0--30} & {\small{}1-{}-30} & {\small{}0.3--20.0 $\mu m$} & {\small{}variable}\tabularnewline
\hline 
PCASP/original & P & \href{http://www.eol.ucar.edu/raf/Bulletins/B24/pcasp100.html}{PCASP}$^{b}$ & 0--15 & 1-{}-15 & 0.1--3.0 $\mu$m & variable\tabularnewline
\hline 
PCASP/SPP-200 & P & {\small{}\href{https://www.eol.ucar.edu/instruments/signal-processing-package-200-passive-cavity-aerosol-spectrometer-probe}{SPP-200}} & {\small{}0--30} & {\small{}1-{}-30} & {\small{}0.1--3.0 $\mu m$} & {\small{}variable}\tabularnewline
\hline 
{\small{}200X} & {\small{}X} & OAP-200X{\small{}$^{b}$} & {\small{}0--15} & {\small{}1-{}-15} & {\small{}40--280 $\mu m$} & {\small{}10 $\mu m$}\tabularnewline
\hline 
{\small{}260X} & {\small{}6} & \href{http://www.eol.ucar.edu/raf/Bulletins/B24/260X.html}{OAP-260X} & {\small{}0-63} & {\small{}3-{}-62} & {\small{}40-620 $\mu m$} & {\small{}10 $\mu m$}\tabularnewline
\hline 
{\small{}200Y} & {\small{}Y} & {\small{}OAP-200Y$^{b}$} & {\small{}0-15} & {\small{}1-{}-15} & {\small{}300--4500 $\mu m$} & {\small{}300 $\mu m$}\tabularnewline
\hline 
1DC\footnote{See p.~\pageref{Despite-the-'1D'} for an explanation of this name
convention}  &  & \href{http://www.eol.ucar.edu/raf/Bulletins/B24/2dProbes.html}{2DC}$^{b,}$\footnote{Measurements from this and the next three 2D probes are discussed
in section \vref{subsec:Hydrometeor-Imaging-Probes}} (old) & 0-32 & 1-30$^{e}$ & 25--800 $\mu$m  & 25 $\mu$m\tabularnewline
\hline 
1DP &  & \href{http://www.eol.ucar.edu/raf/Bulletins/B24/2dProbes.html}{2DP}$^{b}$
(old) & 0-32 & 1-30 & 200--6400 $\mu$m & 200 $\mu$m\tabularnewline
\hline 
1DC  &  & \href{https://www.eol.ucar.edu/instruments/two-dimensional-optical-array-cloud-probe}{2DC}
(fast) & 0-63 & 1-62\footnote{Some of the lowest channels are often considered unreliable and excluded
in processing} & 25--1600 $\mu$m & 25 $\mu$m\tabularnewline
\hline 
1DP &  & \href{https://www.eol.ucar.edu/instruments/two-dimensional-optical-array-precipitation-probe}{2DP}
(new) & 0-63 & 1-62 & 100--6400 $\mu$m & 100 $\mu$m\tabularnewline
\hline 
\end{tabular}
\par\end{center}%
\end{minipage}\\
\par\end{center}
\begin{hangparagraphs}
\textbf{Count Rate Per Channel }(number per time interval):\textbf{\uline{\label{punch:5-4}\label{punch:5-7}}}\nop{AFSSP}{}\nop{AS200}{}\nop{AS100}{}\nop{AF300}{}\nop{APCAS}{}\nop{A200X}{}\nop{A260X}{}\nop{A200Y}{}\nop{ACDP}{}\nop{AUHSAS}{}\textbf{\uline{}}\\
\textbf{\uline{AFSSP}}\textbf{\sindex[var]{AFSSP}\index{AFSSP},
}\textbf{\uline{AS100\index{AS100}}}\sindex[var]{AS100}\textbf{\uline{,
AF300}}\sindex[var]{AF300}\textbf{\index{AF300}, }\textbf{\uline{AS200}}\index{AS200}\sindex[var]{AS200},\textbf{
}\textbf{\uline{APCAS}}\textbf{\sindex[var]{APCAS}\index{APCAS},
}\textbf{\uline{A200X}}\textbf{\sindex[var]{A200X}\index{A200X},
}\textbf{\uline{A260X}}\textbf{\sindex[var]{A260X}\index{A260X},
}\textbf{\uline{A200Y}}\sindex[var]{A200Y}\index{A200Y}, \textbf{\uline{ACDP}}\sindex[var]{ACDP}\index{ACDP},
\textbf{\uline{AUHSAS\sindex[var]{AUHSAS}\index{AUHSAS}}}\\
\emph{The size distribution of the number of particles detected by
a 1D hydrometeor probe per unit time.} These measurements have ``vector''
character in the NetCDF\index{NetCDF!vector} output files, with dimension
equal to the number of channels in the table above and with one entry
per channel. The first element in the vector is a historical remnant
from a time when housekeeping information was stored here and should
be ignored. For the size limits of the channels, see the netCDF attributes
of the following variables for ``Size Distribution''.

\textbf{}%
\noindent\begin{minipage}[t]{1\columnwidth}%
\textbf{Size Distribution} ($\mathrm{cm}{}^{-3}$channel$^{-1}$)\textbf{:}\nop{CFSSP}{}\nop{CS100}{}\nop{CF300}{}\nop{CPCAS}{}\nop{CS200}{}\nop{CCDP}{}\nop{CUHSAS}{}\nop{C200X}{}\nop{C260X}{}\nop{C200Y}{}\textbf{
}\textbf{\uline{CFSSP}}\textbf{\index{CFSSP}\sindex[var]{CFSSP},
}\textbf{\uline{CS100}}\textbf{\index{CS100}}\sindex[var]{CS100}\textbf{,
}\textbf{\uline{CF300}}\textbf{\index{CF300}\sindex[var]{CF300},
}\textbf{\uline{CS200}}\index{CS200}\sindex[var]{CS200},\textbf{
}\textbf{\uline{CPCAS\index{CPCAS}}}\textbf{\sindex[var]{CPCAS},
}\textbf{\uline{CCDP\index{CCDP}\sindex[var]{CCDP}}}\textbf{,
}\textbf{\uline{CUHSAS\index{CUHSAS}\sindex[var]{CUHSAS}}}\\
\textbf{Size Distribution }(L\textbf{$^{-1}$}channel$^{-1}$):\textbf{
}\textbf{\uline{C200X}}\textbf{\index{C200X}\sindex[var]{C200X},
}\textbf{\uline{C260X}}\textbf{\index{C260X}\sindex[var]{C260X},
}\textbf{\uline{C200Y\index{C200Y}\sindex[var]{C200Y}}}%
\end{minipage}\textbf{\label{CUHSAS}}\\
\emph{The particle concentrations}\index{concentration!hydrometeor}\index{concentration!hydrometeor, size distribution}\index{FSSP-100!size distribution}
\emph{in each usable bin of the probe.} These netCDF variables have
``vector'' character with dimension equal to the number of channels
in the table above. The first vector member should be ignored. For
some scattering spectrometer\index{spectrometer!hydrometeor}\index{hydrometeor spectrometer}
probes (FSSP-100, FSSP-300\index{FSSP-300}, PCASP\index{PCASP})
the concentration value is modified by the probe activity (FACT, PACT)
as described below. The concentration is obtained from the total number
of particles detected and a calculated, probe-dependent sample volume
that is specified in recent projects by attributes (e.g., depth of
field and beam diameter) of this variable in the netCDF file. For
additional details, see the links in the table \vpageref{TableOfProbes}
or, for older probes, \href{http://www.eol.ucar.edu/raf/Bulletins/bulletin24.html}{RAF Bulletin 24}\index{Bulletin 24}.

\begin{singlespace}
\textbf{}%
\noindent\begin{minipage}[t]{1\columnwidth}%
\begin{hangparagraphs}
\begin{singlespace}
\textbf{Concentration }(cm$^{-3}$):\nop{CONCF}{}\nop{CONC3}{}\nop{COMCP}{}\nop{CONCD}{}\nop{CONCU}{}\textbf{
}\textbf{\uline{CONCF}}\textbf{\sindex[var]{CONCF}\index{CONCF},
}\textbf{\uline{CONC3}}\sindex[var]{CONC3}\textbf{\index{CONC3},
}\textbf{\uline{CONCP\sindex[var]{CONCP}\index{CONCP}, CONCD\sindex[var]{CONCD}\index{CONCD}}}\textbf{,
}\textbf{\uline{CONCU\sindex[var]{CONCU}\index{CONCU}}}

\textbf{Concentration (}L\textbf{$^{-1}$):}\nop{CONCX}{}\nop{CONC6}{}\nop{CONCY}{}\textbf{
}\textbf{\uline{CONCX}}\textbf{\sindex[var]{CONCX}\index{CONCX},
}\textbf{\uline{CONC6}}\sindex[var]{CONC6}\textbf{\index{CONC6},
}\textbf{\uline{CONCY\sindex[var]{CONCY}\index{CONCY}}}
\end{singlespace}
\end{hangparagraphs}

%
\end{minipage}\textbf{}\\
\emph{The particle concentrations}\index{FSSP-100!concentration}\index{concentration!FSSP}
\emph{summed over all channels to give the total concentration in
the size range of the probe.} For example, \{CONCF\} = $\sum_{i}\mathrm{\{CFSSP\}_{i}}$.
For additional details, see \href{http://www.eol.ucar.edu/raf/Bulletins/bulletin24.html}{RAF Bulletin 24}.\\
\end{singlespace}

\textbf{Mean Diameter }($\mu m$):\nop{DBARF}{}\nop{DBAR3}{}\nop{DBARP}{}\nop{DBARX}{}\nop{DBAR6}{}\nop{DBARY}{}\nop{DBARD}{}\nop{DBARU}{}\textbf{
}\textbf{\uline{DBARF}}\textbf{\index{DBARF}, }\textbf{\uline{DBAR3}}\textbf{\index{DBAR3},
}\textbf{\uline{DBARP}}\textbf{\index{DBARP}, }\textbf{\uline{DBARX}}\textbf{\index{DBARX},
}\textbf{\uline{DBAR6}}\textbf{\index{DBAR6}, }\textbf{\uline{DBARY}}\index{DBARY},
\textbf{\uline{DBARD}}\index{DBARD}, \textbf{\uline{}}\\
\textbf{\uline{DBARU}}\index{DBARU}\\
\emph{The arithmetic average of all measured particle diameters from
a particular probe.}\index{diameter!mean, 1D probes}\index{FSSP-100!mean diameter}
This mean is calculated as follows: \\
\\
\fbox{\begin{minipage}[t]{0.9\textwidth}%
\{Cy$_{i}$\} = concentration\sindex[lis]{Cyi@$Cy_{i}$= concentration from hydrometeor probe y in channel i}
from probe y in channel i\label{MeanDiameter}\\
\hspace*{0.6cm}(e.g., y=FSSP to calculate DBARF)\\
i1 = lowest usable channel for the probe\\
i2 = highest usable channel for the probe\\
$d_{i}$\sindex[lis]{di@$d_{i}$= diameter of hydrometeor in channel $i$}
= diameter of particles in channel i for this probe ($\mu m$)\\
\hspace*{0.6cm}(calculated as the average of the lower and upper
size limits for the channel)\\
\rule[0.5ex]{1\linewidth}{1pt}

\begin{equation}
\mathrm{\{DBARx\}}=\frac{{\textstyle \sum_{i=i1}^{i2}}{\displaystyle {\displaystyle \left\{ \mathrm{Cy}_{i}\right\} d_{i}}}}{\sum_{i=i1}^{i2}\left\{ \mathrm{Cy}_{i}\right\} }\label{eq:dbar1d}
\end{equation}
%
\end{minipage}}\\

\textbf{Dispersion }(dimensionless)\textbf{:}\nop{DISPF}{}\nop{DISP3}{}\nop{DISPP}{}\nop{DISPX}{}\nop{DISP6}{}\nop{DISPY}{}\nop{DISPD}{}\nop{DISPU}{}\textbf{
}\textbf{\uline{DISPF}}\textbf{\sindex[var]{DISPF}\index{DISPF},
}\textbf{\uline{DISP3}}\textbf{\sindex[var]{DISP3}\index{DISP3},
}\textbf{\uline{DISPP}}\textbf{\sindex[var]{DISPP}\index{DISPP},
}\textbf{\uline{DISPX}}\textbf{\sindex[var]{DISPX}\index{DISPX},
}\textbf{\uline{DISP6}}\textbf{\sindex[var]{DISP6}\index{DISP6},
}\textbf{\uline{DISPY}}\sindex[var]{DISPY}\index{DISPY}, \textbf{\uline{DISPD}}\sindex[var]{DISPD}\index{DISPD},
\textbf{\uline{}}\\
\textbf{\uline{DISPU}}\sindex[var]{DISPU}\index{DISPU}\\
\index{dispersion}\emph{The ratio of the standard deviation of particle
diameters to the mean particle diameter.}\index{FSSP-100!dispersion}\index{dispersion!1D probes}
\\
\\
\fbox{\begin{minipage}[t]{0.9\textwidth}%
\{DBARx\} = mean particle diameter ($\mu m$)\\
\{Cy$_{i}$\}, i1, i2, $d_{i}$ as for mean diameter \vpageref{MeanDiameter}\\
\rule[0.5ex]{1\linewidth}{1pt}

\begin{equation}
\mathrm{\{DISPx\}}==\frac{1}{\{\mathrm{DBARx}\}}\,\left\{ \frac{{\textstyle \sum_{i=i1}^{i2}}{\displaystyle {\displaystyle \left\{ \mathrm{Cy}_{i}\right\} d_{i}^{2}}}}{\sum_{i=i1}^{i2}\left\{ \mathrm{Cy}_{i}\right\} }-\{\mathrm{DBARx}\}^{2}\right\} ^{1/2}\label{eq:disp1d}
\end{equation}
%
\end{minipage}}

\textbf{Liquid Water Content }(g\,m\textbf{$^{-3}$}):\nop{PLWCF}{}\nop{PLWCX}{}\nop{PLWC6}{}\nop{PLWCY}{}\nop{PLWCD}{}\textbf{
}\textbf{\uline{PLWCF}}\textbf{\sindex[var]{PLWCF}\index{PLWCF},
}\textbf{\uline{PLWCX}}\textbf{\sindex[var]{PLWCX}\index{PLWCX},
}\textbf{\uline{PLWC6}}\textbf{\sindex[var]{PLWC6}\index{PLWC6},
}\textbf{\uline{PLWCY}}\sindex[var]{PLWCY}\index{PLWCY}, \textbf{\uline{PLWCD}}\sindex[var]{PLWCD}\index{PLWCD}\\
\index{liquid water content!1D probes}\emph{The density of liquid
water represented by the size distribution measured by a hydrometeor
probe.}\index{liquid water content}\index{FSSP-100!liquid water content}\index{liquid water content!FSSP-100}
These variables are calculated from the measured concentration (CONCx)
and the third moment of the particle diameter, with the assumption
that the particle is a water drop. The following box describes the
calculation in terms of an equivalent droplet diameter\index{diameter!equivalent},
the diameter that represents the mass in the detected particle. The
equivalent droplet diameter is normally the measured diameter for
liquid hydrometeors, but some processing has used other assumptions
and this is a choice that can be made based on project needs. Using
this definition allows for the approximate estimation of \index{ice water content}ice
water content in cases where it is known that all hydrometeors are
ice. \\
:\\
\fbox{\begin{minipage}[t]{0.9\textwidth}%
$d_{e,i}$\sindex[lis]{dei@$d_{e,i}$= equivalent melted diameter for channel i of a hydrometeor
spectrometer} = equivalent melted diameter for channel $i$ of probe x\\
\{Cy$_{i}$\}, i1, i2 as for mean diameter \vpageref{MeanDiameter}\\
$\varrho_{w}$\sindex[lis]{rhow@$\rho_{w}$= density of liquid water}
= density of water ( $10^{3}kg/m^{3}$)\\
\rule[0.5ex]{1\linewidth}{1pt}
\begin{equation}
\mathrm{\{PLWCx\}}=\frac{\pi\varrho_{w}}{6}{\textstyle \sum_{i=i1}^{i2}}{\displaystyle {\displaystyle \left\{ \mathrm{Cy}_{i}\right\} d_{e,i}^{3}}}\label{eq:lwc1d}
\end{equation}
(units and a scale factor are selected so that the output variable
is in units of g\,m$^{-3}$)%
\end{minipage}}\\

\textbf{Radar Reflectivity Factor }(dbZ)\textbf{:}\nop{DBZ}{}\textbf{
}\textbf{\uline{DBZF}}\textbf{\sindex[var]{DBZF}\index{DBZF},
}\textbf{\uline{DBZX}}\textbf{\sindex[var]{DBZX}\index{DBZX},
}\textbf{\uline{DBZ6}}\textbf{\sindex[var]{DBZ6}\index{DBZ6},
}\textbf{\uline{DBZY}}\sindex[var]{DBZY}\index{DBZY}, \textbf{\uline{DBZD\sindex[var]{DBZD}\index{DBZD}}}\\
\emph{The radar reflectivity factor}\index{reflectivity factor}\index{dBz}\index{radar reflectivity factor!1D probes}
\emph{calculated from the measured size distribution from a hydrometeor
probe.} This is calculated from the measured concentration and the
sixth moment of the size distribution, with the assumption that the
particles are water drops. An equivalent radar reflectivity factor
can be calculated from the hydrometeor size distribution if another
assumption is made about composition of the particles, but this variable
is not part of normal data files. The radar reflectivity factor is
a characteristic only of the hydrometeor size distribution; it is
\emph{not }a measure of radar reflectivity, because the latter also
depends on wavelength, dielectric constant, and other characteristics
of the hydrometeors. The normally used radar reflectivity factor is
measured on a logarithmic scale that depends on a particular choice
of units, so (although it is not conventional) an appropriate scale
factor $Z_{r}$ is included in the following equation to satisfy the
convention that arguments of logarithms should be dimensionless\index{dimensions in equations}.
\\
\\
\fbox{\begin{minipage}[t]{0.9\textwidth}%
$d_{i}$ = diameter for channel $i$ of probe x\\
\{Cy$_{i}$\}, i1, and i2 as for mean diameter \vpageref{MeanDiameter}\\
$Z_{r}$ \sindex[lis]{Zr@$Z_{r}$= scale factor for calculation of the radar reflectivity
factor}= reference factor for units = 1~mm$^{6}$m$^{-3}$\\
\rule[0.5ex]{1\linewidth}{1pt}

\begin{equation}
\mathrm{\{DBZx\}}=10\log_{10}\left({\textstyle \frac{1}{Z_{r}}\sum_{i=i1}^{i2}}{\displaystyle {\displaystyle \left\{ \mathrm{Cy}_{i}\right\} d_{i}^{6}}}\right)\label{eq:dbz1d}
\end{equation}
%
\end{minipage}}

\label{punch:5-5}

\textbf{Effective Radius }($\mu$m):\nop{REFFD}{}\nop{REFFF}{} \textbf{\uline{REFFD}}\index{REFFD}\sindex[var]{REFFD}\textbf{\uline{,
REFFF}}\index{REFFF}\sindex[var]{REFFF}\\
\index{radius, effective}\emph{One-half the ratio of the third moment
of the diameter measurements to the second moment.} This variable
is useful in some calculations that relate the liquid water content
of a cloud layer to its optical properties.\\
\fbox{\begin{minipage}[t]{0.9\textwidth}%
$d_{i}$ = diameter for channel $i$ of probe x\\
\{Cy$_{i}$\}, i1, and i2 as for mean diameter \vpageref{MeanDiameter}\\
\rule[0.5ex]{1\linewidth}{1pt}

\begin{equation}
\mathrm{\{REFFx\}}=\frac{1}{2}\frac{\sum{\displaystyle {\displaystyle \left\{ \mathrm{Cy}_{i}\right\} d_{i}^{3}}}}{\sum{\displaystyle {\displaystyle \left\{ \mathrm{Cy}_{i}\right\} d_{i}^{2}}}}\label{eq:reff1d}
\end{equation}
%
\end{minipage}}

\textbf{FSSP-100 Range }(dimensionless):\nop{FRNG}{}\nop{FRANGE}{}\textbf{
}\textbf{\uline{FRNG}}\textbf{\sindex[var]{FRNG}\index{FRNG},
}\textbf{\uline{FRANGE}}\sindex[var]{FRANGE}\index{FRANGE}\\
\emph{The size range in use for the FSSP-100}\index{FSSP-100!range}\index{range!FSSP}
\emph{probe}.

\begin{minipage}[t]{0.9\textwidth}%
\hspace*{0.7in}%
\begin{tabular}{|c|c|c|}
\hline 
Range & \textbf{Nominal Size Range} & \textbf{Nominal Bin Width}\tabularnewline
\hline 
\hline 
0 & 2--47 $\mu m$ & 3 $\mu m$\tabularnewline
\hline 
1 & 2--32 $\mu m$ & 2 $\mu m$\tabularnewline
\hline 
2 & 1--15 $\mu m$ & 1 $\mu m$\tabularnewline
\hline 
3 & 0.5--7.5 $\mu m$ & 0.5 $\mu m$\tabularnewline
\hline 
\end{tabular}%
\end{minipage}\\
\\
In recent NETCDF data files, the actual bin boundaries used for processing
are recorded in the header. That header should be consulted because
processing sometimes uses non-standard sizes selected to adjust for
Mie scattering, which causes departures from the nominal linear bins.
Recent projects have all used range 0, but other choices have been
made in some older projects and other ranges are still available to
future projects.
\end{hangparagraphs}


\subsection{Hydrometeor Imaging Probes\label{subsec:Hydrometeor-Imaging-Probes}}

Instruments used to obtain hydrometeor images include the two-dimensional
imaging probes (\href{https://www.eol.ucar.edu/instruments/two-dimensional-optical-array-cloud-probe}{2DC}
and \href{https://www.eol.ucar.edu/instruments/two-dimensional-optical-array-precipitation-probe}{2DP})
and some others that require special processing and separate data
records. The former are described in this subsection. The latter include
a three-view cloud particle imager (\href{https://www.eol.ucar.edu/instruments/three-view-cloud-particle-imager}{3V-CPI}),
a small ice detector (\href{https://www.eol.ucar.edu/instruments/small-ice-detector-version-2-hiaper}{SID-2H}),
and a holographic imager (\href{https://www.eol.ucar.edu/instruments/holographic-detector-clouds}{HOLODEC}).
For information regarding use of data from the latter set of instruments,
consult EOL/RAF data management via \href{mailto:mailto:raf-dm@eol.ucar.edu}{this email address}.

In addition to the standard processing that produces the variables
in this subsection, an alternate processor is available that provides
some additional options and capabilities, including the production
of two sets of variables that include either all particles or all
particles that pass a roundness test. Additional options include different
ways of defining the particle size (including circle fitting or sizing
based on the dimension along the direction of flight. Corrections
to sizing are made to account for diffraction, and a shattering correction
can be applied based on interarrival times. Some categories of spurious
images (e.g., ``streakers'') can be recognized and rejected. This
processing is described in \href{https://drive.google.com/open?id=0B1kIUH45ca5AOFpZUGxGVUg3VWc}{this document}
and at \href{https://www.eol.ucar.edu/software/process2d}{this web page}
and is made available by special arrangement. 

Measurements based on the two 2D probes will be discussed together
in this section because the 2DC and 2DP probes function similarly,
differing primarily in the size resolution (typically 25 $\mu$m or
less for the 2DC and 100 or 200 $\mu$m for the 2DP). The following
variables have names like CONC1DC or CONC1DP to designate the two
types of hydrometeor imagers. In addition, variables normally have
location designations like '\_LWIO' as described at the beginning
of section \prettyref{subsec:1DProbes}; see page \pageref{VariableNames1DProbes}.
In the following 'y' is sometimes used to designate either 'C' or
'P'.

For the images from the 2D probes, separate data files need to be
used. RAF provides a routine ``\href{https://www.eol.ucar.edu/software/xpms2d}{XPMS2D}''
that can be used to view the images and calculate various properties
of the hydrometeor population based on these separate  files.

\label{Despite-the-'1D'}Despite the '1D' nomenclature, the following
variables are measured by 2D instruments; the '1D' designation is
used to indicate that this is the dimension that would be sized by
an equivalent 1D probe using a test that requires unshadowed end diodes
so that the full dimension of the particle can be determined. As a
consequence, the effective sample volume becomes smaller as the measured
dimension increases.
\begin{hangparagraphs}
\textbf{2D Count Rate Per Channel }(count per time interval)\textbf{:}\nop{A1DC}{}\nop{A1DP}{}\textbf{
}\textbf{\uline{A1DC}}\index{A1DC}\sindex[var]{AIDC}, \textbf{\uline{A1DP}}\index{A1DP}\sindex[var]{AIDP}\\
\index{count rate!2D}\emph{The number of particles counted by a 2D
probe in each of 62 size bins in a specified time interval, usually
1 s.} These are used to calculate the derived variables like CONC1DC,
C1DC, and others that follow, but are provided to allow re-calculation
if a user wants to use different sample volumes or sizing assumptions.\\

\textbf{2D Size Distribution }(L$^{-1}$channel$^{-1}$):\nop{C1DC}{}\nop{C1DP}{}\textbf{
}\textbf{\uline{C1DC}}\index{C1DC}\sindex[var]{C1DC}, \textbf{\uline{C1DP}}\index{C1DP}\sindex[var]{C1DP}\\
\index{size distribution!2D}\emph{The concentration of particles
measured by a 2D probe in each of 62 bins in a specified time interval,
usually 1 s.} These are calculated from A1DC by application of an
assumed size-dependent sample volume based on probe characteristics
and the flight speed. These are provided in a 64-element array for
historical convention; the first element should be ignored, and the
technique requires that the end elements be unshadowed and so precludes
any measurement with width of 63 bins, so the 64-element vector has
valid information only in bins 1--63. The cell boundaries are specified
in the netCDF header as an attribute of C1DC or C1DP, and they specify
the end points of the bin; e.g,, in the 64-element array of provided
cell boundaries, the first element is the lower size limit of the
first data cell which is the second element in C1DC. For a typical
2DC with 25-$\mu$m size resolution, the cell sizes increase by 25
$\mu$m per bin for each of the C1DC bins. Also included as attributes
with the netCDF variable C1DC or C1DP are the size-dependent depth
of field (mm) and effective sample area\footnote{commonly called ``EffectiveAreaWidth'' in the netCDF files}
(mm), the latter having values of zero for the first and last elements
in the 64-value vector. \\

\textbf{2D Concentration }(L$^{-1}$):\nop{CONC1DC}{}\nop{CONC1DC100}{}\nop{CONC1DC150}{}\nop{CONC1DP}{}\textbf{
}\textbf{\uline{CONC1DC}}\textbf{\index{CONC1DC}\sindex[var]{CONC1DC},
}\textbf{\uline{CONC1DC100}}\textbf{\sindex[var]{CONC1DC100},
}\textbf{\uline{CONC1DC150\sindex[var]{CONC1DC150},}}\textbf{
}\textbf{\uline{CONC1DP}}\textbf{\index{CONC1DP}\sindex[var]{CONC1DP}}\textbf{\uline{}}\\
\index{concentration!2D}\emph{The total concentration of all particles
detected by a 2D hydrometeor imager,} or in the case of CONC1DC100
or CONC1DC150, the concentration of all particles sized to be at least
xxx $\mu$m in the dimension perpendicular to the direction of flight,
where xxx may be 100 150. These concentrations are the sum of the
particle size distribution given below (C1DC or C1DP), with appropriate
channels excluded for CONC1DC100 and CONC1DC150.\\

\textbf{2D Dead Time }(ms):\nop{DT1DC}{}\textbf{ }\textbf{\uline{DT1DC}}\index{DT1DC}\sindex[var]{DT1DC}\\
\index{dead time}\emph{The time in the sample interval during which
the data rate exceeded the recording capability of a 2DC probe.} This
is used as a correction factor when concentrations like CONC1DC or
C1DC are calculated. The variable does not apply to measurements from
a 2DP probe.\\

\textbf{2D Mean Diameter }($\mu$m):\nop{DBAR1DC}{}\nop{DBAR1DP}{}\textbf{
}\textbf{\uline{DBAR1DC}}\index{DBAR1DC}\sindex[var]{DBAR1DC},
\textbf{\uline{DBAR1DP}}\index{DBAR1DP}\sindex[var]{DBAR1DP}\\
\index{diameter!mean, 2D probes}\emph{The mean diameter calculated
from the measured size distribution. }In this calculation, the bin
sizes are taken to be the averages of the lower and upper limits of
the size bins\emph{. }The calculation is as described by \eqref{eq:dbar1d}.\emph{
}\\

\textbf{2D Dispersion (dimensionless):}\nop{DISP1DC}{}\nop{DISP1DP}{}\textbf{
}\textbf{\uline{DISP1DC}}\index{DISP1DC}\sindex[var]{DISP1DC},
\textbf{\uline{DISP1DP}}\index{DISP1DP}\sindex[var]{DISP1DP}\\
\index{dispersion}\emph{The standard deviation in particle diameter
divided by the mean diameter.} The formula used is given by \eqref{eq:disp1d}.\\

\textbf{2D Liquid Water Content} (g\,m$^{-3}$):\nop{PLWC1DC}{}\nop{PLWC1DP}{}\textbf{
}\textbf{\uline{PLWC1DC}}\index{PLWC1DC}\sindex[var]{PLWC1DC},
\textbf{\uline{PLWC1DP}}\index{PLWC1DP}\sindex[var]{PLWC1DP}\\
\emph{\index{liquid water content}The liquid water content (mass
per volume) calculated from C1DC or C1DP.} The calculation is as described
by \eqref{eq:lwc1d}, To conform to common usage, the liquid water
content is expressed in non-MKS units of g\,m$^{-3}$. \\

\textbf{2D Radar Reflectivity Factor (dBZ):}\nop{DBZ1DC}{}\nop{DBZ1DP}{}\textbf{
}\textbf{\uline{DBZ1DC}}\index{DBZ1DC}\sindex[var]{DBZ1DC}, \textbf{\uline{DBZ1DP}}\index{DBZ1DP}\sindex[var]{DBZ1DP}\\
\index{radar reflectivity factor}\emph{The radar reflectivity factor
calculated from the measured size distribution under the assumption
that all particles are spherical water drops. }The calculation is
as described by\emph{ }\eqref{eq:dbz1d}.\\

\textbf{2D Effective Radius ($\mu$m):}\nop{REFF2DC}{}\nop{REFF2DP}{}\textbf{
}\textbf{\uline{REFF2DC}}\index{REFF2DC}\sindex[var]{REFF2DC},
\textbf{\uline{REFF2DP}}\index{REFF2DP}\sindex[var]{REFF2DP}\\
\index{radius, effective}\emph{One-half the ratio of the third moment
of the particle diameter to the second moment. }The formula used is
given by \eqref{eq:reff1d}.\label{punch:5-6}\\
 
\end{hangparagraphs}





\section{AIR CHEMISTRY MEASUREMENTS}

\subsection{Variables in Standard Data Files\label{punch:6-1}}
\begin{hangparagraphs}
\textbf{Carbon Monoxide Preliminary Mixing Ratio (ppbv):}\hypertarget{CORAW_AL}{}\textbf{
}\textbf{\uline{CORAW\_AL}}\textbf{\sindex[var]{CORAW_AL@CORAW\_AL}\index{CORAW_AL@CORAW\_AL}}\\
\emph{The preliminary measurement of CO mixing ratio from the Aero-Laser
model AL-5002 CO analyzer, before final calibrations are applied.}
This instrument measures CO\index{CO} by vacuum ultraviolet resonance
fluorescence. It is a commercial version of the instrument described
by Gerbig et al.\footnote{Journal of Geophysical Research, Vol. 104, No. D1, 1699-1704, 1999}
The instrument is described further at \href{https://www.eol.ucar.edu/instruments/aero-laser-vuv-resonance-fluorescence-carbon-monoxide-instrument}{this URL}.
The time resolution is 1 second. This variable is sometimes present
in flight and in preliminary ground processing, but normally it is
replaced by COMR\_AL in final processing.

\textbf{Carbon Monoxide Mixing Ratio (ppbv):}\hypertarget{COMR_AL}{}\textbf{
}\textbf{\uline{COMR\_AL}}\textbf{\sindex[var]{COMR_AL@COMR\_AL}\index{COMR_AL@COMR\_AL}}\\
\emph{The mixing ratio measured by the Aero-Laser model AL-5002 CO
analyzer.} See also CORAW\_AL above. The calculation of COMR\_AL is
based on in-flight calibrations conducted 1-2 times per hour, when
a gas of known concentration\index{concentration!calibration gas}
is supplied to the instrument and then a catalyst trap removes CO
to provide a zero reference. The calibration\index{calibration!gas}
results in a sensitivity and zero that are then used to convert the
measurements from the instrument (recorded as counts per second) to
a mixing ratio in units of ppbv. Time-dependent sensitivity and zero
coefficients are computed post-flight as a linear interpolation between
flight calibrations. This variable normally appears in final data
sets for a project.\footnote{In isolated cases XCOMR\sindex[var]{XCOMR} or XCOMR\_AL\sindex[var]{AL}
was used for this variable name.} The algorithm is described in the following box:\\
\\
\fbox{\begin{minipage}[t]{0.95\columnwidth}%
CPS\index{CPS=counts per second} = counts per second from the instrument\\
S(t) = sensitivity\sindex[lis]{S(t)=sensitivity function, calibration}
at time t = (CPS when exposed to cal gas) / concentration of cal gas

Z(t) = zero\sindex[lis]{Z(t)=zero function, calibration} at time
t = CPS when exposed to air passing through the catalyst trap

\rule[0.5ex]{1\columnwidth}{1pt}

\begin{equation}
\{\mathrm{COMR\_AL}\}=(\mathrm{\{\mathrm{CPS}\}-Z(t))/S(t)}\label{eq:COMR}
\end{equation}
%
\end{minipage}}\\
See also the obsolete variables in Section \ref{sec:OBSOLETE-VARIABLES},
where variables from an earlier TECO Model 48 CO analyzer, in use
before 2000, are described.

\textbf{}%
\noindent\begin{minipage}[t]{1\columnwidth}%
\textbf{Carbon Dioxide Mixing Ratio (ppmv):}\hypertarget{CO2_PIC}{}\textbf{
}\textbf{\uline{CO2\_PIC}}\uline{x}\index{CO2_PICx@CO2\_PICx}\sindex[var]{CO2_PICx@CO2\_PICx}\\
\textbf{Methane Mixing Ratio (ppmv): }\textbf{\uline{CH4\_PICx}}\index{CH4_PICx@CH4\_PICx}\sindex[var]{CH4_PICx@CH4\_PICx}%
\end{minipage}\textbf{}\\
\emph{Respectively, the carbon dioxide and methane mixing ratio measured
by a Picarro CO2/CH4 instrument.} The letter 'x' may be replaced by
the model number of the instrument (e.g., 1301) or it may be blank.
The Picarro CO2/CH4 G1301-f flight analyzer is a fast response trace
gas monitor that measures CO$_{2}$ and CH$_{4}$ using wavelength-scanned
cavity ring-down spectroscopy. The time resolution is 0.2 \textendash{}
1 seconds. Additional information characterizing the instrument can
be found at \href{https://www.eol.ucar.edu/instruments/picarro-instrument-airborne-measurement-co2-and-ch4}{this URL}.
During flight, both measurements are calibrated 1-2 times per hour
via sampling of a working standard, and linear calibration coefficients
are applied based on multi-point lab calibration data and in-flight
calibration checks. The procedure is analogous to that used for COMR\_AL,
as described immediately above. When water vapor is not removed from
the ambient sample stream (the normal case), a correction factor for
water present in the sensing cell must be applied following the approach
of Richardson et al.,\footnote{Richardson, S.~J., N.~L.~Miles, K.~J.~Davis, E.~R.~Crosson,
C.~W.~Rella, and A.~E.~Andrews, 2012\emph{: }Field testing of
cavity ring-down spectroscopy analyzers measuring carbon dioxide and
water vapor.\emph{ J.~Atmos.~Oceanic\_Technol, }\textbf{\emph{29,}}\emph{
397\textendash 406.}} as follows: \\
\\
\noindent\fbox{\begin{minipage}[t]{1\columnwidth - 2\fboxsep - 2\fboxrule}%
{[}CO$_{2}${]}$_{wet}$ = carbon dioxide mixing ratio as measured
in the sensing cell (with water)

{[}CO$_{2}${]}$_{dry}$ = carbon dioxide mixing ratio in dry air,
corrected for the effects of water vapor

{[}CH$_{4}${]}$_{wet}$ = methane mixing ratio as measured in the
sensing cell (with water)

{[}CH$_{4}${]}$_{dry}$ = methane mixing ratio in dry air, corrected
for the effects of water vapor

$W$ = water vapor mixing ratio measured in the instrument cell (percent
by volume)

\{$c_{0}$, $c_{1}$\} = \{-0.01200, -2.674$\times10^{-4}$\} (dimensionless)

\{$d_{0}$, $d_{1}$\} = \{-0.00982, -2.393$\times10^{-4}$\} (dimensionless)

\rule[0.5ex]{1\columnwidth}{1pt}

\begin{equation}
\{\mathrm{CO2\_PICX\}}=[\mathrm{CO_{2}]_{dry}=}\frac{[\mathrm{CO_{2}]_{wet}}}{1+c_{0}W+c_{1}W^{2}}\label{eq:CO2pic}
\end{equation}
\begin{equation}
\{\mathrm{CH4\_PICX\}}=[\mathrm{CH_{4}]_{dry}=}\frac{[\mathrm{CH_{4}]_{wet}}}{1+d_{0}W+d_{1}W^{2}}\label{eq:CH4pic}
\end{equation}
%
\end{minipage}}

\noindent\begin{minipage}[t]{1\columnwidth}%
\textbf{Chemiluminescent Ozone Sample Flow Rate (sccm):}\hypertarget{XFO3FS}{}\textbf{
}\textbf{\uline{XFO3FS}}\textbf{\index{XFO3FS}\sindex[var]{XFO3FS}}\\
\textbf{Chemiluminescent Ozone Nitric Oxide Flow Rate (sccm):}\hypertarget{XFO3FNO}{}\textbf{
}\textbf{\uline{XFO3FN}}\textbf{O\index{XFO3FNO}\sindex[var]{XFO3FNO}}\\
\textbf{Chemiluminescent Ozone Sample Pressure (mb):}\hypertarget{XFO3P}{}\textbf{
}\textbf{\uline{XFO3P\index{XFO3P}\sindex[var]{XFO3P}}}%
\end{minipage}\\
\emph{Flows and pressure within the chemiluminescence ozone sensor.}
The sample rate, in standard $cm^{3}/s$, is XFO3FS, while XFO3FNO
gives the NO flow rate in the same units and XFO3P is the pressure
in the ozone sample cell. These variables apply to measurements made
by an earlier version of the fast ozone instrument. They have not
been present in projects since 2006.

\textbf{Fast response NO chemiluminescence ozone mixing ratio (ppbv):}\hypertarget{FO3_ACD}{}\hypertarget{FO3_CL}{}\hypertarget{O3MR_CL}{}\hypertarget{XO3}{}\textbf{
}\textbf{\uline{FO3\_ACD}}\textbf{, }\textbf{\uline{FO3\_CL}},
\textbf{XO3, O3MR\_CL\sindex[var]{O3MR_CL@O3MR\_CL}\index{O3MR_CL@O3MR\_CL}\sindex[var]{XO3}}\index{XO3}\index{FO3_x@FO3\_x}\sindex[var]{FO3_x@FO3\_x}\index{FO3_ACD@FO3\_ACD}\index{FO3_CL@FO3\_CL}\sindex[var]{FO3_ACD@FO3\_ACD}\sindex[var]{FO3_CL@FO3\_CL}\\
\emph{The ozone mixing ratio (by volume) measured by an NO chemiluminescence
instrument. } The instrument detects chemiluminescence from the reaction
of nitric oxide (NO) with ambient ozone, using a dry-ice cooled, red-sensitive
photomultiplier employing photon-counting electronics. The measurement
principle is described by Ridley et al.~(1992)\footnote{Ridley, B.~A., F.~E.~Grahek, and J.~G.~Walega, 1992: A small,
high-sensitivity, medium-response ozone detector suitable for measurements
from light aircraft. \emph{J.~Atmos.~Oceanic Technol., }\textbf{9,
}142\textendash 148. }, and there is additional information describing the instrument at
\href{https://www.eol.ucar.edu/instruments/nitric-oxide-chemiluminescence-ozone-instrument}{this URL}.
The time resolution is 0.2 seconds, and typical uncertainty is 5\%.
The background signal is measured 1-2 times hourly during flights.
Linear calibration coefficients are applied to the photon count rate
to produce mixing ratios, and a correction is applied for water vapor
during final processing, as follows: \\
\noindent\fbox{\begin{minipage}[t]{1\columnwidth - 2\fboxsep - 2\fboxrule}%
CPS\index{CPS=counts per second} = counts per second from the instrument\\
{[}O$_{3}${]}$_{wet}$ = ozone mixing ratio as measured in the sensing
cell (with water)

{[}O$_{3}${]}$_{dry}$ = ozone mixing ratio in dry air, corrected
for the effects of water vapor\\
$S(t)$ = sensitivity at time t = (\{CPS\} when exposed to cal gas)
/ concentration of cal gas

$Z(t)$ = background at time t = \{CPS\} when exposed to zero-ozone
air

$W^{\prime}$ = water vapor mixing ratio by volume\sindex[lis]{rv@$r_{v}$=water vapor mixing ratio by volume}
(expressed as a fraction; dimensionless)

$\kappa$ = correction factor for water vapor = 4.3 (dimensionless)

\rule[0.5ex]{1\columnwidth}{1pt}

\begin{equation}
[\mathrm{O}_{3}]_{wet}=\frac{\{\mathrm{CPS}\}-Z(t)}{S(t)}\label{eq:FO3}
\end{equation}
\begin{equation}
\{\mathrm{F03\_ACD}\}=[\mathrm{O_{3}]}_{dry}=[\mathrm{O_{3}]}_{wet}\times(1+\kappa r_{v})\label{eq:FO3WaterCorr}
\end{equation}
%
\end{minipage}}

\textbf{Uncorrected TECO Ozone Mixing Ratio (ppb):}\hypertarget{TEO3}{}\textbf{
}\textbf{\uline{TEO3}}\index{TEO3}\sindex[var]{TEO3}\\
\emph{The uncorrected ozone mixing ratio output from the TECO model
49c UV ozone analyzer.} See TEO3C. 

\noindent\begin{minipage}[t]{1\columnwidth}%
\textbf{Internal TECO Ozone Sampling Pressure (hPa):}\hypertarget{TEP}{}\hypertarget{TEO3P}{}\textbf{
}\textbf{\uline{TEP}}\textbf{\index{TEP}\sindex[var]{TEP}, }\textbf{\uline{TEO3P}}\index{TEO3P}\sindex[var]{TEO3P}\\
\textbf{Internal TECO Ozone Sampling Temperature ($^{\circ}C$):}\hypertarget{TET}{}\textbf{
}\textbf{\uline{TET\index{TET}}}\sindex[var]{TET}%
\end{minipage}\\
\emph{The pressure (TEP) or temperature (TET) inside the detection
cell of the TECO 49 UV ozone analyzer.} These are used to convert
the measurements from the instrument to units of ppbv. In many projects,
the cell temperature was not recorded so an expected cell temperature
in the aircraft cabin must be used in processing.

\textbf{Corrected TECO Ozone Mixing Ratio (ppbv):}\hypertarget{TEO3C}{}\textbf{
}\textbf{\uline{TEO3C}}\index{TEO3C}\sindex[var]{TEO3C}\\
\emph{The ozone mixing ratio (by volume) determined by the TECO model
49c UV ozone analyzer (cf.~\href{https://www.eol.ucar.edu/instruments/thermo-environmental-instruments-model-49-ozone-analyzer}{this description})
after correction for the pressure and temperature in the cell by application
of the ideal gas law.} Because the basic measurement is ozone density
in the chamber, this measurement must be converted to a mixing ratio
by dividing by the air density, calculated from the pressure and temperature
measured in the chamber (TEP and TET respectively).\label{punch:6-3}
The instrument provides output only each ten seconds, and measurements
are collected in the 3 s preceding the update. The measurements may
be artificially high or low when rapid changes in humidity are present,
as may occur when crossing the top of the boundary layer or when going
through clouds. In operation on the ground prior to takeoff or immediately
after landing, a high concentration of hydrocarbons can cause spuriously
high measurements. The detection limit is 1 ppbv with an uncertainty
of $\pm$5\%. This instrument is seldom used as of 2014 and may soon
be classified as obsolete.

\textbf{}%
\noindent\begin{minipage}[t]{1\columnwidth}%
\textbf{NO Raw Counts (counts per sample interval):}\hypertarget{XNO}{}\textbf{
}\textbf{\uline{XNO}}\textbf{\index{XNO}}\sindex[var]{XNO}\textbf{}\\
\textbf{NOy Raw Counts (counts per sample interval):}\hypertarget{XNOY}{}\textbf{
}\textbf{\uline{XNOY\index{XNOY}}}\sindex[var]{XNOY}\textbf{\uline{}}\\
\textbf{NO Calibration Flow (SLPM):}\hypertarget{XNOCF}{}\textbf{
}\textbf{\uline{XNOCF}}\textbf{\index{XNOCF}}\sindex[var]{XNOCF}\textbf{}\\
\textbf{NOy Calibration Flow (SLPM):}\hypertarget{XNCLF}{}\textbf{
}\textbf{\uline{XNCLF}}\textbf{\index{XNCLF}}\sindex[var]{XNCLF}\textbf{}\\
\textbf{NO, NOy Measurement Status (dimensionless):}\hypertarget{XNST}{}\textbf{
}\textbf{\uline{XNST}}\textbf{\index{XNST}}\sindex[var]{XNST}\textbf{}\\
\textbf{NO Zero Air Flow (SLPM):}\hypertarget{XNOZA}{}\textbf{ }\textbf{\uline{XNOZA}}\textbf{\index{XNOZA}}\sindex[var]{XNOZA}\textbf{}\\
\textbf{NOy Zero Air Flow (SLPM):}\hypertarget{XNZAF}{}\textbf{ }\textbf{\uline{XNZAF}}\textbf{\index{XNZAF}}\sindex[var]{XNZAF}\textbf{}\\
\textbf{NO Sample Flow (SLPM):}\hypertarget{XNOSF}{}\textbf{ }\textbf{\uline{XNOSF}}\textbf{\index{XNOSF}}\sindex[var]{XNOSF}\textbf{}\\
\textbf{NOy Sample Flow (SLPM):}\hypertarget{XNSAF}{}\textbf{ }\textbf{\uline{XNSAF}}\textbf{\index{XNSAF}}\sindex[var]{XNSAF}\textbf{}\\
\textbf{NOy Reaction Chamber Pressure (mb):}\hypertarget{XNOYP}{}\textbf{
}\textbf{\uline{XNOYP}}\textbf{\index{XNOYP}}\sindex[var]{XNOYP}\textbf{}\\
\textbf{Gold NOy Converter Temperature ($\text{�}$C):}\hypertarget{XNMBT}{}\textbf{
}\textbf{\uline{XNMBT\index{XNMBT}}}\sindex[var]{XNMBT}%
\end{minipage}\\
\emph{The measurements provided by the NO+NO$_{2}$ instrument, }which
is described at \href{https://www.eol.ucar.edu/instruments/no-no2-instrument}{this link}\emph{.}
XNO and XNOY are the raw data counts from the NO and NO$_{2}$ instruments,
respectively, and XNCLF and XNOCF are the respective calibration flows
for these instruments. \label{punch:6-2} XNST records the status
for both instruments: In measurement mode, XNST is 0, while XNST is
5 when the instruments are in zero mode and 10 when the instruments
are in calibration mode. the NOy and NO instruments. The instrument
is in the measure mode for XNST of 0. For a XNST reading of 5 the
instruments are in the zero mode. XNST value of 10 is the calibration
mode. XNOZA and XNZAF are flow rates for zero air used to back flush
inlets, typically at takeoff and landing, and for calibration using
``zero'' air. Even if the status, XNST, is 0, indicating the instrument
is in the measurement mode, when XNOZA and XNZAF are approximately
1 SLPM the instrument is measuring zero air and not ambient air. \label{punch:6-4}
XNOSF and XNSAF are the sample flow rates through the NO and NO$_{2}$
instruments respectively. These values are typically about 1 SLPM.
XNMBT is the temperature of the gold NO$_{2}$ converter.

\noindent\begin{minipage}[t]{1\columnwidth}%
C\textbf{orrected NO Mixing Ratio (ppbv):}\hypertarget{XNOCAL}{}\textbf{
}\textbf{\uline{XNOCAL}}\textbf{\index{XNOCAL}}\\
\textbf{Corrected NO$_{2}$ Mixing Ratio (ppbv):}\hypertarget{XNYCAL}{}\textbf{
}\textbf{\uline{XNYCAL\index{XNYCAL}}}%
\end{minipage}\\
\emph{The calibrated NO and NO$_{2}$ volumetric mixing ratio, respectively,
measured by the NO-NO$_{2}$ instrument.} See \href{https://www.eol.ucar.edu/instruments/nitric-oxide-chemiluminescence-ozone-instrument}{this link}
for a description of the instrument.\label{punch:6-5} The NO and
NO$_{2}$ data are represented by a cubic spline for baseline subtraction,
and then the calibration coefficients are applied and the measurements
are converted to units of ppbv. The quality of the data can be assessed
by examining the accuracy of the zero correction. This instrument
\label{punch:6-6} adds water vapor to the sample stream to reduce
the effect of ambient water on the final signal. The water vapor addition
is not sufficient to saturate the sample stream, but enough to remove
much of the interference. The detection limits of the NO,NO$_{2}$
instruments are 50 ppbv for a one-second averaging time. The uncertainty
is $\pm$ 5\%.
\end{hangparagraphs}


\subsection{Variables in Special Data Sets}

Research projects often incorporate user-supplied instruments into
payloads, and those instruments produce data files that are either
recorded independently or merged into the standard netCDF data files
for the projects. In addition, NCAR offers a set of instruments that
require additional data processing and analysis, often because the
measurements require special interpretation to obtain the desired
measurements. The following instruments can provide such air-chemistry
measurements: 

\fbox{\begin{minipage}[t]{0.98\columnwidth}%
\begin{itemize}
\item \setlength{\itemsep}{-1\parsep}Advanced Whole Air Sampler (\href{http://www.eol.ucar.edu/instruments/advanced-whole-air-sampler}{AWAS})\index{AWAS}
\item Chemical Ionization Mass Spectrometer (\href{http://www.eol.ucar.edu/instruments/georgia-tech-chemical-ionization-mass-spectrometer}{CIMS})\index{CIMS}
\item Quantum Cascade Laser Spectrometer (\href{http://www.eol.ucar.edu/instruments/quantum-cascade-laser-spectrometer}{QCLS})\index{QCLS}
\item Trace Organic Gas Analyzer (\href{http://www.eol.ucar.edu/instruments/trace-organic-gas-analyzer}{TOGA})\index{TOGA}
\end{itemize}
%
\end{minipage}}\\
Follow the links in the box to descriptions of these instruments on
the EOL web site. Those descriptions include brief explanations of
how data are acquired and handled. The process varies with instrument;
The CIMS and QCLS instruments produce variables that are often merged
into the standard netCDF archived data files for projects, the AWAS
collects samples that are later analyzed using ground-based instruments
but result in a special dataset dependent on analysis technique and
sample location and duration, while the TOGA is usually analyzed to
produce dozens of trace-gas measurements, some of which can be merged
into standard netCDF files. 

Users interested in using these measurements should contact \href{mailto:mailto:raf-dm@eol.ucar.edu}{EOL/RAF data management}
for data access and assistance.



\section{AEROSOL PARTICLE MEASUREMENTS\label{sec:AEROSOL-PARTICLE-MEASUREMENTS}}

\subsection{Condensation Nucleus Counters}

RAF uses two modified TSI, Inc. \href{http://www.eol.ucar.edu/instruments/condensation-nucleus-counter-water-or-butanol}{condensation nucleus counters}\index{condensation nucleus counter|see {CN counter}}\index{CN counter}\index{concentration!aerosol}\index{concentration!ultrafine particles}
to measure the total concentration of ultrafine particles\index{particles!ultrafine}
in the atmosphere, a 3760A\index{CN counter!3760A} using n-butyl
alcohol and a water-based 3786\index{CN counter!3786} WCN (water
condensation nucleus) counter. Both are sensitive to particles in
the approximate diameter range from 0.010\textendash 3~\textgreek{m}m.
\\

\begin{hangparagraphs}
\textbf{CN Counter Inlet Pressure (hPa): }\textbf{\uline{PCN}}\sindex[var]{PCN}\index{PCN}\\
\emph{The absolute pressure inside the inlet tube of the instrument.}
It\index{pressure!inlet} as measured by a Heise Model 623 pressure
sensor for the 3760A, and internally by the 3786 WCN.. The measurement
is used to convert the measured mass flow (FCN or XICN) to volumetric
flow and to convert measured particle concentration to equivalent
ambient concentration.

\textbf{CN Counter Inlet Temperature ($^{\circ}C$): }\textbf{\uline{CNTEMP}}\sindex[var]{CNTEMP}\index{CNTEMP}\uline{,}
\textbf{\uline{TEMP1}}\sindex[var]{TEMP1}\textbf{\index{TEMP1},
}\textbf{\uline{TEMP2}}\sindex[var]{TEMP2}\textbf{\index{TEMP2}}\\
\emph{The sample air tempeerature measured at the intake of the 3760A
or within the 3786.} The value\index{temperature!inlet} is used to
convert the measured mass flow (FCN or XICN) to true volumetric flow
and to convert measured particle concentration to equivalent ambient
concentration.\index{concentration!ambient}

\textbf{}%
\noindent\begin{minipage}[t]{1\columnwidth}%
\textbf{Raw CN Counter Sample Flow Rate (SLPM): }\textbf{\uline{FCN}}\sindex[var]{FCN}\index{FCN}\\
\textbf{Corrected CN Counter Sample Flow Rate (VLPM): }\textbf{\uline{FCNC}}\sindex[var]{FCNC}\textbf{\uline{\index{FCNC}}}%
\end{minipage}\\
\emph{The raw and corrected sample flows in the CN counters} are treated
differently for the two models of CN counter. In the 3760A, FCN is
measured in standard liters per minute (SLPM) with a mass flow meter.
The flow meter gives the volumetric flow rate that would apply under
standard conditions of 1013.25~hPa and 21$^{\circ}$C.\index{CN counter!flow rate}
FCNC is the corrected sample flow rate in volumetric liters per minute
(VLPM) \emph{at instrument pressure and temperature}. For the 3760A:\\
\fbox{\begin{minipage}[t]{0.9\textwidth}%
PCN = pressure at the inlet to the CN counter (hPa)\\
CNTEMP = temperature at the inlet of the sample tube ($^{\circ}$C)\\
$p_{std}$ = \sindex[lis]{Pstd@$p_{std}$=standard pressure}standard
reference pressure, 1013.25~hPa\\
$T_{std}$ = \sindex[lis]{Tstd@$T_{std}$=standard temperature}standard
reference temperature, 294.15 K \\
$T_{0}$ = 273.15~K\\
\\
\rule[0.5ex]{1\linewidth}{1pt}
\[
\mathrm{FCNC=\{FCN\}}\frac{p_{std}}{\mathrm{\{PCN\}}}\frac{(\{\mathrm{CNTEMP\}}+T_{0})}{T_{std}}
\]
%
\end{minipage}}\\
In the 3786, flows are determined in volumetric cm$^{3}\thinspace\mathrm{min}^{-1}$
from the pressure drop across an orifice. The 3786 firmware makes
density corrections internally, so its reported sample flow is brought
directly into the variable FCNC in units of VLPM.\\
\vfill\eject

\noindent\begin{minipage}[t]{1\columnwidth}%
\textbf{Raw CN Isokinetic Side Flow Rate (SLPM): }\textbf{\uline{XICN}}\sindex[var]{XICN}\index{XICN}\\
\textbf{Corrected CN Isokinetic Side Flow Rate (VLPM): }\textbf{\uline{XICNC}}\sindex[var]{XICNC}\textbf{\uline{\index{XICNC}}}%
\end{minipage}\\
\emph{XICN is the raw isokinetic side flow rate in standard liters
per minute (SLPM) measured with a mass flow meter, and XICNC is that
flow corrected for pressure and temperature to be the true volumetric
flow.} The side flow\index{CN counter!side flow} is adjusted for
isokinetic sampling at the inlet, but it is not used further in processing.
\\
\fbox{\begin{minipage}[t]{0.9\textwidth}%
XICN = side-flow rate (SLPM)

PCN = pressure at the inlet to the CN counter (hPa)\\
CNTEMP = temperature at the inlet of the sample tube ($^{\circ}$C)\\
$p_{std}$ = standard reference pressure, 1013.25 mb\\
$T_{std}$\sindex[lis]{Tr@$T_{std}$= absolute reference temperature, STP}
= 294.15 K \\
$T_{0}$ = 273.15~K\\
\\
\rule[0.5ex]{1\linewidth}{1pt}
\[
\mathrm{XICNC=\{XICN\}}\frac{pP_{std}}{\mathrm{\{PCN\}}}\frac{(\{\mathrm{CNTEMP\}}+T_{0})}{T_{std}}
\]
%
\end{minipage}}\\

\textbf{CN Counter Output (counts per sample interval); }\textbf{\uline{CNTS}}\sindex[var]{CNTS}\index{CNTS}\\
\emph{The raw output count from the condensation nucleus counter.}
For the 3760A condensation nucleus counter, the project-dependent
sample rate may be chosen in the range from 1\textendash 50~Hz but
it is typically 10~Hz.. In some unusual cases the counts are divided
by a selected power of two to keep the counter from overflowing; see
project documentation. The 3786 WCN may be programmed to report data
at intervals from 0.1\textendash 3600~s.\label{punch:7-1}

\textbf{Condensation Nucleus (CN) Concentration (cm$^{-3}$): }\textbf{\uline{CONCN}}\sindex[var]{CONCN}\index{CONCN}\\
\emph{The number concentration of condensation nuclei} in units of
particles per cm$^{3}$ \emph{in the ambient air} at flight level.\emph{
}The calculation leading to CONCN\index{concentration!CN} includes
two corrections. The first\index{CN counter!coincidence in} accounts
for coincidence of particles in the viewing volume at high concentrations
and is handled differently in the two types of CN counter. For the
3760A, a statistical adjustment is made based on the average time
of a particle in the viewing volume. This correction increases from
about 1\% at a total concentration of 10$^{3}$~cm$^{-3}$ to nearly
11\% at 10$^{4}$~cm$^{-3}$, but for concentrations above about
2\texttimes 10$^{4}$~cm$^{-3}$ significant uncertainty remains.
The 3786 instead measures the time each detected particle occupies
the viewing volume, and this\emph{ }accumulated \textquotedblleft dead
time\textquotedblright \index{CN counter!dead time} in each sampling
interval is subtracted from the elapsed time yielding a \textquotedblleft live
time\textquotedblright{} for the determination of sample volume. With
this correction an accuracy of 12\%, not including statistical counting
error, is specified by the manufacturer at concentrations up to 10$^{5}$~cm$^{-3}$.
The second correction, applied to all CN counters, is a conversion
from instrument to ambient conditions.\footnote{Prior to Dec.~2007 the conversion to ambient concentration was not
made and concentration was reported for instrument conditions.} In the following formulae, the corrected flow FCNC in VLPM is explicitly
converted to cm$^{3}$s$^{-1}$ by the factor (1000/60).\emph{}\\
\emph{}\\
\noindent\begin{minipage}[t]{1\columnwidth}%
\textbf{For the 3760A:}\\
\fbox{\begin{minipage}[t]{0.9\textwidth}%
CNTS = particle counts per sample interval from the CN counter\index{CNTS}\\
$\Delta t$ = \sindex[lis]{Deltat@$\Delta t$=time interval}interval
between recorded samples (s)\\
$D$ = scale factor (legacy; normally 1)\\
$C_{flow}$ = \sindex[lis]{Cflow@$C_{flow}$ = flow conversion factor}conversion
factor, (1000/60) cm$^{3}$L$^{-1}$min s$^{-1}$\\
FCNC = corrected sample flow rate (VLPM) for instrument conditions\index{FCNC}\\
$t_{vv}$ = average time a particle is in the view volume = 0.4$\times10^{-6}$~s\\
PCN = pressure at the inlet to the CN counter (hPa)\index{PCN}\\
CNTEMP = temperature at the inlet of the sample tube ($^{\circ}$C)\index{CNTEMP}\\
PSXC = corrected ambient pressure (hPa)\index{PSXC}\\
ATX = ambient temperature ($^{\circ}$C)\index{ATX}\\
$T_{0}$ = 273.15~K\\
\\
\rule[0.5ex]{1\linewidth}{1pt}
\[
\mathrm{A=\frac{\{CNTS\}}{\mathrm{(\{FCNC\}\times C_{flow})}\Delta t}\,D}
\]
The flow under instrument conditions, corrected for coincidence, is
then\\
\[
B\mathrm{=A}\,e^{At_{vv}(\mathrm{\{FCNC\}\times C_{flow})}}
\]
and the concentration under ambient conditions is\index{CONCN}\\

\begin{equation}
\mathrm{\{CONCN\}}=B\frac{\mathrm{\{PSXC\}}}{\mathrm{\{PCN\}}}\frac{\mathrm{(\{CNTEMP\}}+T_{0})}{(\mathrm{\{ATX\}}+T_{0})}\label{eq:12.1}
\end{equation}
%
\end{minipage}} %
\end{minipage}\\
\\
\noindent\begin{minipage}[t]{1\columnwidth}%
\textbf{For the 3786 WCN:}\\
\fbox{\begin{minipage}[t]{0.9\textwidth}%
CNTS = particle counts per sample interval from the CN counter\\
$\Delta t$ = \sindex[lis]{Deltat@$\Delta t$=time interval}interval
between recorded samples (s)\\
$t_{d}$ = cumulative dead time during the sampling interval (s)\\
$C_{flow}$ {[}see preceding box{]}\\
FCNC = corrected sample flow rate (VLPM) for instrument conditions\\
PCN = internal pressure of the CN counter (hPa)\\
CNTEMP = temperature of the optics block ($^{\circ}$C)\\
PSXC = corrected ambient pressure (hPa)\\
ATX = ambient temperature ($^{\circ}$C)\\
$T_{0}$ = 273.15~K\\
\index{CONCN}\\
\rule[0.5ex]{1\linewidth}{1pt}
\[
\mathrm{A=\frac{\{CNTS\}}{\mathrm{(\{FCNC\}\times C_{flow})}(\Delta t-t_{d})}}
\]

\begin{equation}
\mathrm{\{CONCN\}}=A\frac{\mathrm{\{PSXC\}}}{\mathrm{\{PCN\}}}\frac{\mathrm{(\{CNTEMP\}}+T_{0})}{(\mathrm{\{ATX\}}+T_{0})}\label{eq:12.2}
\end{equation}
%
\end{minipage}} %
\end{minipage}

\vfill\eject
\end{hangparagraphs}


\subsection{Aerosol Spectrometers}

\index{aerosol!spectrometer}For size-resolved measurements of the
concentration of aerosol particles, RAF deploys two instruments. The
\href{https://www.eol.ucar.edu/instruments/ultra-high-sensitivity-aerosol-spectrometer}{Ultra High Sensitivity Aerosol Spectrometer}
(UHSAS)\index{UHSAS} sizes particles in 99 bins from 0.06 to 1.0~\textgreek{m}m
diameter, and the \href{http://www.eol.ucar.edu/instruments/passive-cavity-aerosol-spectrometer-probe}{Passive Cavity Aerosol Spectrometer Probe}
(PCASP)\index{PCASP} has 31 channels covering the diameter range
0.1 to 3~\textgreek{m}m. Flow and total concentration variables for
these instruments are described in this section, while additional
variables are covered along with other 1-D probes in Sect.~\ref{subsec:1DProbes},
``Sensors of individual Particles (1-D Probes).''
\begin{hangparagraphs}
\textbf{UHSAS Absolute Pressure in Canister (hPa): }\textbf{\uline{UPRESS}}\index{UPRESS}\sindex[var]{UPRESS}\\
The pressure internal to the UHSAS instrument. This is an analog measurement
with calibration coefficients as recorded in the attributes for the
variable.\label{punch:7-2}

\noindent\begin{minipage}[t]{1\columnwidth}%
\textbf{Raw Sample Flow Rate (cm$^{3}$s$^{-1}$): }\textbf{\uline{USMPFLW}}\textbf{,
}\textbf{\uline{PFLW}}\index{PFLW}\index{USMPFLW}\sindex[var]{PFLW}\sindex[var]{USMPFLW}\\
\textbf{Corrected Sample Flow Rate (cm$^{3}$s$^{-1}$): }\textbf{\uline{USFLWC}}\textbf{,
}\textbf{\uline{PFLWC}}\index{PFLWC}\index{USFLWC}\sindex[var]{PFLWC}\sindex[var]{USFLWC}%
\end{minipage}\\
Unlike the other 1-d probes, both UHSAS and PCASP have internal pumps
so their sample volumes are determined from the measured flows and
do not depend on true air speed. The UHSAS measures volumetric flow\index{UHSAS!flow}
directly, and it is adjusted to ambient conditions for the calculation
of ambient concentration. The PCASP\index{flow!PCASP} returns a mass
flow referenced to standard conditions, and this also is converted
to equivalent ambient volumetric flow.\\
\\
\fbox{\begin{minipage}[t]{0.9\textwidth}%
UPRESS = internal UHSAS pressure (hPa)\index{UPRESS}\\
USMPFLW = measured volumetric sample flow (cm$^{3}$s$^{-1}$)\\
PFLW = sample mass flow referenced to standard conditions (cm$^{3}$s$^{-1}$)\\
$T_{blk}$ = UHSAS optical block temperature, 305 K\\
$p_{std}$ = standard pressure, 1013.25 hPa\\
$T_{std}$ = standard temperature, 298.15 K\index{UHSAS!STP}\\
PSXC = corrected ambient pressure (hPa)\\
ATX = ambient temperature ($^{\circ}$C)\\
$T_{0}$ = 273.15~K\\
\\
\rule[0.5ex]{1\linewidth}{1pt}

\begin{equation}
\mathrm{\{PFLWC\}}=\mathrm{\{PFLW\}}\frac{p_{std}}{\mathrm{\{PSXC\}}}\frac{(\mathrm{\{ATX\}}+T_{0})}{T_{std}}\label{eq:12.3}
\end{equation}
\begin{equation}
\mathrm{\{USFLWC\}}=\mathrm{\{USMPFLW\}}\frac{\mathrm{\{UPRESS\}}}{\mathrm{\{PSXC\}}}\frac{\mathrm{(\{ATX\}}+T_{0})}{T_{blk}}\label{eq:12.4}
\end{equation}
%
\end{minipage}} \\
\\

\textbf{Total particle counts per sample interval, UHSAS or PCASP:
}\textbf{\uline{TCNTU}}\textbf{, }\textbf{\uline{TCNTP}}\index{TCNTP}\index{TCNTU}\sindex[var]{TCNTP}\sindex[var]{TCNTU}\\
The total particle counts in each sample interval for, respectively,
the UHSAS and PCASP instruments. These values are the sum of counts
in all cells of the spectrometers, as represented in the vector variables
CUHSAS or CS200. See the discussion of these variables in Sect.~\ref{subsec:1DProbes},
\vpageref{CUHSAS}.

\textbf{Concentration, sum over all channels (cm$^{3}$s$^{-1}$):
}\textbf{\uline{}}\\
\textbf{\uline{CONCU}}\textbf{, }\textbf{\uline{CONCP}}\textbf{,
}\textbf{\uline{CONCU100}}\textbf{, }\textbf{\uline{CONCU500}}\index{CONCU}\index{CONCP}\index{CONCU100}\index{CONCU500}\sindex[var]{CONCU}\sindex[var]{CONCP}\sindex[var]{CONCU100}\sindex[var]{CONCU500}\\
\emph{The particle concentrations summed over all or a subset of channels.}
\index{concentration!particle}CONCU and CONCP are summed over all
channels in the UHSAS\index{UHSAS} and PCASP\index{PCASP}, respectively,
and are calculated as in the following boxed equations. CONCU100 and
CONCU500 are concentrations summed over channels in the UHSAS giving
particle concentrations for diameters greater than or equal to 100~nm
and 500~nm, respectively, and are calculated as for CONCU except
with TCNTU replaced by the sum over the appropriate channels.\\
\\
\fbox{\begin{minipage}[t]{0.9\textwidth}%
TCNTU = total particle counts per sample interval, UHSAS

TCNTP = total particle counts per sample interval, PCASP

$\Delta t$ = \sindex[lis]{Deltat@$\Delta t$=time interval}sample
interval (s)

USFLWC = corrected sample flow rate, UHSAS (cm$^{3}$s$^{-1}$)

PFLWC = corrected sample flow rate, PCASP (cm$^{3}$s$^{-1}$)\\
\\
\rule[0.5ex]{1\linewidth}{1pt}

\begin{equation}
\mathrm{\{CONCU\}}=\frac{\mathrm{\{TCNTU\}}}{\mathrm{\{USFLWC\}}\Delta t}\label{eq:12.5}
\end{equation}
\begin{equation}
\mathrm{\{CONCP\}}=\frac{\mathrm{\{TCNTP\}}}{\mathrm{\{PFLWC\}}\Delta t}\label{eq:12.6}
\end{equation}
%
\end{minipage}} \\
\\
\end{hangparagraphs}


\subsection{Other Specialized Aerosol Measurements}
\begin{hangparagraphs}
Data from an aerosol mass spectrometer\index{spectrometer!aerosol mass},
a scanning mobility particle spectrometer\index{SMPS}, and a giant
nucleus impactor\index{impactor!giant nucleus} are recorded by these
instruments in separate data files and are not recorded by the aircraft
data system. The ancillary\index{aerosol!ancillary datasets} data
sets are not merged into the netCDF archives produced by EOL, so the
special data files must be used for these measurements, The data formats
are described with the instruments at the references given below:\\
\\
\textbf{Aerosol Mass Spectrometer (AMS) data files} contain size-segregated
chemical composition of non-refractory, submicron aerosol particles.
The instrument is described here: \href{https://www.eol.ucar.edu/instruments/time-flight-aerosol-mass-spectrometer}{https://www.eol.ucar.edu/instruments/time-flight-aerosol-mass-spectrometer}.\\
\textbf{Scanning Mobility Particle Spectrometer (SMPS) files} contain
fine particle differential size distributions. The number of channels
and covered size range are variable. Diameter ranges from about 7.5~nm
up to about 500~nm (pressure-dependent), and 15 size bins are typical.
The instrument is described here: \href{https://www.eol.ucar.edu/instruments/scanning-mobility-particle-spectrometer}{https://www.eol.ucar.edu/instruments/scanning-mobility-particle-spectrometer}.\\
\textbf{Auto-GNI, GNI Giant Nuclei Impactor (GNI) files} contain dry
differential particle size distributions. The instrument is described
here: \href{https://www.eol.ucar.edu/instruments/giant-nuclei-impactor }{https://www.eol.ucar.edu/instruments/giant-nuclei-impactor }.
\end{hangparagraphs}





\section{RADIATION VARIABLES}

\subsection{Measurements of Irradiance\index{irradiance} and Radiometric Temperature\index{temperature!radiometric}}

The following references, although in part obsolete now, have additional
information on radiation\index{radiation} measurements from NCAR
aircraft: \href{http://www.eol.ucar.edu/raf/Bulletins/bulletin25.html}{RAF Bulletin 25},\index{Bulletin 25}
\href{http://nldr.library.ucar.edu/repository/assets/technotes/TECH-NOTE-000-000-000-175.pdf}{Bannehr and Glover, 1991, NCAR Technical Note NCAR/TN-364+STR},
and \href{http://journals.ametsoc.org/doi/pdf/10.1175/1520-0450\%281977\%29016\%3C0190\%3APFIPP\%3E2.0.CO\%3B2}{Albrecht and Cox, 1977}.\footnote{Albrecht, B. and Cox, S.K.: 1977, Procedure for Improving Pyrgeometer
Performance, J\emph{. Appl. Meteorol.}, \textbf{16, }188--197.} The instruments\index{instruments!radiation} are described in the
``Radiation'' section on \href{https://www.eol.ucar.edu/aircraft-instrumentation}{the EOL web site}.
Some other radiometric measurements appear in Section \ref{sec:State Variables}
because the measurements fit better there with measurements of state
variables for the atmosphere; these include two measurements of air
temperature by radiometric thermometers, AT\_ITR (p.~\pageref{AT_ITR}),
OAT (p.~\pageref{OAT}), and the Microwave Temperature Profiler (MTP,
p.~\pageref{subsec:MTP}) that measures temperature profiles above
and below the aircraft by radiometric measurements.\\

\begin{hangparagraphs}
\textbf{Radiometric (Surface or Sky/Cloud-Base) Temperature ($^{\circ}C$):}\nop{RSTx}{}\textbf{
}\textbf{\uline{RSTx}}\sindex[var]{RSTx}\index{RSTx}\\
\emph{The equivalent black body temperature measured by an infrared
radiometer.} The radiometers\index{radiometers} used on the GV and
C-130 are Heimann\index{radiometer!Heimann} Model KT-19.85 precision
radiation thermometers. The KT19.85 spectral band extends from 9.6
to 11.5 \textmu m, and it has a 2\r{ } field of view. The x in the
variable name denotes the instrument location on either the bottom
(B) or top (T) of the aircraft. The KT-19.85 instruments are calibrated
using a black-body source manufactured by Eppley.\footnote{Some archived projects used this variable name for measurements from
a narrow bandwidth, narrow field-of-view (2$\text{�}$) Barnes Engineering
Model PRT-5 precision radiation thermometer. This instrument is now
retired. The spectral bandwidth available was either 8 to 14 $\mu m$
or 9.5 to 11.5 $\mu m$. Its cavity temperature was monitored and
recorded as either TCAVB\index{TCAVB}\sindex[var]{TCAVB} or TCAVT.\index{TCAVT}\sindex[var]{TCAVT}}

\textbf{Radiometer Sensor Head Temperature ($^{\circ}C$):}\nop{TRSTx}{}\textbf{
}\textbf{\uline{TRSTx}}\sindex[var]{TRSTB, TRSTT}\index{TRSTB, TRSTT}\\
\emph{The temperature of the sensing head of the KT19.85 radiometer
sensing head},\index{Heimann radiometer} usually applying to RSTB,
the primary down-looking instrument. The down-looking instrument is
normally heated to maintain a sensor-head temperature near the scene
temperature. Consult the archived netCDF files or project reports
for the calibration coefficients used, which often varied among projects.

\textbf{Pyrgeometer Output (V):}\nop{IRxV}{}\textbf{ }\textbf{\uline{IRxV}}\index{IRxV}\sindex[var]{IRXV}
\\
\emph{The voltage representing long-wave irradiance,} from a pyrgeometer\index{pyrgeometer}
manufactured by Kipp \& Zonen. The CGR4 model used on the GV and C-130
includes a meniscus dome that provides a 180� field of view with negligible
directional response error over the spectral range of 4.2 to 45 \textmu m.
The thermal stability of the dome construction and coupling to the
instrument body eliminates the need for dome temperature measurements
or dome shading. It is calibrated at the Naval Research Lab over a
range of temperatures encountered during flight according to procedures
specified by Bucholtz et al. (2008).\index{pyrgeometer!calibration}\footnote{\label{fn:Bucholtz-2008}Bucholtz , Anthony, Robert T. Bluth , Ben
Kelly, Scott Taylor, Keir Batson, Anthony W. Sarto , Tim P. Tooman
, Robert F. McCoy, 2008: The Stabilized Radiometer Platform (STRAP)
--- An Actively Stabilized Horizontally Level Platform for Improved
Aircraft Irradiance Measurements. \emph{J. Atmos. Oceanic Technol.
}, \textbf{25,} 2161 -- 2175.} The pyrgeometers are usually flown in pairs, one looking upward and
one looking downward. The letter \textquoteright x\textquoteright{}
denotes location on either bottom (B) or top (T) of the aircraft.
The primary derived variable from this instrument is IRxC, below.

\textbf{Pyrgeometer Housing Temperature ($^{\circ}$C):}\nop{IRxHT}{}\textbf{
}\textbf{\uline{IRxHT}}\index{IRxHT}\sindex[var]{IRxHT}\\
\emph{The temperature of the modified pyrgeometer housing,} measured
by a platinum resistance temperature sensor. The calibrated temperature
(IRxHT) is derived from the raw signal (IRxHTV)\sindex[var]{IRxHTV}
as described below:\\
\fbox{\begin{minipage}[t]{0.9\columnwidth}%
IRxHTV = %
\begin{minipage}[t]{0.85\columnwidth}%
voltage from a platinum resistance thermometer attached to the housing
of the pyrgeometer (V)%
\end{minipage}\\
$\{a_{4},\,a_{5}\}$ = calibration coefficients {[}$^{\circ}$C{]}\\
$V_{1}$ = 1 V (for consistency of units) \\
\\
\rule[0.5ex]{1\columnwidth}{1pt}

\begin{equation}
\mathrm{IRxHT}=a_{4}+a_{5}\log_{10}(\mathrm{\{IRxHTV\}/V_{1})}\label{eq:pyrgeometerHousingT}
\end{equation}
%
\end{minipage}}

\textbf{Calibrated Infrared Irradiance (W~m$^{-2}$): }\textbf{\uline{IRxC}}\index{IRxC}\sindex[var]{IRxC}
\\
\emph{The infrared irradiance measured by a Kipp \& Zonen CGR4 instrument,}\footnote{Prior to 2009, IRx and IRxC were used to denote measurements from
Eppley pyrgeometers. Processing methods for these obsolete variables
are described in Section \ref{sec:OBSOLETE-VARIABLES}; see p.~\pageref{EppleyReference}.}\emph{ }after application of a calibration function.\index{irradiance!long-wave}
The relationship between IRxV (V) and IRxC (W m$^{-2}$) is determined
by a calibration in which the CGR4 views a NIST-referenced source
over a range of sensor temperatures controlled by a cold bath. The
processing algorithm is described in the following box:\\
\fbox{\begin{minipage}[t]{0.9\columnwidth}%
IRxV\index{IRxV} = pyrgeometer output voltage (V)\\
IRxHT\index{IRxHT} = temperature of the instrument housing ($^{\circ}$
C)\\
$T_{0}$ = 273.15~K\\
$\{a_{1},\,a_{2},\,a_{3}\}$ = calibration coefficients \\
\\
\rule[0.5ex]{1\columnwidth}{1pt}
\begin{hangparagraphs}
\begin{equation}
\mathrm{IRxC}=(a_{1}\mathrm{\{IRxV\}}+a_{2})+a_{3}(\mathrm{\{IRxHT\}}+T_{0})^{4}\label{eq:pyrgeometerCalibration}
\end{equation}
\end{hangparagraphs}

%
\end{minipage}}

\textbf{Pyranometer Output (V):}\nop{VISxV}{}\textbf{ }\textbf{\uline{VISxV}}\index{VISxV}\sindex[var]{VISxV}\\
\emph{The voltage from a pyranometer,}\index{pyranometer} representing
visible irradiance. On the GV and C-130, Kipp \& Zonen CMP22 pyranometers\index{radiometer!Kipp & Zonen@Kipp \& Zonen}
measure visible irradiance. A high-quality quartz dome allows for
a wide spectral range, improved directional response, and reduced
thermal offsets. The spectral range is 0.32 to 3.6 \textmu m. The
pyranometers are usually flown in pairs, one looking upward and one
downward. On the C-130, these sensors are mounted on stabilized platforms\index{platform!stabilized}
that remain level during aircraft pitch and roll variations. They
are calibrated\index{pyranometer!calibration} pre- and post-project
at the Naval Research Lab (Bucholtz et al, 2008; see footnote \ref{fn:Bucholtz-2008}
on page \pageref{fn:Bucholtz-2008}) using a sun-tracking shadow device
and diffuse sunlight as a source. The letter \textquoteright x\textquoteright{}
denotes either bottom (B, nadir-viewing) or top (T, zenith-viewing).
The primary derived variable from this instrument is VISxC, below.\\

\textbf{Pyranometer Housing Temperature ($^{\circ}$C):}\nop{VISxHT}{}\textbf{
}\textbf{\uline{VISxHT}}\index{VISxHT}\sindex[var]{VISxHT}\\
\emph{The temperature of the modified housing unit of a pyranometer,}
measured by a platinum resistance temperature sensor. A calibrated
temperature (VISxHT) is derived from the raw signal, VISxHTV,\index{VISxHTV}\sindex[var]{VISxHTV}
which is normally not included in archive netCDF files. The equation
used for the calibration is VISxHT = $a_{1}+a_{2}\log_{10}$(\{VISxHTV\}/$V_{1}$)
where $V_{1}$is 1~V and $\{a_{1},\,a_{2}\}$ are calibration coefficients
having dimensions of {[}$^{\circ}$C{]}.\\

\textbf{Calibrated Visible Irradiance (W~m$^{-2}$):}\nop{VISxC}{}\textbf{
}\textbf{\uline{VISxC}}\index{VISxC}\sindex[var]{VISxC}\\
\emph{The visible irradiance\index{irradiance!visible} measured by
a Kipp \& Zonen CMP22 pyranometer.} The relationship between VISxV
(V) and VISxC (W m$^{-2}$) is determined by calibration procedures
in which the CMP22 views a clear sky source while a sun-tracking device
blocks direct solar radiation. The normal processing algorithm is
to apply a simple linear calibration, as follows:\\
\fbox{\begin{minipage}[t]{0.9\columnwidth}%
VISxV = voltage output by a pyranometer (V)\\
$a_{1}$ = linear calibration coefficient {[}W~m$^{-2}$~V$^{-1}${]}\\

\rule[0.5ex]{1\columnwidth}{1pt}

\begin{equation}
\mathrm{VISxC=a_{1}\{VISxV\}}\label{eq:pyranometerCalibration}
\end{equation}
%
\end{minipage}}\\

\textbf{Stabilized Platform Angles ($^{\circ}$):}\nop{SPxPitch}{}\nop{SPxRoll}{}\textbf{
SPxPitch,}\index{SPxPitch}\sindex[var]{SPxPitch} \textbf{SPxRoll}\index{SPxRoll}\sindex[var]{SPxRoll}\\
\emph{The pitch and roll angles of the stabilized platforms, relative
to the aircraft reference frame.} Upward- and downward-looking pyrgeometers
and pyranometers on the C-130 are mounted on stabilized platforms\index{platform!stabilized}
that compensate for aircraft pitch and roll. These variables record
the movement of the top (x=T) and bottom (x=B) platforms in response
to aircraft pitch and roll changes. The platforms are mounted with
2.85$^{\circ}$ downward pitch angle to compensate for the normal
upward pitch of the aircraft. The range of motion is $\pm5^{\circ}$
in pitch and $\pm10^{\circ}$ in roll. The sign convention is that
of the aircraft, for which nose-upward pitch and right-wing-down roll
are positive.

\end{hangparagraphs}


\subsection{Spectral Irradiance and Actinic Flux}

The HIAPER Atmospheric Radiation Package (HARP)\index{HIAPER Atmospheric Radiation Package}\index{HARP|see {HIAPER Atmospheric Radiation Package}}
includes separate components that measure spectral irradiance (both
upwelling and downwelling) and actinic flux.\index{flux!actinic}
The instrument is described at \href{http://www.eol.ucar.edu/instruments/hiaper-airborne-radiation-package}{this URL}.
Data are recorded on dedicated disk drives associated with the instrument,
not in the standard aircraft data-system files. This is an ancillary
data set, for which special Matlab and IDL analysis routines have
been developed, but the measurements are not merged into the netCDF
archives produced by EOL. For data access and assistance with analysis
routines, contact EOL/RAF data managers at \url{mailto:raf-dm@eol.ucar.edu}.

\subsection{Solar Angles}

The calculations\index{angles!solar} described in this group are
used primarily when interpreting the calibrated visible irradiance
(VISxC) but can be used by themselves or in conjunction with other
measurements that need them. For additional documentation see \href{http://nldr.library.ucar.edu/repository/assets/technotes/TECH-NOTE-000-000-000-175.pdf}{Bannehr and Glover, 1991, NCAR Technical Note NCAR/TN-364+STR}
and \href{http://www.esrl.noaa.gov/gmd/grad/solcalc/calcdetails.html}{this NOAA web site}.\footnote{The descriptions of SOLZE, SOLEL, and SOLAZ in Bulletin 9 were incorrect,
but the code in use has been consistent and correct and continues
to be used unchanged. For reference, that code is contained in the
nimbus subroutine 'solang.c'.}\index{Bulletin 25} The calculator at \href{http://www.esrl.noaa.gov/gmd/grad/solcalc/}{this link}
can also be used to find these angles from the position and time in
data files.\\

\begin{hangparagraphs}
\textbf{Solar Declination Angle (radians):}\nop{SOLAR}{}\textbf{
}\textbf{\uline{SOLDE\sindex[var]{SOLDE}\index{SOLDE}}} \\
\label{SOLDE}\emph{The solar declination angle\index{angle!solar declination},
the angular distance of the sun north of the earth's equator.} (Negative
values are south.) To obtain this, the solar hour angle is calculated
(taking leap years into account). \\
\\
\fbox{\begin{minipage}[t]{0.9\textwidth}%
$N$ = day number\sindex[lis]{N@$N$=day number} = %
\noindent\begin{minipage}[t]{1\columnwidth}%
number of days (corrected for leap years) since 1 January 1980

~~~~(including fractional day from UTC time)\\
= %
\noindent\begin{minipage}[t]{1\columnwidth}%
(year-1980){*}365+(int)(year-1980)/4+day\\
~~~+(hour+min/60.+sec/3600.)/24.+$M$%
\end{minipage}\\
where $M$=(int)($k+$(int)((month-i){*}30.6+$b$)\\
with \{i,b,k\}=\{1,0.5,0\} for month <= 2 

~~~and otherwise \{3, 59.5, (1 for leap years, else 0)\}%
\end{minipage}

$\theta_{h}$= UTC time expressed as radians after solar noon\\
$f$, $\alpha$, $\epsilon$ = internal-calculation variables\\
\{SOLDE\} = solar declination angle

\rule[0.5ex]{1\linewidth}{1pt}
\begin{lyxcode}
\begin{equation}
\theta_{h}=2\pi\frac{N}{365.25}\label{eq:HourAngle}
\end{equation}
\[
f=-0.031271-4.53963\times10^{-7}N+\theta_{h}
\]
\begin{eqnarray}
\alpha & = & \theta_{h}+4.900968+0.000349\,\sin(2f)+3.67474\times10^{-7}N\nonumber \\
 &  & +(0.033434-2.3\times10^{-9}N)\,\sin(f)\label{eq:SOLDEalpha}
\end{eqnarray}
\begin{equation}
\epsilon=0.409140-6.2149\times10^{-9}N\label{eq:SOLDEeps}
\end{equation}
\begin{equation}
\mathrm{{\{SOLDE\}}=}\arcsin(\sin\alpha\sin\epsilon)\label{eq:SOLDE}
\end{equation}

\end{lyxcode}
%
\end{minipage}}\\

\textbf{Solar Elevation Angle (radians): }\textbf{\uline{SOLEL}}\sindex[var]{SOLEL}\index{SOLEL}\\
\emph{The solar elevation angle\index{angle!solar elevation}, describing
how high the sun appears in the sky.} The angle is measured between
a line from the observer to the sun and the horizontal plane on which
the observer is standing. The elevation angle is negative when the
sun drops below the horizon, and the sum of the elevation angle and
the zenith angle is $\pi/2.$\\
\\
\fbox{\begin{minipage}[t]{0.9\textwidth}%
$\theta_{G}$ = Greenwich hour angle\sindex[lis]{thetaG@$\theta_{G}$=Greenwich hour angle}
(radians)\\
$\theta_{L}$ = local hour angle\sindex[lis]{thetaL@$\theta_{L}$=local hour angle}
(radians)\\
$N$ = day number (see SOLDE box above)\\
$Y$ = year (format as in 1980)

$\lambda$ = latitude\sindex[lis]{lambda@$\lambda$=latitude} (radians)\\
$\psi$ = longitude\sindex[lis]{psi@$\psi$=longitude} (radians)

$h$ = fractional hour = (hour + minute/60. + second/3600.)

$\alpha$~~~~~~see (\ref{eq:SOLDEalpha}) in the SOLDE box above\\
$\epsilon$~~~~~~see (\ref{eq:SOLDEeps}) in the SOLDE box\\
\{SOLDE\} = solar declination angle (radians) described above (Eq.~\ref{eq:SOLDE},
p.~\pageref{SOLDE})\\
\\
\rule[0.5ex]{1\linewidth}{1pt}
\begin{lyxcode}
\begin{equation}
\theta_{G}=\arctan(\frac{\sin\alpha\cos\epsilon}{\cos\alpha})\label{eq:GreenichHourAngle}
\end{equation}
\begin{eqnarray}
\theta_{L} & = & \theta_{G}+\psi-2\pi\frac{h}{24}-1.759335\label{eq:LHA}\\
 &  & -2\pi(\frac{N}{365}-Y+1980)-3.694\times10^{-7}N\nonumber 
\end{eqnarray}
\begin{equation}
\mathrm{\mathrm{\{SOLEL\}}=\arcsin\left(\sin\lambda\sin\mathrm{\{SOLDE\}+\cos\lambda}\cos\mathrm{\{SOLDE\}}\cos\theta_{L}\right)}\label{eq:SOLEL}
\end{equation}
\end{lyxcode}
%
\end{minipage}}

\textbf{Solar Zenith Angle (radians): }\textbf{\uline{SOLZE}}\sindex[var]{SOLZE}\index{SOLZE}\\
\emph{The angle of the sun from the zenith\index{angle!solar zenith},
or the solar zenith angle. }Cf. also the discussion of the solar elevation
angle, SOLEL. $\mathrm{\{SOLZE\}=(\pi/2)-\mathrm{\{SOLEL\}}}$ with
\{SOLEL\} given by (\ref{eq:SOLEL}) above.

\textbf{Solar Azimuth Angle (radians): }\textbf{\uline{SOLAZ}}\sindex[var]{SOLAZ}\index{SOLAZ}\\
\emph{The solar azimuth angle\index{angle!solar azimuth}, the angular
distance between due south and the projection of the line of sight
to the sun on the ground.} A positive solar azimuth angle indicates
a position east of south (i.e., morning).\\
\\
\fbox{\begin{minipage}[t]{0.95\textwidth}%
$\theta_{L}$ = local hour angle (radians): see (\ref{eq:LHA})\\
\{SOLDE\} = solar declination angle (radians): see (\ref{eq:SOLDE})\\
\{SOLEL\} = solar elevation angle (radians): see (\ref{eq:SOLEL})\\
\{SOLAZ\} = solar azimuth angle (radians)\\
\rule[0.5ex]{1\linewidth}{1pt}
\begin{lyxcode}
\begin{equation}
\mathrm{\{SOLAZ\}=\arcsin\left(\frac{\cos\mathrm{\{SOLDE\}\sin\theta_{L}}}{\cos\mathrm{\{SOLEL\}}}\right)}\label{eq:SOLAZ}
\end{equation}
If~sin(\{SOLAZ\})~<~sin(\{SOLDE\})/sin($\phi):$~

~~~~~~\{SOLAZ\}~$\leftarrow$$\pi/2-$\{SOLAZ\}
\end{lyxcode}
%
\end{minipage}}\\
\end{hangparagraphs}




\include{Section14}


\section{OBSOLETE VARIABLES\label{sec:OBSOLETE-VARIABLES}}

\hypertarget{OBSOLETE}{}RAF retired the ``GENPRO'' processor, the
software program previously used to produce data sets, in 1993, but
data files produced by that processor are still retained and available
for use. Also, there are some instruments that are now retired but
provided measurements in some archived data files. Obsolete variable
names that are associated only with GENPRO or a retired instrument
are discussed below, for reference and to facilitate use of old data
files. 
\begin{hangparagraphs}
\textbf{Unaltered Tape Time (s): }\textbf{\uline{TPTIME}}\sindex[var]{TPTIME (obsolete)}\index{TPTIME}\\
This variable is derived by converting the HOUR, MINUTE and SECOND
to elapsed seconds after midnight of the current day. If time increments
to the next day, its value is not reset to zero, but 86400 seconds
are added to produce ever-increasing values for the data set.

\textbf{Processor Time (s): }\textbf{\uline{PTIME}}\sindex[var]{PTIME (obsolete)}\index{PTIME}\\
This is an internal time variable created by the GENPRO processor.
It represents elapsed seconds after midnight. It differs from TPTIME
in that, after it has been set at the beginning of the data set, it
is incremented internally for each second of data processed. If duplicate
or missing raw data records exist, it can differ from TPTIME. It is
guaranteed to be a monotonically increasing and continuous series
of values.

\textbf{INS: Data System Time Lag (s): }\textbf{\uline{TMLAG}}\sindex[var]{TMLAG (obsolete)}\index{TMLAG}\\
TMLAG is the amount of time between the reference time of a Litton
LTN-5l Inertial Navigation System (INS) and the data system clock,
in seconds. TMLAG will always be greater than zero and less than 2.

\textbf{}%
\noindent\begin{minipage}[t]{1\columnwidth}%
\textbf{LORAN-C Latitude (}$\text{�}$\textbf{): }\textbf{\uline{CLAT}}\textbf{\sindex[var]{CLAT (obsolete)}\index{CLAT}}\\
\textbf{LORAN-C Longitude (}$\text{�}$\textbf{): }\textbf{\uline{CLON}}\textbf{\sindex[var]{CLON (obsolete)}\index{CLON}}\\
\textbf{LORAN-C Circular Error of Probability (n mi): }\textbf{\uline{CCEP}}\textbf{\sindex[var]{CCEP (obsolete)}\index{CCEP}}\\
\textbf{LORAN-C Ground Speed (m/s): }\textbf{\uline{CGS}}\textbf{\sindex[var]{CGS (obsolete)}\index{CGS}}\\
\textbf{LORAN-C Time (s): }\textbf{\uline{CSEC}}\textbf{\sindex[var]{CSEC (obsolete)}\index{CSEC}}\\
\textbf{LORAN-C Fractional Time (s): }\textbf{\uline{CFSEC\sindex[var]{CFSEC (obsolete)}\index{CFSEC}}}%
\end{minipage}\\
Before the advent of GPS, NCAR/RAF operated a LORAN-C receiver that
provided information on the position and groundspeed of the aircraft.
The measurements of latitude and longitude from this system are CLAT
and CLON, measured at 1 Hz and with positive values of longitude to
the east and positive values of latitude to the north. and CCEP provides
an estimate of the uncertainty in those measurements (in units of
nautical miles). A status word, CSTAT, was used to record a value
of 15 when the system was operational. The ground speed and reference
times were also recorded in the above corresponding variables. The
sum of CSEC and CFSEC represented the time of the measurement, which
was not always the time in the data file when the measurements were
recorded, 

\noindent\begin{minipage}[t]{1\columnwidth}%
\textbf{INS Latitude (}$\text{�}$\textbf{): }\textbf{\uline{ALAT}}\textbf{\sindex[var]{ALAT (obsolete)}\index{ALAT}}\\
\textbf{INS Longitude (}$\text{�}$\textbf{): }\textbf{\uline{ALON}}\textbf{\sindex[var]{ALON (obsolete)}\index{ALON}}\\
\textbf{Raw INS Ground Speed X Component (m/s): }\textbf{\uline{XVI}}\sindex[var]{XVI (obsolete)}\textbf{\index{XVI}}\\
\textbf{Raw INS Ground Speed Y Component (m/s): }\textbf{\uline{YVI}}\textbf{\sindex[var]{YVI (obsolete)}\index{YVI}}\\
\textbf{Raw INS True Heading (}$\text{�}$\textbf{): }\textbf{\uline{THI}}\textbf{\sindex[var]{THI (obsolete)}\index{THI}}\\
\textbf{INS Wander Angle (}$\text{�}$\textbf{): }\textbf{\uline{ALPHA}}\textbf{\sindex[var]{ALPHA (obsolete)}\index{ALPHA}}\\
\textbf{INS Platform Heading (}$\text{�}$\textbf{): }\textbf{\uline{PHDG\sindex[var]{PHDG (obsolete)}\index{PHDG}}}%
\end{minipage}

~~~~~~~~~\label{LTN51}These variables from the Litton LTN-51
Inertial Navigation System\index{IRU!Litton LTN-51} (INS) are analogous
to the modern variables discussed in section \ref{sec:INS}. The measurements
of latitude and longitude were provided with 1-Hz frequency and had
a resolution of 0.0014$\text{�}$, while the ground speed components
were provided at 10 Hz and had resolution equal to 0.012 m/s. The
X component of the ground speed was along the longitudinal axis of
the aircraft \emph{at the time of alignment,}  and the Y axis was
in the starboard direction at the time of alignment. PHDG recorded
the orientation of the platform relative to true north, with resolution
0.0028$\text{�}$. THI was the true heading of the aircraft, produced
at 5 Hz with resolution of 0.0014$\text{�}$. The ``wander angle''
is an INS-only variable that recorded the angle of the INS platform
x-axis relative to its original orientation; it ``wandered'' in
response to east-west motion of the aircraft on a spherical Earth. 

\textbf{Raw Aircraft Vertical Velocity (m/s): }\textbf{\uline{VZI}}\sindex[var]{VZI (obsolete)}\index{VZI}\\
This is an integrated output from an up/down binary counter connected
to the INS vertical accelerometer. Resolution is 0.012 m/s. Due to
changes in local gravity and accumulated errors, this often develops
a significant offset during flight. 

\textbf{Aircraft True Heading (}$\text{�}$\textbf{): }\textbf{\uline{THF}}\sindex[var]{THF (obsolete)}\index{THF}\\
This measurement of aircraft heading was derived from the angle between
the horizontal projection of the aircraft center and true north: THF
= PHDG + ALPHA. Resolution is 0.0028$\text{�}$. 

\noindent\begin{minipage}[t]{1\columnwidth}%
\textbf{Aircraft Ground Speed (m/s): }\textbf{\uline{GSF}}\textbf{\sindex[var]{GSF (obsolete)}\index{GSF}}\\
\textbf{Aircraft Ground Speed East Component (m/s): }\textbf{\uline{VEW}}\textbf{\sindex[var]{VEW (obsolete)}\index{VEW}}\\
\textbf{Aircraft Ground Speed North Component (m/s): }\textbf{\uline{VNS\sindex[var]{VNS (obsolete)}\index{VNS}}}%
\end{minipage}

~~~~~~~~~These variables have the same names as the modern
variables for ground speed. (Cf.~section \ref{sec:INS}.) GSF is
the magnitude of the ground speed determined by the INS, as derived
from XVI and YVI: \\
\[
\mathrm{GSF=\sqrt{\{XVI\}^{2}+\{YVI\}^{2}}}
\]
\\
VEW and VNS are the east and north projections of this ground speed,
derived using THF for the aircraft heading.

\noindent\begin{minipage}[t]{1\columnwidth}%
\textbf{Wind Speed (m/s): }\textbf{\uline{WSPD}}\textbf{\sindex[var]{WSPD (obsolete)}\index{WSPD}}\\
\textbf{Wind Direction (}$\text{�}$\textbf{): }\textbf{\uline{WDRCTN\sindex[var]{WDRCTN (obsolete)}\index{WDRCTN}}}%
\end{minipage}

~~~~~~~~~These variables are calculated from UI and VI, the
east and north components of the wind determined as described in RAF
Bulletin No.~23 and summarized in section \ref{sec:WIND}:\label{punch:10-1}

\begin{eqnarray*}
\mathrm{WS} & = & \sqrt{\mathrm{\{UI\}^{2}+\{VI\}^{2}}}\\
\mathrm{WD} & = & \mathrm{\frac{180^{\circ}}{\pi}atan2(-\{UI\},}-\{VI\})+180^{\circ}
\end{eqnarray*}
\\

\textbf{Raw Attack Force (Fixed Vane) (g): }\textbf{\uline{AFIXx}}\sindex[var]{AFIXx (obsolete)}\index{AFIXx}\\
AFIXx is an amplified output from a strain-gauge, fixed-vane sensor
mounted in the horizontal plane of the aircraft at the end of a gust
boom. The ``force'' on the vane (calibrated in ``equivalent grams''
at Jefferson County Airport gravity) varies as a function of the aircraft
attack angle and dynamic pressure. Here x refers to left or right.

\textbf{Raw Sideslip Force(Fixed Vane) (g): }\textbf{\uline{BFIXx}}\sindex[var]{BFIXx (obsolete)}\index{BFIXx}\\
BFIXx is an amplified output from a strain-gauge, fixed-vane sensor
mounted in the vertical plane of the aircraft at the end of a gust
boom. The ``force'' on the vane (calibrated in ``equivalent grams''
at Jefferson County Airport gravity) varies as a function of the aircraft
sideslip angle and dynamic pressure. Here x refers to top or bottom.

\textbf{Attack Angle (Fixed Vane) (}$\text{�}$\textbf{): }\textbf{\uline{AKFXx}}\sindex[var]{AKFXx (obsolete)}\index{AKFXx}\\
AKFXx is the angle of attack, computed from AFIXx and QCx (either
boom or gust dynamic pressure). An empirically derived function, HSSATK,
is used to determine the attack angle based upon wind tunnel test
data.

\textbf{Sideslip Angle (Fixed Vane) (}$\text{�}$\textbf{): }\textbf{\uline{SSFXx}}\sindex[var]{SSFXx (obsolete)}\index{SSFXx}\\
SSFXx is the sideslip angle, computed from BFIXx, and QCx (either
boom or gust dynamic pressure). An empirically derived function, HSSATK,
is used to determine the sideslip angle based upon wind tunnel test
data.

\noindent\begin{minipage}[t]{1\columnwidth}%
\textbf{Dynamic Pressure (Boom) (mb): }\textbf{\uline{QCB}}\textbf{\sindex[var]{QCB (obsolete)}\index{QCB},
}\textbf{\uline{QCBC}}\textbf{\sindex[var]{QCBC (obsolete)}\index{QCBC}}\\
\textbf{Dynamic Pressure (Gust Probe) (mb): }\textbf{\uline{QCG}}\textbf{\sindex[var]{QCG (obsolete)}\index{QCG},
}\textbf{\uline{QCGC\sindex[var]{QCGC (obsolete)}\index{QCGC}}}%
\end{minipage}

~~~~~~~~~These variables, measured by a differential pressure
gauge, record the difference between a pitot (total) pressure and
a static pressure. The QCBC and QCGC values are corrected for local
flow-field distortion. The boom and gust probe measurements referred
to the same aircraft structure. The different designations used for
those measurements specified the transducer used and its location.
In the gust probe dynamic pressure measurement (QCG), a Rosemount
Model 1332 differential pressure transducer was located closer to
the sensor in the gust probe itself, whereas in the boom measurement
(QCB), a Rosemount Model 1221 pressure transducer was typically located
in the aircraft nose.

\textbf{Ambient Temperature }($^{\circ}C$):\textbf{\uline{ ATC}}\textbf{\index{ATC}}\sindex[var]{ATC}\\
A variable obtained by combining the avionics temperature on the GV,
AT\_A, with a Rosemount temperature, so that the absolute value tracked
AT\_A but faster response was provided by the Rosemount temperature.
This was used in some early GV projects because there were unresolved
problems with the data-system temperature sensors and it was thought
that AT\_A provided a more accurate result, but AT\_A was filtered
to have slow response to it was combined with the faster-response
signal from the Rosemount sensor.

\textbf{Total Temperature ($^{\circ}$C): TTx\index{TTx}\sindex[var]{TTx}}\\
This variable was used before 2014 for measurements of the recovery
temperature, for which the variable is now \textbf{RTx}. Because the
quantity measured is not the total temperature,\index{temperature!total}
the variables TTx were replaced by RTx, but the meaning historically
was the same as that now described for \textbf{RTX}, apart from how
humidity is now handled. 

\textbf{Total Temperature, Reverse Flow ($^{\circ}C$): }\textbf{\uline{TTRF}}\sindex[var]{TTRF (obsolete)}\index{TTRF}\\
TTRF is the recovery temperature from a calibrated NCAR reverse-flow
temperature sensor\index{reverse-flow temperature sensor}, for which
the housing was designed to separate water droplets and protect the
element from wetting in cloud. 

\textbf{Total Temperature (Fast Response) ($^{\circ}C$): }\textbf{\uline{TTKP}}\sindex[var]{TTKP (obsolete)}\index{TTKP}\\
This is the output of recovery temperature from the NCAR fast-response
temperature probe, originally designed by Karl Danninger. (See discussion
of total temperature in section \ref{subsec:PTq}.)

\textbf{Ambient Temperature ($^{\circ}C$): }\textbf{\uline{ATRF}}\sindex[var]{ATRF (obsolete)}\index{ATRF}\\
The ambient temperature computed using the NCAR reverse-flow temperature
sensor. (See discussion in Section \ref{subsec:PTq} above.)

\textbf{Ambient Temperature (Fast Response) ($^{\circ}C$): }\textbf{\uline{ATKP}}\sindex[var]{ATKP (obsolete)}\index{ATKP}\\
The ambient temperature computed using the fast-response temperature
probe. (See discussion of ambient temperature in section \ref{subsec:PTq}.)

\textbf{Raw Cloud Technology (Johnson-Williams) }\\
\textbf{Liquid Water Content ($g/m^{3}$): }\textbf{\uline{LWC}}\sindex[var]{LWC (obsolete)}\index{LWC}\\
This is the raw output of a Johnson-Williams\index{Johnson-Williams sensor}
liquid water content sensor converted to units of grams per cubic
meter. The Johnson-Williams indicator measures the evaporative cooling
caused by the latent heat of vaporization of droplets contacting the
heated sensing element by sensing changes in its resistance as it
cools. Through calibration this resistance is converted to a liquid
water content. A ``compensation'' wire is also mounted in the J-W
sensor, parallel to the droplet stream, to compensate for cooling
effects of the airstream. Typically the instrument is set for a true
airspeed of 200 knots. The instrument must be zeroed in ``cloud-free
air.'' The Johnson-Williams liquid water content sensor is designed
for the cloud droplet spectrum. There is some evidence to indicate
that droplets larger than 30 $\mu m$ are shed before completely vaporizing
on the sensor element. This tends to underestimate the liquid water
content.

\textbf{Corrected Cloud Technology (Johnson-Williams) }\\
\textbf{Liquid Water Content (g/M3): }\textbf{\uline{LWCC}}\sindex[var]{LWCC (obsolete)}\index{LWCC}\\
This is the corrected liquid water content obtained by using the aircraft's
true airspeed after removing the zero offset: LWCC=LWC$U_{a}/U_{ref}$
where $U_{a}$ is the true airspeed of the aircraft and $U_{ref}$
is the true airspeed set on the dial of the instrument. $U_{ref}$
was normally 200 kts = 102.88889 m/s.

\textbf{Indicated Airspeed (knots): }\textbf{\uline{IAS\index{indicated airspeed}\index{airspeed!indicated}\sindex[var]{IAS}}}\\
In some old data files, a variable representing the indicated airspeed
was included because this was used for some derived variables. The
indicated airspeed is the airspeed that would produce the observed
difference between dynamic and static pressure under standard conditions
of 1013.25 mb and $15^{\circ}$C.

\textbf{Water Vapor Pressure (mb): }\textbf{\uline{EDPC}}\sindex[var]{EDPC (obsolete)}\index{EDPC}\\
This is a derived intermediate variable used in the calculation of
several derived thermodynamic variables. The vapor pressure over a
plane water surface is obtained by the method of Paul R. Lowe (1977),
a derived, sixth-order, Chebyshev polynomial fit to the Goff-Gratch
Formulation (1946) as a function of temperature expressed in $^{\circ}C$.
The error is much less than 1\% over the range -50$\text{�}$C to
+50$\text{�}$C. EDPC was calculated using this method for most RAF
research projects between 1993 and 1996. This variable did not have
the enhancement factor applied that was discussed in Appendix C of
Bulletin 9. A variable of the same name but calculated differently
replaced this in 1996, and with changes described in Section \ref{sec:State Variables}
continues in use, recently replaced by EWx.\label{punch:10-2}\\
\\
\fbox{\begin{minipage}[t]{0.9\textwidth}%
A. T $<$ -50 C:\\
\begin{eqnarray*}
\mathrm{EDPC} & = & 4.4685+T(0.27347+T\{6.83811\times10^{-3}\\
 & + & T[8.7094x10^{-5}+T(5.63513x10^{-7}+T\,1.47796\times10^{\mbox{-9}})]\})
\end{eqnarray*}
\\
B. T $>$= -50$\text{�}$C:\\
\begin{eqnarray*}
\mathrm{EDPC} & = & 6.107799961+T\,[0.4436518521+T(0.01428945805\\
 & + & T\{2.650648471\times10^{-4}+T\,[3.031240396\times10^{-6}\\
 & + & T(2.034080948\times10^{-8}+T\,6.136820929\times10^{-11})]\})]
\end{eqnarray*}
%
\end{minipage}}

\textbf{Cryogenic Hygrometer Inlet Pressure (hPa) and Frost Point
Temperature ($^{\circ}C$): }\textbf{\uline{CRHP\sindex[var]{CRHP}\index{CRHP}}}
\textbf{and }\textbf{\uline{VCRH}}\sindex[var]{VCRH}\index{VCRH}
(obsolete)\\
These are measurements made directly in the chamber of the cryogenic
hygrometer\index{hygrometer!cryogenic}, a now obsolete cabin-mounted
instrument connected to outside air by an inlet line. CRHP is the
pressure and VCRH is the frost-point temperature measured inside that
chamber. VCRH is determined from a third-order calibration equation
applied to the voltage measured by the instrument. \label{punch:10-3}

\textbf{Corrected Cryogenic Frost Point Temperature and Dew Point
Temperature ($\text{�}$C): }\textbf{\uline{FPCRC\sindex[var]{FPCRC}\index{FPCRC}}}\textbf{
and }\textbf{\uline{DPCRC}}\sindex[var]{DPCRC}\index{DPCRC}\\
\emph{The frost point or dew point determined after corrections are
applied to the direct measurements from a cryogenic hygrometer. }These
measurements were from a now obsolete instrument but the variables
are included here because they appear in some old data files. To obtain
estimates of the ambient frost point and dew point, the measurements
made inside the chamber of the cryogenic hygrometer (CVRH and CRHP)
must be corrected for the difference in water vapor pressure between
that chamber and ambient conditions. The ratio of the chamber pressure
to the ambient pressure is assumed to be the same as the ratio of
the chamber vapor pressure to the ambient vapor pressure. The vapor
pressure in the chamber was determined from the Goff-Gratch (1946)
equation\footnote{Goff, J. A., and S. Gratch (1946) Low-pressure properties of water
from \textminus 160 to 212 \textdegree F, referenced and used in the
Smithsonian Tables (List, 1980).} for saturation vapor pressure with respect to a plane ice surface.
This vapor pressure was then used with CRHP and a measure of the ambient
pressure (PSXC) to determine the vapor pressure in the outside air,
and this was converted to an equivalent dew-point. The instrument
was only used for measurements of frost point\index{frost point}
less than -15$^{\circ}$C because it did not function well above that
frost point. \label{punch:10-4}The steps are documented below:\\
\fbox{\begin{minipage}[t]{0.9\textwidth}%
VCRH = frost point inside the cryogenic hygrometer ($^{\circ}C$)\\
CRHP = pressure inside the chamber of the cryogenic hygrometer (hPa)\\
PSXC = reference ambient pressure (hPa)\\
f$_{i}$ = enhancement factor (see Appendix C of Bulletin 9)\\
$F_{1}$($T_{d}$) =Goff-Gratch formula for vapor pressure at dew
point $T_{d}$\\
$F_{2}(T_{f})$ = Goff-Gratch formula for vapor pressure at frost
point $T_{f}$ \\
$T_{3}$ = temperature at the triple point of water = 273.16 K\\
\\
\\
\rule[0.5ex]{1\linewidth}{1pt}

chamber vapor pressure $e_{ic}$ (hPa):

\[
e_{ic}=(6.1071\,\mathrm{mb})\times10^{A}
\]

\begin{eqnarray*}
\mathrm{where}\,\,\,A & = & -9.09718\left(\frac{T_{3}}{\mathrm{VCRH}+T_{3}}-1\right)\\
 & + & 3.56654\log_{10}\left(\frac{T_{3}}{\mathrm{VCRH}+T_{3}}\right)\\
 & + & 0.876793\left(1-\frac{\mathrm{VCRH}+T_{3}}{T_{3}}\right)
\end{eqnarray*}

ambient vapor pressure $e_{a}$ (hPa):

\[
e_{a}=e_{ic}\left(\frac{\mathrm{PSXC}}{\mathrm{CRHP}}\right)f_{i}
\]

ambient dew and frost point DPCRC and FPCRC: (iterative solution)

\begin{eqnarray*}
e_{a} & = & F_{1}\left(\mathrm{DPCRC}\right)\\
 & = & F_{2}\left(\mathrm{FPCRC}\right)
\end{eqnarray*}

\begin{lyxcode}
\end{lyxcode}
%
\end{minipage}}

\textbf{Voltage Output From the Lyman-alpha Sensor (V): }\textbf{\uline{VLA}}\sindex[var]{VLA}\textbf{\index{VLA},
}\textbf{\uline{VLA1}}\index{VLA1} (obsolete)\\
\emph{The voltage output from the Lyman-alpha absorption hygrometer}.\index{hygrometer!Lyman-alpha}
This instrument provided fast-response, high-resolution measurements
of water vapor density. (If a second sensor was used, a 1 was added
to the variable name associated with the second sensor.) The sensors
are now obsolete.

\textbf{Voltage Output from the UV Hygrometer (V): }\textbf{\uline{XUVI}}\sindex[var]{XUVI}\textbf{\index{XUVI}}\\
\index{hygrometer!UV}\emph{The voltage from a modern (as of 2009)
version of the Lyman-alpha hygrometer, which provides a signal that
represents water vapor density.} The instrument also provides measurements
of pressure and temperature inside the sensing cavity; they are, respectively,
\textbf{\uline{XUVP}}\sindex[var]{XUVP}\textbf{\index{XUVP}}
and \textbf{\uline{XUVT}}\sindex[var]{XUVT}\textbf{\index{XUVT}}.
These variables and the processing algorithm below have now been replaced
by XSIGV\_UVH and the algorithm discussed  with the variable EW\_UVH.\\
\fbox{\begin{minipage}[t]{0.95\columnwidth}%
XUVI = output from the UV Hygrometer, after application of calibration
coefficients\\
DPXC\index{DPXC} = corrected dewpoint from some preferred source,
$^{\circ}$C\\
ATX\index{ATX} = preferred temperature, $^{\circ}$C\\
RHODT\index{RHODT} =water vapor density determined by a chilled-mirror
sensor\\
Tau = time constant for the exponential update (typically 300 s)\\
\rule[0.5ex]{1\columnwidth}{1pt}
\begin{lyxcode}
For~valid~measurements:\footnote{i.e., DPXC<ATX and XUVI and RHODT are not missing}~

~~~~Offset~+=~(RHODT-XUVI-Offset)/Tau~\\
RHOUV~=~XUVI~+~Offset
\end{lyxcode}
%
\end{minipage}}

\textbf{Raw Pyrgeometer Output (W\,m$^{-2}$): }\textbf{\uline{IRx}}\sindex[var]{IRx}\index{IRx}\\
\label{EppleyReference}A pyrgeometer\index{pyrgeometer} manufactured
by Eppley Laboratory, Inc. measures \index{radiation!long-wave}long-wave
irradiance using a calibrated thermopile. It has a coated glass hemisphere
that transmits radiation in a bandwidth between 3.5 $\mu m$ and 50
$\mu m$. It is calibrated at RAF according to procedures specified
by Albrecht and Cox (1977). (See the reference in the next paragraph.)
The pyrgeometers are usually flown in pairs, one up-looking and one
down-looking. The letter 'x' denotes either bottom (B) or top (T).\\

\textbf{Corrected Infrared Irradiance (W\,m$^{-2}$): }\textbf{\uline{IRxC}}\sindex[var]{IRxC}\index{IRxC}\\
Because the pyrgeometer measures net radiation, IRx must be corrected
for emission from the dome covering the sensor and for emission from
the thermopile itself. IRxC is the corrected infrared irradiance,
determined following procedures of \href{http://journals.ametsoc.org/doi/pdf/10.1175/1520-0450\%281977\%29016\%3C0190\%3APFIPP\%3E2.0.CO\%3B2}{Albrecht and Cox, 1977}.
. \\
\fbox{\begin{minipage}[t]{0.9\textwidth}%
IRx = raw pyrgeometer output {[}W\,m$^{-2}${]}\\
$T_{D}$ = dome temperature {[}K{]}\\
$T_{S}$ = ``sink'' temperature (approx.~the thermopile temperature)
{[}K{]}\\
$\epsilon$ = emissivity of the thermopile (dimensionless) = 0.986\\
$\beta$ = empirical constant dependent on the dome type = 5.5\\
$\sigma$ = Stephan-Boltzmann constant = 5.6704$\times10^{-8}$ W\,m$^{-2}$K$^{-4}$\\
\\
\rule[0.5ex]{1\linewidth}{1pt}
\[
\mathrm{IRxC}=\mathrm{IRx}-\beta\sigma(T_{D}^{4}-T_{S}^{4})+\epsilon\sigma T_{S}^{4}
\]
%
\end{minipage}}\\

\textbf{Shortwave Irradiance (W/m$^{2}$): }\textbf{\uline{SWx}}\sindex[var]{SWx}\index{SWx}\\
An Eppley Laboratory, Inc., pyranometer\index{pyranometer} measures
\index{radiation!short wave}short-wave irradiance. The dome normally
used is UG295 glass, which gives wide coverage of the solar spectrum
(from 0.285 $\mu m$ to 2.8 $\mu m$). Different bandwidths can be
obtained by use of different glass domes, available from RAF upon
request. (See Bulletin No. 25.) The pyranometers are usually flown
in pairs, one up-looking and one down-looking. They are calibrated
periodically at the NOAA Solar Radiation Facility in Boulder, Colorado.
The letter 'x' denotes either bottom (B) or top (T).

\textbf{Corrected Incoming Shortwave Irradiance (W/m$^{2}$): }\textbf{\uline{SWTC}}\sindex[var]{SWTC}\index{SWTC}\\
The down-welling shortwave irradiance measured by the difference between
SWT and SWB) is corrected to take into account the sun angle and small
variations in the aircraft attitude angles (pitch and roll). The correction
is limited to $\pm6^{\circ}$ in either angle, so these measurements
should be considered invalid beyond these limits. This is the derived
output of incoming (down-welling) shortwave irradiance, taking into
account both solar position (sun angle) and modest variations in aircraft
attitude (at present, restricted to less than 6$\text{�}$ in pitch
and/or roll). (For more information, refer to \href{http://www.eol.ucar.edu/raf/Bulletins/bulletin25.html}{RAF Bulletin 25}.)\index{Bulletin 25}\label{punch:10-5}

\textbf{Ultraviolet Irradiance (W/m$^{2}$): UVx}\sindex[var]{UVx}\index{UVx}\\
A pair of UV radiometer/photometers measure either down-welling (x=T)
or up-welling (x=B) irradiance in the ultraviolet, approximately from
0.295 $\mu m$ to 0.385 $\mu m$. These units are periodically returned
to the Eppley Laboratories for recalibration. 

\textbf{Raw Carbon Monoxide Concentration (ppb): }\textbf{\uline{CO}}\index{CO}\sindex[var]{CO}\\
CO is the uncorrected output of the TECO model 48 CO analyzer. \label{punch:10-6}This
instrument measures the concentration of CO by gas filter correlation.
The optics of the version operated by the RAF have been modified to
increase the light through the absorption cell, and a zero trap has
been added that periodically removes CO from the sample air stream
to obtain an accurate zero. This permits correction for the significant
temperature-dependent drift of the zero level of the measurement.

\textbf{}%
\noindent\begin{minipage}[t]{1\columnwidth}%
\textbf{Carbon Monoxide Analyzer Status (V): }\textbf{\uline{CMODE}}\index{CMODE}\sindex[var]{CMODE}\\
\textbf{Carbon Monoxide Baseline Zero Signal (V): }\textbf{\uline{COZRO}}\textbf{\index{COZRO}}\sindex[var]{COZRO}\textbf{}\\
\textbf{Raw Carbon Monoxide, Baseline Corrected (V): }\textbf{\uline{COCOR\index{COCOR}}}\sindex[var]{COCOR}%
\end{minipage}\\
CMODE records if the CO analyzer is supplied with air from which CO
has been removed and so is recording its zero level. When CMODE is
less than 0.2 V, the instrument is in the normal operational mode,
and when CMODE is greater than 8.0 V the instrument is in the ``zero''
mode. When measurements are processed, the zero-mode signals are represented
by a cubic spline to obtain a reference baseline for the signal (COZRO),
and this baseline is subtracted from the measured value (CO) to obtain
COCOR. This variable still jumps to zero periodically and does not
include the calibration that enters the following variable, COCAL. 

\textbf{Corrected Carbon Monoxide Concentration (ppmv): }\textbf{\uline{COCAL}}\index{COCAL}\sindex[var]{COCAL}\\
\label{punch:10-7}The calibrated signal from the CO instrument after
correction for drift of the baseline and after application of the
appropriate calibration coefficients to produce units of ppmv. The
quality of the baseline fit can be judged by examining the offset
at the zero points. If there are relatively small changes in the baseline,
the zero offset will be only a few ppbv. If there have been rapid
changes in the baseline, the zero offset can be up to 50 ppbv. The
magnitude of the offset at the zero values gives a good measure of
uncertainty in the data set. The detection limit is 10 ppbv, with
an uncertainty of $\pm15\%$. At 1 Hz, data will have considerable
variability, so 10-s averaging is often useful when the measurements
are used for analysis. 

\textbf{Raw Chemiluminescent Ozone Signal (V): }\textbf{\uline{O3FS}}\index{O3FS}\sindex[var]{O3FS}\\
\emph{Voltage output from the chemiluminescence ozone instrument,}
which operates on the basis of reacting nitric oxide with ozone and
detecting the resulting chemiluminescence.

\textbf{Derived Supercooled Liquid Water Content (g/m$^{3}$): }\textbf{\uline{SCLWC}}\sindex[var]{SCLWC}\index{SCLWC}\label{SCLWC}\\
This variable is the supercooled liquid water content obtained from
the change in accreted mass on the Rosemount 871F ice-accretion probe
over one second. The output is not valid during the probe deicing
cycle. This cycle is apparent in the RICE output (a peak followed
by a decrease to near zero). Supercooled liquid water content is determined
by first calculating a water drop impingement rate which is a function
of the effective surface area, the collection efficiency, the true
airspeed, and the supercooled liquid water content. The impingement
rate obtained is equated to the accreted mass of ice collected by
the probe in one second (empirical voltage/mass relationship). The
resulting equation is solved for supercooled water content. This calculation
is not included in normal processing or special processing, but some
users of the instrument use an approach like the following to calculate
supercooled liquid water:\label{punch:10-8}\\
\fbox{\begin{minipage}[t]{0.9\textwidth}%
A = \sindex[lis]{A=area}effective surface area of the probe (m$^{2}$)\\
$\Delta t$ = time interval\sindex[lis]{Deltat@$\Delta t$=time interval}\sindex[lis]{t@$t$=time}
during which an increment of mass accretes (s)\\
$\Delta m$ = mass\sindex[lis]{m@$m$=mass} of ice accreted on the
probe in the time interval $\Delta t$ (g)\\
$U_{a}$ = true airspeed (m/s)\\
\\
\\
\rule[0.5ex]{1\linewidth}{1pt}

\[
\mathrm{SCLWC}=AU_{a}\frac{\Delta m}{\Delta t}
\]
%
\end{minipage}}\\

\textbf{FSSP-100 Fast Resets }(number per sample interval):\textbf{
}\textbf{\uline{FRST}}\textbf{\sindex[var]{FRST}\index{FRST},
}\textbf{\uline{FRESET}}\textbf{\sindex[var]{FRESET}\index{FRESET}}\\
\emph{The rate at which fast resets occur in an FSSP-100 probe. }The
FSSP-100\index{FSSP-100!fast resets}\index{resets!FSSP-100} records
events called ``fast resets'' that occur when a particle traverses
the beam outside the depth-of-field and therefore is not accepted
for sizing. To avoid the processing time associated with sizing, the
probe resets quickly in this case, but there is still some dead time\index{FSSP-100!dead time}
when the probe cannot record another event. Fast resets consume a
time determined by circuit characteristics, and that time has been
determined in laboratory tests of the FSSP circuitry. This variable
is needed in addition to the ``Total Stobes'' to determine what
fraction of the time the probe is unable to accept another particle,
and this ``dead time'' enters calculation of the concentration for
the original (old) FSSP. \\

\textbf{FSSP-100 Total Strobes }(number per sample interval)\textbf{:
}\textbf{\uline{FSTB}}\textbf{\sindex[var]{FSTB}\index{FSTB},
}\textbf{\uline{FSTROB}}\sindex[var]{FSTROB}\index{FSTROB}\\
\emph{The rate at which strobes are generated in an FSSP-100 probe.
}A ``strobe'' is generated in the FSSP-100\index{FSSP-100!total strobes}\index{strobes!FSSP}
whenever a particle is detected within its depth-of-field. Not all
such particles are accepted for inclusion in the size distribution,
however, because some pass through the outer regions of the illuminating
laser beam and therefore produce shorter and smaller-amplitude pulses
than those passing through the center of the beam. The probe maintains
a running estimate of the average transit time and rejects particles
with transit times shorter than this average. The total number of
strobes recorded is therefore more than the number of sized particles,
and the ratio of strobes to accepted particles can indicate quality
of operation of the probe. Also, the strobes require processing and
so contribute to the dead time of the probe, affecting the concentration
unless a correction is made. See \href{http://www.eol.ucar.edu/raf/Bulletins/bulletin24.html}{RAF Bulletin 24}\index{Bulletin 24}
for more discussion on the operation of the ``old'' FSSP.\\

\textbf{FSSP-100 Beam Fraction }(dimensionless)\textbf{: }\textbf{\uline{FBMFR}}\sindex[var]{FBMFR}\index{FBMFR}\\
\index{FSSP-100!beam fraction}\emph{The ratio of the number of velocity-accepted
particles (particles that pass through the effective beam diameter)
to the total number of particles detected in the depth-of-field of
the beam (the total strobes).} See the discussion of Total Strobes
for more information.\\
 \\
\fbox{\begin{minipage}[t]{0.9\textwidth}%
\{AFSSP\}$_{i}$ = valid particles sized in size interval i\\
\{FSTROB\} = strobes generated by particles in the depth-of-field,
\\
\hspace*{0.7in}per sample interval\\
\\
\rule[0.5ex]{1\linewidth}{1pt}
\[
\mathrm{FBMFR=\{AFSSP\}/\{FSTROB\}}
\]
%
\end{minipage}}\\

\textbf{FSSP-100 Calculated Activity Fraction} (dimensionless):\textbf{
}\textbf{\uline{FACT}}\index{FACT}\sindex[var]{FACT}\\
This variable\index{FSSP-100!activity} represents the fraction of
the time that the FSSP is unable to count and size particles (its
``dead time\index{dead time!FSSP}\index{FSSP-100!dead time}'').
The activity fraction is not measured directly but is estimated from
fast resets and total strobes along with measurements of the dead
times associated with each (as determined in laboratory tests). The
characteristic times are in the NetCDF header (for recent projects).
.\\
\noindent\fbox{\begin{minipage}[t]{1\columnwidth - 2\fboxsep - 2\fboxrule}%
FSTROB = strobes generated by particles in the depth-of-field, \\
\hspace*{0.7in}per sample interval\\
FRESET = ``fast resets'' generated per sample interval\\
$t_{1}$ = slow reset time (for each strobe)\\
$t_{2}$ - fast reset time (for each fast reset)\\
\\
\rule[0.5ex]{1\linewidth}{1pt}

\[
FACT=\{FSTROB\}\,t_{1}+\mathrm{\{FRESET\}}\,t_{2}
\]
%
\end{minipage}}\\
\\

\textbf{PCAS Raw Activity }(dimensionless):\textbf{ }\textbf{\uline{AACT}}\uline{\sindex[var]{AACT}\index{AACT},
}\textbf{\uline{PACT}}\sindex[var]{PACT}\index{PACT}\\
The PCAS probe provides this measure of dead time, the time that the
probe is unable to sample particles because the electronics are occupied
with processing particles. The manufacturer suggests that the actual
dead time ($f_{PCAS}$) is given by the following formula, which is
used in determining concentrations\index{concentration!PCAS} for
the PCAS:\\
\[
f_{PCAS}=0.52\frac{\mathrm{\{PACT\}}}{F_{PCAS}}
\]
\\
where $F_{PCAS}=1024\,s^{-1}$. However, PACT (or AACT) is the variable
archived in the data files. \\
\end{hangparagraphs}




\clearpage\phantomsection\label{sec:Symbols}

\addcontentsline{toc}{section}{List of Symbols}
\printindex[lis]{}
\clearpage
\phantomsection
\addcontentsline{toc}{section}{Variable Names}
\printindex[var]{}
\clearpage
\phantomsection
\addcontentsline{toc}{section}{Index}
\printindex[idx]{}
\cleardoublepage
%\begin{appendices}
\appendix

\section*{Suggested Additional Steps}

\begin{longtable}[c]{|c|>{\raggedright}p{12cm}|>{\centering}p{3cm}|}
\hline 
\textbf{page} & \textbf{suggested action} & \textbf{who?}\tabularnewline
\hline 
\endhead
\hline 
\pageref{Punch1.1} & Add, to constants table, a reference to what has been in use previously.
(See Code.amlib as saved 2011, for examples) & WAC -- not sure if needed? \tabularnewline
\hline 
\pageref{punch1.2} & Ask Teresa and Mike R. to review the discussion re trace-gas units & done, MR\tabularnewline
\hline 
\pageref{punch1.3} & Get info from Chris W describing interpolation and time adjustments,
for inclusion as an addition to the section on times. Revise section. & \tabularnewline
\hline 
\pageref{punch1.4} & Get place to put algorithm notes, and include links to those additional
discussions in this document. & partly done\tabularnewline
\hline 
\pageref{punch1.4} & In algorithm boxes, when variables are referenced, make those references
active links to the discussion of the variable & WAC: use hyperlink\{\}\{\}\tabularnewline
\hline 
\pageref{punch3.1} & Get description of the history of the C-130 INS, with characteristics
for the Litton at least. & \tabularnewline
\hline 
\pageref{punch3.3} & Add history of GPS systems: What was used when (C-130 at least) & \tabularnewline
\hline 
\pageref{punch3.4} & Add/clarify section on height-above-terrain; modify to ref. geoid.
Need to change HeightAboveTerrain() script. & partly done, WAC,\tabularnewline
\hline 
\pageref{punch3.4-1} & check/clarify discussion of height-above-geoid and, generally, geopotential
vs geometric vs geoid height  & done - WAC\tabularnewline
\hline 
 & Add a variable representing geopotential height and change DVALUE
to be based on it minus PALT & WAC - done (proposed)\tabularnewline
\hline 
\pageref{punch3.5} & Clarify meaning of mode and status for old GPS units, and if used
anymore & \tabularnewline
\hline 
\pageref{punch3.6} & Add new section on ALTC? Info is there in comments. Implement? & \tabularnewline
\hline 
\pageref{punch3.8} & should there be a vertical velocity of the AC based on data-system
GPS? ROC as used for reprocessing, and WIR as backup to WIC? & \tabularnewline
\hline 
\pageref{punch4.1} & Check/update sensors used on both aircraft. & \tabularnewline
\hline 
\pageref{punch4.2} & Add to historical description of PCORs, esp. re subroutine references
(QCF, MACH\_A, ADIFR) & \tabularnewline
\hline 
\pageref{punch4.3} & Add a discussion of the additional corrections to QCR that could make
this less sensitive to AOA? Algorithm is developed and documented;
implement? Coefficients in ProcessingAlgorithms.pdf are based on ARISTO2016
flight 6. & WAC - done (proposed)\tabularnewline
\hline 
\pageref{punch3.10} & Suggestion: consider ALT\_G and avoid ALT for GPS avionics variable & \tabularnewline
\hline 
\pageref{punch:3.11} & Consider change to spherical geometry for distance north and east
of reference point because range of GV is so great & \tabularnewline
\hline 
\pageref{punch:3-12} & need to explain how the two measures of longitude, with high and low
resolution, are used together. & \tabularnewline
\hline 
\pageref{punch:3-13} & Need to implement the discussion re correction for the displacement
of the GPS antenna from the INS. When done, need to add LG=-4.30 m
to the attributes for GGVSPD, GGVEW, GGVNS (GV) and get appropriate
values for the C-130 & WAC - done (proposed)\tabularnewline
\hline 
\pageref{punch:3-14} & Get Dick Friesen or someone to review and update the discussion of
GPS GSTAT & \tabularnewline
\hline 
\pageref{punch:3-15} & Revise the values listed for the complementary-filter feedback to
match what is used now -- better with lower values than listed & \tabularnewline
\hline 
\pageref{punch:3-16} & It might be useful to disable the roll test in gpsc.c, now that GPS
is better than when this was implemented & \tabularnewline
\hline 
\pageref{punch:4-4} & Goodrich Technical Report 5755: should we get permission and post
this? (FAAM has it posted) & \tabularnewline
\hline 
\pageref{punch:4-5} & In-cloud air T radiometer: could use more detail re the processing
algorithm & \tabularnewline
\hline 
\pageref{punch:4.6} & Check all the complex M-K section, esp. $T_{k}$ and DP interp. function & partly done - WAC\tabularnewline
\hline 
\pageref{punch:4-7} & Consider changing name to FP\_CR2 in preference to MIRRORT\_CR2? & \tabularnewline
\hline 
\pageref{punch:4-8} & For CONCV\_VXL, I think we need cal coefficients and equations used  & \tabularnewline
\hline 
\pageref{punch:4-9} & For RHOx, the code now uses 216.68 instead of 100000/461.5228=216.674
as specified here; change? & \tabularnewline
\hline 
\pageref{punch:4-10} & Check that current code uses the modified PCOR function with humidity
correction and early-error corrected & \tabularnewline
\hline 
\pageref{punch:4-11} & The PSURF definition references PSFDC; replace with PSXC? & \tabularnewline
\hline 
\pageref{punch:4-12} & ``ATTACK'' and ``SSLIP'' differ from other ``preferred'' variables
by not having ``X'' at the end. Consider name change? & \tabularnewline
\hline 
\pageref{punch:4-13} & The variable WIC is described as ``GPS-corrected'' but that is misleading
because it is really based, for aircraft motion, solely on GPS in
recent usage (where dependence is on GGVSPD). Contrast to WDC/WSC
which are really GPS-corrected. Suggest a different name, like ``Wind
Vector, Vertical Component, using GPS''? & \tabularnewline
\hline 
\pageref{punch:5-1} & Re Gerber probe, I didn't find code for this; need to describe the
algorithm. & \tabularnewline
\hline 
\pageref{punch:5-2} & It would be useful to update Bulletin 24 re hydrometeor spectrometers,
with info from Bansemer and reference to work by Korolev, Strapp,
Jensen, etc. & \tabularnewline
\hline 
\pageref{punch:5-3} & Is ``PMS/CSIRO King'' correct? DMT? & OK - JBJ\tabularnewline
\hline 
\pageref{punch:5-4} & Variables like AS200 have names with ``Raw Accumulation'' -- seems
awkward, consider name change? Maybe ``Count'' per channel? & \tabularnewline
\hline 
\pageref{punch:5-7} & add the variables for total counts? & \tabularnewline
\hline 
\pageref{punch:5-5} & some additions are needed here: RAFFD, PVOLU, TCNTD (total counts
all cells, CDP); housekeeping variables? All: first and last bins?
UHSAS: T and P in canister including UPRESS intensity, etc., better
in sect 7 & \tabularnewline
\hline 
\pageref{punch:5-6} & REFF2DC seems mis-named; all others based on 1D sizing from 2D have
names involving 2DC & \tabularnewline
\hline 
\pageref{punch:6-1} & need Teresa and/or Andy W to check this section & \tabularnewline
\hline 
\pageref{punch:6-3} & need to understand and document what te03c.c does, and perhaps move
to obsolete? & \tabularnewline
\hline 
\pageref{punch:6-2} & Is NO-related discussion OK? is this right: The one named NO2 is actually
for NOy? & \tabularnewline
\hline 
\pageref{punch:6-4} & In true measurement mode, XNOZA and XNZAF will be near zero -- is
this right? & \tabularnewline
\hline 
\pageref{punch:6-5} & I think the corrected-NO mixing ratio section may need revision? & \tabularnewline
\hline 
\pageref{punch:6-6} & ``has the provision for the addition of water vapor '' -- does
that mean this is always done, or only sometimes? & \tabularnewline
\hline 
\pageref{punch:7-1} & to ``0.1-360 s'' add ``but is typically set to 0.1~s''? & \tabularnewline
\hline 
\pageref{punch:7-2} & UPRESS: the attributes for this variable say the units are kPa; is
that incorrect? Mike R lists it as hPa. & resolved\tabularnewline
\hline 
\pageref{punch:10-1} & check signs here for WD and WS; this differs from the section-9 equation
... ?? & \tabularnewline
\hline 
\pageref{punch:10-2} & For EDPC, the <-50 branch looks suspicious and needs checking & \tabularnewline
\hline 
\pageref{punch:10-3} & For old cryogenic hygrometer, find and include the 3rd-order equation
referenced here & \tabularnewline
\hline 
\pageref{punch:10-4} & check Goff-Gratch formulas; there was some ambiguity in what was in
B9 & \tabularnewline
\hline 
\pageref{punch:10-5} & Should include basic equation for SWTC & \tabularnewline
\hline 
\pageref{punch:10-6} & re TECO CO: is the direct measurement (ppb) a mass ratio? Need explanation
here if so to describe difference between ppb and ppbv & \tabularnewline
\hline 
\pageref{punch:10-7} & COCAL: how does this differ from XCOMR? Why is this in the ``obsolete''
section? Same for O3FS? & \tabularnewline
\hline 
\pageref{punch:10-8} & For SCLWC, this is missing crucial information like how accreted mass
is obtained from voltage. Couldn't find the algorithm. Consider Mazin
version? Or old one for Wyo KA? & \tabularnewline
\hline 
 & There are some additional notes regarding obsolete variables, esp.
involving FSSP processing, that are not included here. & \tabularnewline
\hline 
WIC, GGALT, etc. & Review and correct descriptive attributes (e.g., WIC, GGALT, {*}DGPS & \tabularnewline
\hline 
 & StdSpeedofSound is wrong; enters ias.c -- obsolete now? & \tabularnewline
\hline 
 & fix Rd in xlate/const.c: calculated with wrong Md, although right
one is listed later in routine. (trivial difference) & \tabularnewline
\hline 
 & Lv defined in xlate/const.c is not latent heat but derivative of latent
heat vs T. Used correctly in thetap.c and plwcc.c, but deceptively
commented & \tabularnewline
\hline 
 & Review and approve new AKRD description & \tabularnewline
\hline 
\end{longtable}

\section*{How to Edit This Document}

Here are some notes regarding the construction and structure of this
document:
\begin{enumerate}
\item The reference version is ProcessingAlgorithms.lyx, which needs 'LyX',
a user interface to TeX. It is available on EOL machines like tikal.
Start it with ``lyx ProcessingAlgorithms.lyx''
\item The document is broken into many sections, referenced by the above
file, so they must be present also. Then have names like Section3.lyx
\item The document generates three indices: a regular index, a list of symbols,
and a list of variables. The references for these are embedded in
the .lyx files, and they can be modified or more can be added via
the ``Insert -> Index Entry'' controls. These practices are useful
when generating index entries:
\begin{enumerate}
\item entries like 'wind!relative' will generate index entries as subordinate
entries with 'relative' below 'wind'
\item I have tried to emphasize using nouns to start index entries, so for
example I would favor ``coefficient!calibration'' over ``calibration
coefficient.
\item It is sometimes useful to generate ``see xxx'' entries, which can
be done as follows: ``INS|see \{Inertial Navigation System\}'' where
the part in braces is also in LaTeX code, generated by pressing CNTL-L.
\end{enumerate}
\item Creating a PDF-format file in LyX usually will generate these lists
also. To be safe and ensure that the lists are updated, follow these
steps:
\begin{enumerate}
\item Export a LaTeX file from LyX in pdflatex format, using the ``export''
option.
\item Run ``pdflatex ProcessingAlgorithms'' three times to be sure all
references are resolved.
\item Run ``splitindex ProcessingAlgorithms'' to generate files for the
lists.
\item Run ``makeindex ProcessingAlgorithms'' to generate the lists.
\item Run ``pdflatex ProcessingAlgorithms'' again.
\end{enumerate}
\item The LyX files have embedded notes with additional information that
should be retained, and exporting to LaTeX will lose this information,
so it will be useful to retain the LyX format. The suggested next
steps in the table above, for example, almost all have associated
notes that will appear in yellow and will help identify where the
comment applies.
\item It is sometimes easiest to edit the PDF file directly. Some of the
web references have been changed in this way and can be adjusted as
the reference files are moved, e.g., from my Google Drive to the EOL
web pages. For this purpose, I found master-pdf-editor useful. This
will lose continuity, however, because then the links can't be re-generated
by running LyX.
\item As of Feb 2019, manylinks formerly to google-drive addresses or eol
system files have been changed to https://github.com/NCAR/aircraft\_ProcessingAlgorithms
links. In that directory there is a file ('links') with a list of
all the links in the document. It is worthwhile when updating this
document to check that all the links remain current. One way is to
use these R statements: \\
links <- readlines('./links'); EURL <- rep(FALSE, length(links));\\
for (i in 1:length(links)) \{EURL{[}i{]} <- RCurl::url.exists(links{[}i{]})\}\\
\#\# and then check EURL to see that the links are all found.
\end{enumerate}

\section*{How to Reference Specific Sections or Pages of this Document:}

\subsection*{Variables}

The document includes named destinations for each variable name, so
when used in a URL that destination can be reached. This is done differently
in different browsers or PDF viewers:
\begin{itemize}
\item For a web browser like Chrome or Firefox, use the ``nameddest''
reference; e.g., for the discussion of variable ATX, use\\
\texttt{firefox http://www.eol.ucar.edu/system/files/ProcessingAlgorithms.pdf\#nameddest=ATX}
\item For a pdf viewer like evince, use this syntax:\\
\texttt{evince -n ATX http://www.eol.ucar.edu/system/files/ProcessingAlgorithms.pdf}
\end{itemize}
Most variable names can be used in these URL modifiers. Here is a
list of available targets by section in the report:

\uline{Section 1:} Time

\uline{Section 2:} {[}none{]}

\uline{Section 3:} ACINS ALT BLATA BLONA BNORMA BPITCHR BROLLR
BYAWR DEI DNI FXAZIM FXDIST GGALT GGLAT GGLON GGNSAT GGOIDHT GGSPD
GGSTATUS GGTRK GGVEW GGVNS GGVSPD GGWUAL GMODE GSF HGM HGM232 HGME
HI3 LAT LATC LON LONC PALT PITCH ROLL THDG VEW VEWC VNS VNSC VSPD 

\uline{Section 4:} ADIFR AKRD AT\_ITR ATx ATX ATxD ATxJ BDIFR CAVP\_x
CONCH\_UVH CONCV\_VXL DP\_CR2C DP\_VXL DPx DP\_x DPxC DPXC DVALUE
EDPC EW\_UHV EWx EWX FP\_CR2 MACHx MACHX MIRRORT\_CR2 MIRRTMP\_DPX
MR MRCR MRLA MRLH MRVCL OAT PCAB PS\_A PSDPx PSFD PSFRD PSURF PSx
PSX PSxC PSXC QCx QCX QCxC QCXC RAWCONC\_VXL RHOx RHUM RHUMI RTHRx
RTx RTX RTxH SPHUM SSLIP TASHC TASx TASX TASxD THETA THETAE THETAP
THETAQ THETAV TVIR UI UIC UX UXC VI VIC VY VYC WD WDC WI WIC WS WSC
XSIGV\_UHV 

\uline{Section 5:} A1DC A1DP A200X A200Y A260X ACDP AF300 AFSSP
APCAS AS100 AUHSAS C1DC C1DP C200X C200Y C260X CCDP CF300 CFSSP COMCP
CONC1DC CONC1DC100 CONC1DC150 CONC1DP CONC3 CONC6 CONCD CONCF CONCU
CONCX CONCY CPCAS CS100 CUHSAS DBAR1DC DBAR1DP DBAR3 DBAR6 DBARD DBARF
DBARP DBARU DBARX DBARY DBZ DBZ1DC DBZ1DP DISP1DC DISP1DP DISP3 DISP6
DISPD DISPF DISPP DISPU DISPX DISPY DT1DC FRANGE FRNG PLWC1 PLWC1DC
PLWC1DP PLWC6 PLWCC PLWCC1 PLWCD PLWCF PLWCG PLWCX PLWCY REFF2DC REFF2DP
REFFD REFFF RICE 

\uline{Section 6:} CO2\_PIC COMR\_AL CORAW\_AL FO3\_ACD FO3\_CL
O3MR\_CL TEO3 TEO3C TEO3P TEP TET XFO3FNO XFO3FS XFO3P XNCLF XNMBT
XNO XNOCAL XNOCF XNOSF XNOY XNOYP XNOZA XNSAF XNST XNYCAL XNZAF XO3 

\uline{Section 7:} CNTEMP CNTS CONCN CONCP CONCU CONCU100 CONCU500
FCN FCNC PCN PFLW PFLWC TCHTP TCNTL TEMP1 TEMP2 UPRESS USFLWC USMPFLW
XICN XICNC 

\uline{Section 8:} IRxHT IRxV RSTx SPxPitch SPxRoll TRSTx VISxC
VISxHT VISxV 

\uline{Section 9:} {[}none{]}

\uline{Section 10:} OBSOLETE

\subsection*{Page Numbers}

To reference a specific page in the document, use a web reference
like this:\\
\texttt{ProcessingAlgorithms.pdf\#page=44}

In evince, this syntax will work, or the page number can be specified
in this way:

\texttt{evince -p 115 ProcessingAlgorithms.pdf}

\subsection*{Sections and Subsections}

Targets have not been provided for other parts of the document, but
the above method of referencing pages can be used to link to specific
sections and other components of the document.

\subsection*{Adding New Targets}

When a new variable is added, a new anchor point can be added by inserting,
in LaTeX mode, \texttt{\textbackslash nop\{LAT\} }at the appropriate
point in the LyX document. (\textbackslash nop has been defined to
use \textbackslash hypertarget but displace the reference upward
one line.) In addition, when a new variable is added, entries should
be made in the index items and the variable-names list, following
the pattern used for existing variables, and if appropriate any new
symbols used in discussing the algorithm should be added to the similar
symbols list.

\section*{}

\vfill\eject

\pagebreak
\end{document}
