
\section{THE STATE OF THE ATMOSPHERE\label{sec:State Variables}}

\subsection{Information on Instruments and Calibrations}

The instruments used to collect the measurements that lead to the
variables in this section are described on the EOL web site, in the
``State Parameters'' section at \href{http://www.eol.ucar.edu/aircraft-instrumentation}{this URL}.\index{instrument descriptions!web site}
The data acquisition and processing for these variables and the calibration
coefficients\index{coefficients!calibration}\index{calibration}
used where applicable are described \vpageref{DataAcquisitionDescription}.

\subsection{Variable Names\index{variable names}}

Measurements of some meteorological state variables like pressure,
temperature, and water vapor pressure may originate from multiple
sensors mounted at various locations on an aircraft. To distinguish
among similar measurements, many variable names\index{names!variable}
incorporate an indication of where the measurement was made. In this
document, locations in variable names\index{names!variable!location in}
are represented by ``x'', where ``x'' may be one of the following:

\begin{center}
\begin{tabular}{|c|c|}
\hline 
Character & Location\tabularnewline
\hline 
\hline 
B & bottom (or bottom-most)\tabularnewline
\hline 
B & (obsolete) boom\tabularnewline
\hline 
F & fuselage\tabularnewline
\hline 
G & (obsolete) gust probe\tabularnewline
\hline 
R & radome\tabularnewline
\hline 
T & top (or top-most)\tabularnewline
\hline 
W & wing\tabularnewline
\hline 
\end{tabular} 
\par\end{center}

In addition, a true letter 'X' (not replaced by the above letters)
may be appended to a measurement to indicate that it is the preferred
choice\index{variable!preferred choice} among similar measurements
and is therefore used to calculate derived variables that depend on
the measured quantity. Other suffixes\index{names!variable!suffixes}
sometimes used to distinguish among measurements are these: 'D' for
a digital sensor; 'H' for a heated (usually, anti-iced) sensor, 'L'
for port-side sensors, and 'R' for starboard-side sensors.

\subsection{\label{subsec:PTq}Pressure\index{pressure}}
\begin{hangparagraphs}
\textbf{Static Pressure (hPa): }\textbf{\uline{PSx}}\textbf{,\sindex[var]{PSx@\textbf{PSx}}\index{PSx}
}\textbf{\uline{PSxC}}\sindex[var]{PSxC}\index{PSxC}, \textbf{\uline{PS\_A}}\sindex[var]{PS_A@PS\_A}\index{PS_A@PS\_A}\textbf{,
}\textbf{\uline{PSXC}}\index{PSXC}\sindex[var]{PSXC}\textbf{.
}\textbf{\uline{PSFD}}\textbf{\sindex[var]{PSFD}}\index{PSFD}\textbf{,
}\textbf{\uline{PSFRD}}\textbf{\sindex[var]{PSFRD}}\index{PSFRD}\\
\emph{The atmospheric pressure at the flight level of the aircraft,
measured by a calibrated absolute (barometric) transducer at location
x.} PSx is the measured static or ambient pressure\index{pressure!ambient}\index{transducer!barometric}
before correction, and it may be affected by local flow-field distortion\index{flow distortion}.
PS\_A is the pressure measurement taken from the avionics\index{avionics}
system on the aircraft, processed via unknown algorithms in the avionics
system that may smooth, correct, and perhaps delay the result. PSxC
is PSx corrected for local flow-field distortion. (See \href{http://www.eol.ucar.edu/raf/Bulletins/bulletin21.html}{RAF Bulletin \#{}21}\index{Bulletin 21}
and the discussion in \href{https://drive.google.com/open?id=0B1kIUH45ca5AaW02ZUt1X2kyX2s}{this memo}),
and PSXC is the preferred\index{measurement!preferred} corrected
measurement used for derived calculations. These measurements have
been made using various sensors, so it is best to consult the project
documentation for the transducer used. Recent measurements from both
the C-130 and the GV have been made using a Paroscientific\index{transducer!Paroscientific}
Model 1000 Digiquartz Transducer. \\
\\
Corrections to the pressures\index{PCORS|see {pressure corrections}}\index{pressure!corrections}
have been determined by reference to some standard, including a ``trailing
cone''\index{trailing cone} sensor, the pressure PS\_A from the
cockpit avionics system, or (since 2012) the Laser Air Motion Sensing
System\index{LAMS} (LAMS). The latter correction is discussed in
the memo \href{https://drive.google.com/open?id=0B1kIUH45ca5AaW02ZUt1X2kyX2s}{referenced above},
where corrections used prior to 2011 are also discussed. Beginning
in 2012, the deduced corrections $\Delta p$ \index{correction!pressure}\sindex[lis]{Deltap@$\Delta p$=correction to pressure}to
the measured pressures\index{defect!static} as functions of dynamic
pressure\index{pressure!dynamic}\sindex[lis]{q@$q$= dynamic pressure}
$q,$ angle of attack $\alpha,$\footnote{A weakness is this form for the pressure correction is that occasionally
the radome ports become plugged with ice and the measurement of angle
of attack is not available. When the variable ATTACK representing
angle of attack is invalid, the angle of attack is instead calculated
from PITCH$-$VSPD/TASX, which approximates the angle of attack if
the vertical wind is zero.} and the Mach number $M$ are described by the following equations
and coefficients:\\
\\
\begin{minipage}[t]{1\columnwidth}%
\begin{quote}
\textbf{For the C-130,}\footnote{For C-130 measurements prior to 2012 but after September 2003, the
correction applied to PSF was $\Delta p=p+\max((3.29+\{\mathrm{QCX}\}*0.0273),$4.7915)
using units of hPa. Prior to Sept 2003, the correction was $\Delta p=\max((4.66+11.4405\Delta p_{\alpha}/\Delta q_{r}$),
1.113). For both PSFD and PSFRD, the correction prior to (2012?) was
$\Delta p=p+\max((3.29+\{\mathrm{QCX}\}*0.0273),$4.7915). For GV
measurements Aug 2006 to 2012, $\Delta p=$ (-1.02 + 0.1565{*}q) +
q1{*}(0.008 + q1{*}(7.1979e-09{*}q1 - 1.4072e-05). Before Aug 2006:
$\Delta p=$(3.08 - 0.0894{*}\{PSF\}) + \{QCF\}{*}(-0.007474 + \{QCF\}{*}4.0161e-06).}\\
\begin{equation}
\frac{\Delta p}{p}=d_{0}+d_{1}\frac{\alpha}{a_{r}}+d_{2}\,M\label{eq:PCORC130}
\end{equation}
where, for $p$ = PSFD \sindex[lis]{alpha@$\alpha$=angle of attack},
$\alpha=\mathrm{ATTACK}$ and $a_{r}=1^{\circ}$ (included to keep
the equation and coefficients\index{coefficients!sensitivity} dimensionless),
\{$d_{0},d_{1},d_{2}$\}=\{$-$0.00637, 0.001366, 0.0149\}. For PSFRD,
the coefficients are \{$d_{0}^{\prime},\,d_{1}^{\prime},\,d_{2}^{\prime}$\}\sindex[lis]{di@$d_{0-2}$=coefficients, pressure correction, C-130}
= \{$-$0.00754, 0.000497, 0.0368\}. The latter coefficients are
significantly different from the coefficients for PSFD, but the static
ports where PSFRD is measured are at a different location on the fuselage
so different flow-distortion effects are expected.
\end{quote}
%
\end{minipage}\\
\\
\begin{minipage}[t]{1\columnwidth}%
\begin{quote}
\textbf{For the GV,}\footnote{See \href{https://drive.google.com/a/ucar.edu/file/d/0B1kIUH45ca5AWlFWYXBDRlI1VnM}{this memo}
for details regarding implementation of this representation of $\Delta p$
for the GV: }\\
\begin{equation}
\frac{\Delta p}{p}=a_{0}+a_{1}\frac{q}{p}+a_{2}M^{3}+a_{3}\frac{\alpha}{a_{r}}\label{eq:PCORforGV}
\end{equation}
where, for $p$ = PSF, $q$ = QCF, $\alpha=\mathrm{ATTACK}$, and
$a_{r}=1^{\circ}$ (included to keep the equation and coefficients
dimensionless) \{$a_{0},a_{1},a_{2},a_{3}$\}\sindex[lis]{ai@$a_{0,1,2,3}$=coefficients, pressure correction, GV}
= \{$-0.012255$, $0.075372$, $-0.087508$, $0.002148$\}\}.\label{punch:4-10}
\end{quote}
%
\end{minipage}\\
\\
In equations (\ref{eq:PCORC130}) and (\ref{eq:PCORforGV}) the Mach
number\index{Mach number!uncorrected} is calculated from the uncorrected
measurements of $p$ and $q$, via\\
\begin{equation}
M=\left\{ \left(\frac{2c_{v}}{R}\right)\left[\left(\frac{p+q}{p}\right)^{R/c_{p}}-1\right]\right\} ^{1/2}\,\,\,.\label{eq:MachEquation}
\end{equation}
\\

\textbf{Dynamic Pressure (hPa): QCx, QCxC, QCXC}\index{QCXC}\index{QCx}\index{QCxC}\sindex[var]{QCx}\sindex[var]{QCxC}\sindex[var]{QCXC}\\
\emph{The pressure excess caused by bringing the airflow to rest relative
to the aircraft.} \index{pressure!dynamic}These quantities represent
the difference between the total pressure $p_{t}$\sindex[lis]{pt@$p_{t}$=total pressure}
as measured at the inlet of a pitot tube or other forward-pointing
port and the ambient pressure that would be present in the absence
of motion through the air.\emph{ }The variables ending in ``C''
have been corrected\index{pressure!dynamic!corrected} for flow-distortion
effects, mostly arising from errors in the measurement of static pressure.
Since 2012, the corrections are based on measurements from the LAMS\index{LAMS}
system as described for PSxC, and they have the same functional form
as in (\ref{eq:PCORC130}) and (\ref{eq:PCORforGV}) except that the
correction\index{correction!dynamic pressure} applied to $q$ is
$-\Delta p$ with reversed sign because $q=p_{t}-p_{a}$ and the error
arises primarily from the error in $p_{a}$. The same correction is
applied to QCR\index{QCR!correction} because it is also measured
relative to the static pressure ports so errors in the pressure sensed
at those ports affect QCR in the same way that QCF is affected. \label{punch4.3}See
the notes referenced in the preceding section, and also \href{http://www.eol.ucar.edu/raf/Bulletins/bulletin21.html}{RAF Bulletin 21}\index{Bulletin 21}
for the corrections applied to earlier data files.\footnote{\textbf{}%
\begin{minipage}[t]{1\columnwidth}%
\textbf{C-130}, prior to 2012: 
\begin{itemize}
\item For QCFC: subtract $\max((4.66+11.4405*\mathrm{{ADIFR\}/\{QCR}}$),
1.113
\item For QCFRC prior to Sept 2003: same as for QCFC
\item ~~~~~~~~~after/including Sept 2003, subtract $\max((3.29+\{\mathrm{QCX}\}*0.0273),$4.7915)
\item For QCRC: subtract $\max((3.29+\{\mathrm{QCX}\}*0.0273),$4.7915)
\end{itemize}
\textbf{GV, }Aug 2006 to 2012:
\begin{itemize}
\item For QCF, subtract (1.02+\{PSF\}{*}(0.215 - 0.04{*}\{QCF\}/1000.) +
\{QCF\}{*}(-0.003266 + \{QCF\}{*}1.613e-06))\label{punch4.2}
\end{itemize}
%
\end{minipage}}\index{pressure!dynamic}\index{pressure!total}\index{pressure!pitot|see {pressure, total}}
\index{pressure!ambient}\index{pressure!static|see {pressure ambient}}\index{Bulletin 21}
A Rosemount Model 1221 differential pressure transducer\index{pressure!transducer}
is used for current measurements of dynamic pressure on the C-130,
and a Honeywell PPT transducer is used on the GV.\label{punch4.1}
This measurement enters the calculation of true airspeed\index{true airspeed|see {airspeed}}\index{airspeed}
and Mach number\index{Mach number} and so is needed to calculate
many derived variables. 

\textbf{D-Value (m): DVALUE\index{DVALUE}}\sindex[var]{DVALUE}\\
The difference\index{d-value} between geopotential altitude\index{altitude!geopotential}
and pressure altitude\index{altitude!pressure} (m). This variable
is calculated from \{GGALT\}$-$\{PALT\} and, for appropriate flight
segments, can be used to measure height gradients on a constant-pressure
surface.

\textbf{Special Pressure Measurements (hPa): }\textbf{\uline{PSDPx}}\textbf{,
}\textbf{\uline{CAVP\_x}}\textbf{, }\textbf{\uline{PCAB}}\textbf{,
}\textbf{\uline{PSURF}}\index{PCAB}\index{PSDPx}\index{CAVP_x@CAVP\_x}\index{PSURF}\sindex[var]{PSDPx}\sindex[var]{CAVP_x@CAVP\_x}\sindex[var]{PCAB}\sindex[var]{PSURF}\index{pressure!cabin}\index{pressure!dew-point cavity}\index{pressure!surface}\\
\emph{PSDPx and CAVP\_x are measurements of the pressure in the housing
of the dew-point sensors, }as discussed in connection with DPxC.\emph{
PCAB is a measurement of the pressure in the cabin of the aircraft.
PSURF is the estimated surface pressure }calculated from HGME\index{HGME}
(a radar-altimeter measurement of height), TVIR\index{TVIR}, PSXC\index{PSXC}\label{punch:4-11},
and MR\index{MR} using the thickness equation\index{equation!thickness}.
TVIR and MR are described later in this section (cf.~pages \pageref{TVIR}
and \pageref{Mixing-Ratio-(g/kg):}, respectively), and HGME was described
on page \pageref{HGME} in Section \ref{sec:INS}. The average temperature
for the layer is obtained by using HGME and assuming a dry-adiabatic
lapse rate from the flight level to the surface. Because of this assumption,
the result is only valid for flight in a well-mixed surface layer
or in other conditions in which the temperature lapse rate matches
the dry-adiabatic lapse rate.\footnote{The symbol $^{\dagger}$ indicates that values are included in the
table of constants, p.~\pageref{ConstantsBox}.}\\
\fbox{\begin{minipage}[t]{0.95\columnwidth}%
PSXC\index{PSXC} = ambient pressure (hPa)\\
HGME\index{HGME} = (radar) altitude above the surface (m)\\
TVIR\index{TVIR} = virtual temperature ($^{\circ}\mathrm{C}$)\\
PSURF\index{PSURF} = estimated surface pressure (hPa)\\
$g$ = acceleration of gravity$^{\dagger}$\\
$R_{d}$ = gas constant for dry air$^{\dagger}$\\
$c_{pd}$ = specific heat of dry air at constant pressure$^{\dagger}$\\
\\
\rule[0.5ex]{1\columnwidth}{1pt}

\[
T_{m}=(\mathrm{\{TVIR\}}+T_{0})+0.5\mathrm{\{HGM\}}\frac{g}{c_{pd}}
\]
\begin{equation}
\mathrm{PSURF}=\mathrm{\{PSXC\}}\,\exp\left\{ \frac{g\,\{\mathrm{HGM}\}}{R_{d}T_{m}}\right\} \label{eq:PSURF-1-1}
\end{equation}
%
\end{minipage}}
\end{hangparagraphs}


\subsection{Temperature\index{temperature}}
\begin{hangparagraphs}
\textbf{\label{RTx discussion}Recovery Temperature ($\text{�}$C):
}\textbf{\uline{RTx}}\textbf{\sindex[var]{RTx}\index{RTx}, }\textbf{\uline{RTxH}}\sindex[var]{RTxH}\index{RTxH}\textbf{,
RTHRx\sindex[var]{RTHRx}}\index{RTHRx}\\
\emph{The recovery temperature\index{temperature!recovery} is the
temperature sensed by a temperature probe that is exposed to the atmosphere.}
In flight, the temperature is heated above the ambient temperature\index{temperature!ambient}
because it senses the temperature of air near the sensor that has
been heated adiabatically during compression as it is brought near
the airspeed of the aircraft. These variables are the measurements
of that recovery temperature from calibrated temperature sensors at
location x.\index{TTx}\sindex[var]{TTx}\footnote{Prior to 2012, these variables were called ``total temperature''
and symbols starting with 'TT' instead of 'RT' were used. That name
was misleading because these values are not true total-temperature\index{temperature!total}
measurements, for which the air would be at the same speed as the
aircraft, but instead recovery-temperature measurements. The name
has been changed to correct this mis-labeling, although this was a
long-standing convention in past datasets.} For Rosemount temperature probes, the recovery temperature\index{temperature!recovery vs.~total}
is near the total temperature, but all probes must be corrected to
obtain either true total temperature or true ambient temperature.
In the standard output, the variable name also conveys the sensor
type: RTx is a measurement from a Rosemount Model 102 non-deiced temperature
sensor,\index{sensor!temperature} RTxH is the measurement from a
Rosemount Model 102 anti-iced (heated) temperature\index{sensor!temperature!anti-iced}
sensor, and RTHRx is the measurement from a HARCO heated sensor.\index{sensor!temperature!heated}
Some past experiments also used a reverse-flow temperature\index{temperature!reverse-flow}
housing and a fast-response ``K'' housing;\index{sensor!temperature!K-probe}
the associated variable names for these probes were TTRF\index{TTRF}\sindex[var]{TTRF}
and TTKP.\index{TTKP}\sindex[var]{TTKP}\footnote{See the related obsolete variables\index{variables!obsolete!TTx}
TTx,\sindex[var]{TTx} which are previously used names for these variables.
The names were changed to clarify that the quantity represented is
the recovery temperature, not the total temperature.}

\textbf{Ambient Temperature ($\lyxmathsym{�}$C): }\textbf{\uline{ATx}}\textbf{\sindex[var]{ATx}\index{ATx},
}\textbf{\uline{ATxH}}\sindex[var]{ATxH}\index{ATxH}\textbf{,
}\textbf{\uline{ATxD}}\textbf{\sindex[var]{ATxD}}\index{ATxD}\\
\emph{The temperature of the atmosphere at the location of the aircraft,
as it would be measured by a sensor at rest relative to the air. }The
'x' in the name of the variable used for ambient temperature, ATx,
conveys the same information regarding sensor type and location as
the variable name used with total (recovery) temperature. See the
discussion above regarding RTx. The ambient temperature\index{temperature!ambient}
(also known as the static air temperature\index{temperature!static air})
is calculated from the measured recovery temperature, which is increased
above the ambient temperature by dynamic heating caused by the airspeed
of the aircraft. The calculated temperature\index{temperature!calculation}
therefore depends on the recovery temperature RTx\index{RTx} as well
as the dynamic\index{pressure!dynamic} and ambient pressure\index{pressure!ambient},
usually respectively QCXC\index{QCXC} and PSXC\index{PSXC}. The
ambient and dynamic pressures are first corrected from the raw measurements\index{measurement!raw}
QCX\index{QCX} and PSX\index{PSX} to obtain variables that account
for deviations caused by airflow around the aircraft and/or position-dependent
systematic errors, as discussed in the section describing PSxC. The
following basic equations are developed on the basis of conservation
of energy for a perfect gas\index{perfect gas} undergoing an adiabatic
compression.\index{compression!adiabatic}\\
\\
This section combines discussion of the calculations of temperature
and airspeed\index{airspeed}, to reflect the linkage\index{linkage. temperature and airspeed}
between these derived measurements. To provide accuracy in the equations,
this discussion considers effects of the humidity\index{correction!moist air}
of the air on characteristics like the gas constant\index{gas constant!moist air}
and the specific heats.\index{specific heat!moist air} Most archived
data before 2012 used values for dry air, although a special variable
TASHC\index{TASHC} has been used to represent the true airspeed in
cases where the correction was significant. That variable is based
on a good approximation to the results from the following equations;
see the discussion of TASHC later in this section. TASHC is now considered
an obsolete variable. New variables ATxD and TASxD have been introduced
that neglect the humidity corrections and perform all calculations
as if the humidity is negligible.\\
\\
\label{ambient temperature and TAS calculation}As discussed above,
temperature\index{temperature!sensor} sensors on aircraft that are
exposed to the airflow do not measure the total temperature\index{temperature!total}
but rather the temperature of the air immediately in contact with
the sensing element. This air will not have undergone an adiabatic
deceleration completely to zero velocity and hence will have a temperature
$T_{r}$ somewhat less than the total temperature $T_{t}$ that would
require the air to reach zero velocity. $T_{r}$ is the measured or
``recovery'' temperature\index{temperature!recovery}.\sindex[lis]{Ta@$T_{a}$= ambient air temperature in absolute units; sometimes,
$T_{K}$}\sindex[lis]{Tr@$T_{r}$= recovery temperature}\sindex[lis]{Tt@$T_{t}$= total air temperature},
The ratio of the actual temperature difference attained to the temperature
difference relative to the total temperature is defined to be the
``recovery factor''\index{recovery factor} $\alpha:$\sindex[lis]{ar@$\alpha_{r}$= recovery factor, temperature probe}\\
\begin{equation}
\alpha_{r}=\frac{T_{r}-T_{a}}{T_{t}-T_{a}}\label{eq:8.2-1}
\end{equation}
where $T_{a}$ is the ambient air temperature. From conservation of
energy\index{conservation of energy}:\\
\begin{equation}
\frac{U_{a}^{2}}{2}+c_{p}^{\prime}T_{a}=\frac{U_{r}^{2}}{2}+c_{p}^{\prime}T_{r}=\frac{U_{t}^{2}}{2}+c_{p}^{\prime}T_{t}\label{eq:8.1-1}
\end{equation}
\\
where primes on quantities like $c_{p}^{\prime}$, or (below) $c_{v}^{\prime}$
and $R^{\prime}$ denote properties\index{properties of moist air}
of moist air, respectively the specific heat\index{specific heat}
at constant pressure, specific heat at constant volume, and gas constant.\index{gas constant}
\\
\\
\fbox{\begin{minipage}[t]{1\columnwidth - 2\fboxsep - 2\fboxrule}%
\textbf{Moist-air considerations:}\\
\\
Primes on the symbols denote that these values should be moist-air
values, appropriately weighted averages of the dry-air and water-vapor
contributions. The practice prior to 2014 was to use the dry-air values
for specific heats and the gas constant, except as described in connection
with TASHC below. Since 2014, calculations use the appropriate values
for moist air, except that to avoid errors introduced by unrealistically
high measurements of humidity the humidity correction was limited
to be less than or equal to the equilibrium value at the measured
temperature. The formulas used for the specific heats and gas constant
of moist air in terms of the water vapor pressure $e$, the specific
heats for dry air ($c_{pd}=\frac{7}{2}R_{0},\,c_{vd}=\frac{5}{2}R_{0}$)
and water vapor ($c_{pw}=4R_{0},\,c_{vw}=3R_{0}$), and the ratio
of molecular weights ($\epsilon=M_{W}/M_{d}$) are those of Khelif
et al.~1999:\sindex[lis]{epsilon@$\epsilon=M_{W}/M_{d}$}\index{gas constant!moist air}\index{specific heat!moist air}\sindex[lis]{R@$R^{\prime}=$gas constant for moist air}\sindex[lis]{cpprime@$c_{p}^{\prime}$= specific heat at constant pressure for
moist air}\sindex[lis]{cvprime@$c_{v}^{\prime}=$specific heat at constant volume for moist
air}
\begin{equation}
R^{\prime}=R_{d}/[1+(\epsilon-1)\frac{e}{p}]\label{eq:moistR}
\end{equation}
\begin{equation}
c_{v}^{\prime}=\frac{(p-e)R^{\prime}}{pR_{d}}\frac{5R_{0}}{2M_{d}}+\frac{eR^{\prime}}{pR_{w}}\frac{3R_{0}}{M_{w}}=c_{vd}\frac{R^{\prime}}{R_{d}}\left(1+\frac{1}{5}\frac{e}{p}\right)\label{eq:moistcv}
\end{equation}

\begin{equation}
c_{p}^{\prime}=c_{pd}\frac{R^{\prime}}{R_{d}}\left(1+\frac{1}{7}\frac{e}{p}\right)\label{eq:moistcp}
\end{equation}

\begin{equation}
\gamma\,^{\prime}=\gamma_{d}\frac{1+\frac{1}{7}\frac{e}{p}}{1+\frac{1}{5}\frac{e}{p}}\label{eq:moistgamma}
\end{equation}
See also the discussion of TASHC\index{TASHC} in section \ref{sec:WIND}
and the reference there for Khelif et al.~1999.%
\end{minipage}}\\
\\
\\
\label{DiscussionOfMoistAirVariables}\index{moist-air properties}In
(\ref{eq:8.1-1}), $\{U_{a},\,U_{r},\,U_{t}\}$ \sindex[lis]{Ua@$U_{a}$= true airspeed (sometimes $U$)}are
respectively the aircraft true airspeed\index{airspeed}, the airspeed
relative to the aircraft of the air in thermal contact with the sensor,
and the airspeed of air relative to the aircraft when fully brought
to the motion of the sensor (i.e., zero). Then, from (\ref{eq:8.1-1})\\
\begin{equation}
T_{a}=T_{r}-\alpha_{r}\frac{U_{a}^{2}}{2c_{p}^{\prime}}\label{eq:8.3-1}
\end{equation}
The temperature sensors used on RAF aircraft are designed to decelerate
the air adiabatically to near zero velocity. Recovery factors\index{recovery factor!Rosemount sensors}
determined from wind tunnel testing for the Rosemount sensors are
approximately 0.97 (unheated model) and 0.98 (heated models).\footnote{The recovery factor determined for the now-obsolete NCAR reverse-flow
sensor was 0.6. The recovery factor for the now retired NCAR fast-response
(K-probe) temperature sensor was 0.8. } These values\index{recovery factor} have also been confirmed from
flight maneuvers, often from ``speed runs'' where the aircraft is
flown level through its speed range and the variation of recovery
temperature with airspeed is used with (\ref{eq:8.3-1}), with the
assumption that $T_{a}$ remains constant, to determine the recovery
factor. Data files and project reports normally document what recovery
factor was used for calculating the true airspeed and ambient temperature
for a particular project.\\
\\
Because the values used in processing have varied, the project reports
should be consulted to find what was used for particular projects.
The Goodrich Technical Report 5755\label{punch:4-4} documents wind-tunnel
testing of the probes formerly made by Rosemount. Their plot showed
that, for heated sensors, there is a significant variation with Mach
number\index{Mach number} ($M$); cf Eq.~\pageref{eq:MachEquation}).
The dependence in their plot is represented well by the following
equations, where $\alpha_{r}^{[h]}$ refers to heated probes and $\alpha_{r}^{[u]}$
to unheated probes:
\begin{equation}
\alpha_{r}^{[h]}=0.988+0.053(\log_{10}M)+0.090(\log_{10}M)^{2}+0.091(\log_{10}M)^{3}\label{eq:RecFactorMachDep}
\end{equation}
\begin{equation}
\alpha_{r}^{[u]}=0.9959+0.0283(\log_{10}M)+0.0374(\log_{10}M)^{2}+0.0762(\log_{10}M)^{3}\label{eq:RecFactorUnheated}
\end{equation}
Some studies of the recovery factor are discussed further in \href{https://drive.google.com/open?id=0B1kIUH45ca5AOWlIbGxVcC13SHM}{this memo}.\\
\\
The true airspeed $U_{a}$ is used in (\ref{eq:8.3-1}) to calculate
the ambient temperature $T_{a}$. However, the ambient temperature
is also needed to calculate the true airspeed. Therefore the constraints
imposed on ambient temperature and true airspeed by the measurements
of recovery temperature, total pressure\index{pressure!total} (the
pressure measured by a pitot tube pointed into the airstream and assumed
to be that obtained when the incoming air is brought to rest relative
to the aircraft), and ambient pressure must be used to solve simultaneously
for the two unknowns, temperature and airspeed. \\
\\
The relationship is conveniently derived by first calculating the
dimensionless Mach number\index{Mach number}\sindex[lis]{M@$M$= Mach number, ratio of airspeed to the speed of sound}
($M$), which is the ratio of the airspeed to the speed of sound\index{speed of sound}
($U_{s}=\sqrt{\gamma^{\prime}R^{\prime}T_{a}}$ \sindex[lis]{Us@$U_{s}$= speed of sound}where
$\gamma^{\prime}$ is the ratio\sindex[lis]{gamma@$\gamma^{\prime}=$$c_{p}^{\prime}/c_{v}^{\prime}$}
of specific heats of (moist) air, $c_{p}^{\prime}/c_{v}^{\prime}$).
The Mach number is a function of air temperature only and can be determined
as follows: \\
a). Express energy conservation, as in (\ref{eq:8.1-1}), in the form\\
\begin{equation}
d\left(\frac{U^{2}}{2}\right)+c_{p}^{\prime}dT=0\,\,\,\,.\label{eq:8.1a-1}
\end{equation}
where the total derivatives apply along a streamline as $U$ changes
from $U_{a}$ to $U_{t}=0$ and $T$ changes from $T_{a}$ to $T_{t}$.\\
b). Use the perfect gas law to replace $dT$ with $\frac{pV}{nR}(\frac{dV}{V}+\frac{dp}{p})$
where $V$\sindex[lis]{V@$V$= volume} and $p$\sindex[lis]{p@$p$= pressure}
are the volume and pressure of a parcel of air. Then use the expression
for adiabatic compression\index{adiabatic compression} in the form
$pV^{\gamma}=constant$ to replace the derivative $\frac{dV}{V}$
with $-\frac{1}{\gamma}\frac{dp}{p}$, leading to $dT=\frac{R^{\prime}T}{c_{p}^{\prime}}\frac{dp}{p}$
or, after integration, $T(p)=T_{a}\left(\frac{p}{p_{a}}\right)^{R^{\prime}/c_{p}^{\prime}}.$
Using this expression for $T$ in the formula for $dT$ and then integrating
both total derivatives in (\ref{eq:8.1a-1}) along the streamline
leads to \\
\begin{equation}
\frac{U_{a}^{2}}{2}+c_{p}^{\prime}T_{a}=c_{p}^{\prime}T_{a}\left(\frac{p_{t}}{p_{a}}\right)^{\frac{R^{\prime}}{c_{p}^{\prime}}}\label{eq:8.4-1}
\end{equation}
where $p_{t}$ is the total pressure\index{pressure!total} (i.e.,
PSXC\index{PSXC}+QCXC\index{QCXC}) and $p_{a}$ the ambient pressure\index{pressure!ambient}\sindex[lis]{pa@$p_{a}$= ambient air pressure}
(PSXC\index{PSXC}).\\
\\
c). Use the above definition of the Mach number\index{Mach number}
$M$ ($M=U_{a}/U_{s}$) in the form $U_{a}^{2}=\gamma^{\prime}M^{2}R^{\prime}T_{a}$
to obtain:\\
\begin{equation}
M^{2}=\left(\frac{2c_{v}^{\prime}}{R^{\prime}}\right)\left[\left(\frac{p_{t}}{p_{a}}\right)^{\frac{R^{\prime}}{c_{p}^{\prime}}}-1\right]\label{eq:8.5-1}
\end{equation}
which is the same as (\ref{eq:MachEquation}). This equation shows
that $M$ can be found from $p_{t}$ and $p_{a}$ alone, except for
the moist-air corrections. \sindex[lis]{ptotal@$p_{t}=$total pressure (ambient + dynamic)}\\
\\
d). Use the expression for ambient temperature in terms of recovery
temperature and airspeed, (\ref{eq:8.3-1}), to obtain the temperature
in terms of the Mach number and the recovery temperature:\sindex[lis]{gamma@$\gamma^{\prime}=$$c_{p}^{\prime}/c_{v}^{\prime}$}\\
\begin{eqnarray}
T_{a} & = & T_{r}-\alpha_{r}\frac{U_{a}^{2}}{2c_{p}^{\prime}}=T_{r}-\alpha_{r}\frac{M^{2}\gamma^{\prime}R^{\prime}T_{a}}{2c_{p}^{\prime}}\nonumber \\
 & = & \frac{T_{r}}{1+\dfrac{\alpha_{r}M^{2}R^{\prime}}{2c_{v}^{\prime}}}\label{eq:8.6-1}
\end{eqnarray}
\\
e). Express the true airspeed\index{airspeed} ($U_{a}$) as\\
\begin{equation}
U_{a}=M\sqrt{\gamma\,^{\prime}R^{\prime}T_{a}}\label{eq:8.7-1}
\end{equation}
\\
\label{ATX discussion}Then the temperature is found as described
in the following box:\footnote{A problem sometimes arises from use of the measured humidity, because
that measurement might be obviously in error. For example, following
descents the dew point determined from chilled-mirror hygrometers
sometimes overshoots the correct value significantly, producing dew-point
measurements well above the measured temperature. If such measurements
are used, the result can produce a significant error in derived variables
based on the humidity-corrected gas constant and specific heats. If
the measurements are flagged as bad, there will be gaps in derived
variables. To avoid these two errors, the corrections applied to the
gas constant and specific heats are treated as follows:
\begin{itemize}
\item The humidity correction is limited to not more than that given by
the water-equilibrium humidity at the temperature ATXD, calculated
using dry-air specific heats and gas constant. 
\item If the humidity from the primary sensor is flagged as a missing measurement
(e.g., from a dew-point sensor), a secondary measurement is used (e.g.,
the VCSEL)\index{hygrometer!VCSEL} in cases when the secondary sensor
is almost always present in an experiment.
\item As a backup, the variables TASxD and ATxD are always calculated omitting
the humidity correction to the gas constant and the specific heats.
These variables usually provide continuous measurements, although
they will be offset from the humidity-corrected values. The offset
indicates the magnitude of the correction when both are present, and
one of the variables TASxD (ATxD) may be selected as TASX (ATX) in
cases where missing values might cause a problem for derived variables. 
\end{itemize}
}\\
\\
\fbox{\begin{minipage}[t]{0.9\textwidth}%
RTX\index{RTX} = recovery temperature ($T_{r})$\\
QCxC\index{QCxC} = dynamic pressure, corrected ($q_{a}$)\\
PSXC\index{PSXC} = ambient pressure, after airflow/location correction
($p_{a}$)\\
MACHx\index{MACHx}\sindex[var]{MACH = Mach number} = Mach number
based on QCxC and PSXC; cf.~(\ref{eq:8.5-1})\\
MACHX = best Mach number, based on QCXC and PSXC\\
$\alpha_{r}$ = recovery factor for the particular temperature sensor\\
$R^{\prime}$, $c_{v}^{\prime}$ and $c_{p}^{\prime}$ as defined
above and in the list of symbols\\
\\
\rule[0.5ex]{1\columnwidth}{1pt}

From (\ref{eq:8.5-1}),

\begin{equation}
\mathrm{MACHx}=\left\{ \left(\frac{2c_{v}^{\prime}}{R^{\prime}}\right)\left[\left(\frac{\mathrm{\{PSXC\}+\{QCxC\}}}{\mathrm{\{PSXC\}}}\right)^{\frac{R^{\prime}}{c_{p}^{\prime}}}-1\right]\right\} ^{1/2}\label{eq:8.8-1}
\end{equation}
\\
From (\ref{eq:8.6-1})

\begin{equation}
\mathrm{ATx}=\frac{\mathrm{\left(\{RTx\}+T_{0}\right)}}{\left(1+\dfrac{\alpha_{r}\mathrm{(\{MACHX\})}^{2}R^{\prime}}{2c_{v}^{\prime}}\right)}-T_{0}\label{eq:8.9-1}
\end{equation}

%
\end{minipage}}

\textbf{In-cloud Air Temperature, Radiometric ($^{\circ}$C):} \textbf{\uline{AT\_ITR}}\index{AT_ITR@AT\_ITR}\sindex[var]{AT_ITR@AT\_ITR}
\\
\emph{The radiometric ambient air temperature measured by the In-cloud
Air Temperature Radiometer,} which measures the radiometric temperature\index{temperature!radiometric}\index{temperature!in-cloud}
in the 4.3 $\mu$m CO$_{2}$ band.\label{AT_ITR} Its primary use
is in water cloud when the standard thermometers are affected by wetting.\index{wetting!of thermometers}
In clear air the temperature is an average over an integrating range
of up to 100s of meters away from the aircraft, whereas in clouds
the integrating range is as little as 10 meters because of water droplets.
The calibration is by a polynomial fit of the internal reference temperature
and measured radiance to the ATX temperature.\label{punch:4-5}

\textbf{Ophir Air Temperature ($^{\circ}$C): }\textbf{\uline{OAT}}\index{OAT}\sindex[var]{OAT}\\
\emph{The radiometric temperature reported by the Ophir III radiometer,}
which operates on the same principles as the ITR,\label{OAT} with
the same limitations. For more information on this instrument, see
this \href{http://opensky.library.ucar.edu/collections/TECH-NOTE-000-000-000-822}{Technical Note}.
The in-cloud air temperature radiometer is a later, improved version,
but the Ophir III radiometer remains in use.
\end{hangparagraphs}


\subsection{Humidity\index{humidity}}
\begin{hangparagraphs}
\textbf{Dew/Frost Point ($\text{�}$C): }\textbf{\uline{DPx}}\sindex[var]{DPx}\index{DPx}\textbf{,
}\textbf{\uline{DP\_x}}\index{DP_x@DP\_x}\sindex[var]{DP_x@DP\_x}\textbf{,
}\textbf{\uline{MIRRTMP\_DPx}}\textbf{\sindex[var]{DPx}}\index{MIRRTMP_DPx@MIRRTMP\_DPx}\\
\emph{The mirror temperature measured directly by a dew-point sensor,
without correction. }The dew point\index{dew point} or frost point\index{frost point}
is measured by either an EG\&G Model 137, a General Eastern Model
1011B or a Buck Model 1011C dew-point hygrometer\index{hygrometer!dew point}.
Below 0\textbf{$^{\circ}$C} the instrument is assumed to be responding
to the frost point, although occasionally in climbs there is a short
transition near the freezing level before the condensate on the mirror
of the instrument freezes and there may be a measurement error before
the condensate freezes. The measurements are usually made within a
housing where the pressure ($p_{h})$ may differ from the ambient
pressure, so the pressure in the housing affects the measured dew
point or frost point. The housing pressure is often adjusted to be
near the ambient pressure by appropriate orientation of inlets, and
recently the pressure in the housing is measured and a correction
is applied, as discussed in the next paragraph. 

\textbf{Corrected Dew Point (C): DPxC}\sindex[var]{DPxC}\index{DPxC}\textbf{}\footnote{See also DP\_VXL and DP\_CR2C below}\textbf{}\\
\index{dew point!corrected}\emph{\index{frost point}The dew point
obtained from the original measurement after correction for the housing
pressure, the enhancement of the equilibrium vapor pressure arising
from the total pressure (discussed below), and conversion from frost
point if appropriate,} The result is the temperature at which the
equilibrium vapor pressure\index{pressure!water vapor!equilibrium}
over a plane water surface in the absence of other gases would match
the actual water-vapor pressure. Dew/frost-point hygrometers\index{hygrometer}
measure the equilibrium point in the presence of air\index{enhancement factor},
and the presence of air affects the measurement in a minor way that
is represented by a small correction here named the ``enhancement
factor.'' In the case where the dew-point or frost-point sensor is
exposed to ambient air directly, the enhancement factor is defined
so that the ambient vapor pressure\sindex[lis]{ea@$e_{a}$= ambient water vapor pressure}
$e_{a}$ is related to $T_{p}$, the \emph{measured }dew or frost
point \emph{in the presence of air} having total pressure $p$, by
$e_{a}=f(p,T_{P})\,e_{s}(T_{p})$ \sindex[lis]{fpT@$f(p,T_{p})$=water vapor pressure enhancement factor}where
$e_{s}(T_{p})$ is the vapor pressure in equilibrium with ice or water
at the dew or frost point $T_{p}$ \emph{in the absence of air.} Calculation
of DPxC removes this dependence, so the vapor pressure obtained from
$e_{s}(\{\mathrm{DPxC\}})$ will be that vapor pressure corresponding
to equilibrium \emph{in the absence of air}. In addition, if the measurement
is below 0$^{\circ}$C, it is assumed to be a measurement of frost
point and a corresponding dew point is calculated from the measurement
(also with correction for the influence of the total pressure on the
measurement). Some changes were made to these calculations in 2011;
for more information, see \href{https://drive.google.com/open?id=0B1kIUH45ca5Ab1NTRVo0bjdac0U}{this memo}.\\
\\
An additional correction is needed in those cases where the pressure\index{pressure!dew point housing}
in the housing of the instrument (measured as PSDPx or CAVP\_x) differs
from the ambient pressure, because the changed pressure affects the
partial pressure\index{pressure!partial, water vapor} of water vapor
in proportion to the change in total pressure and so changes the measured
dew point from the desired quantity (that in the ambient air) to that
in the housing\index{hygrometer!housing}. This is especially important
in the case of the GV because the potential effect increases with
airspeed. If the pressure in the housing is measured or otherwise
known (e.g., from correlations with other measurements), then this
correction can be introduced into the processing algorithm at the
same time that the correction for the presence of dry air is introduced,
and the enhancement factor should be evaluated at the pressure in
the housing. \\
\\
The relationship between water-vapor pressure and dew- or frost-point
temperature is based on the Murphy and Koop\footnote{Q. J. R. Meteorol. Soc. (2005), 131, pp. 1539\textendash 1565}\index{Murphy and Koop, 2005}\index{pressure!water vapor!equilibrium}
(2005) equations.\footnote{Prior to 2010, the vapor pressure relationship used was the Goff-Gratch
formula as given in the Smithsonian Tables (List, 1980).} They express the equilibrium vapor pressure as a function of frost
point or dew point \emph{and at a total air pressure $p$} via equations
that are equivalent to the following:\\
\begin{eqnarray}
e_{s,i}(T_{FP})= & b_{0}^{\prime}\exp(b_{1}\frac{(T_{0}-T_{FP})}{T_{0}T_{FP}}+b_{2}\ln(\frac{T_{FP}}{T_{0}})+b_{3}(T_{FP}-T_{0}))\label{eq:MK1-1}
\end{eqnarray}
\begin{equation}
e_{s,w}(T_{DP})=c_{0}\exp\left((\alpha-1)c_{6}+d_{2}(\frac{T_{0}-T_{DP}}{T_{DP}T_{0}})\right)+d_{3}\ln(\frac{T_{DP}}{T_{0}})+d_{4}(T_{DP}-T_{0})\label{eq:proposedNewWater}
\end{equation}
\begin{equation}
f(p,T_{P})=1+p(f_{1}+f_{2}T_{p}+f_{3}T_{P}^{2})\label{eq:EnhancementFactor}
\end{equation}
where $e$\sindex[lis]{e@$e$= water vapor pressure} is the water
vapor pressure, $T_{FP}$\sindex[lis]{Tdp@$T_{DP}$= temperature at the dew point}\sindex[lis]{Tfp@$T_{FP}$= temperature at the frost point}
or $T_{DP}$ is the frost or dew point, respectively, expressed in
kelvin, $T_{0}$=273.15\,K, $e_{s,i}(T_{FP})$\sindex[lis]{esi@$e_{s,i}$= equilibrium vapor pressure over a plane ice surface}
is the equilibrium vapor pressure over a plane ice surface at the
temperature $T_{FP}$, $e_{s,w}(T_{DP})$\sindex[lis]{esl@$e_{s,l}$= equilibrium vapor pressure over a plane water surface}
is the equilibrium vapor pressure over a plane water surface at the
temperature $T_{DP}$ (above or below $T_{0}$), and $f(p,T_{P})$\sindex[lis]{fpT@$f(p,T_{p})$=water vapor pressure enhancement factor}is
the enhancement factor at total air pressure $p$ and temperature
$T_{p}$\sindex[lis]{Tp@$T_{p}=$dew point temperature if above 0$^{\circ}C$, frost point
temperature otherwise}, with $T_{P}$ equal to $T_{DP}-T_{0}$ when above $T_{0}$ and $T_{FP}-T_{0}$
when below 0$\text{�}$C . \\
\\
The coefficients used in the above formulas are given in the following
tables, with the additional definitions that \sindex[lis]{alphaT@$\alpha_{T}$= tanh($e_{s}(T-T_{x})$, Murphy/Koop equations}$\alpha_{T}=\tanh(c_{5}(T-T_{x}))$,
\sindex[lis]{Tx@$T_{x}$= 218.8~K, Murphy/Koop equations}$T_{X}$
= 218.8~K, and $d_{i}=c_{i}+\alpha_{T}c_{i+5}$ for i = \{2,3,4\}:\sindex[lis]{c19@$c_{0-9}$=coefficients, vapor pressure equation}\sindex[lis]{b03@$b_{0-3}$=coefficients, vapor pressure equation}\sindex[lis]{f13@$f_{1-3}$=coefficients, vapor pressure equation}\\
\fbox{\begin{minipage}[t]{0.95\textwidth}%
\textbf{~~~~~}%
\begin{tabular}{|c|c|}
\hline 
\textbf{Coefficient} & \textbf{Value}\tabularnewline
\hline 
\hline 
$b_{0}^{\prime}$ & 6.11536\,hPa\tabularnewline
\hline 
$b_{1}$ & $-5723.265\,K,$\tabularnewline
\hline 
$b_{2}$ & 3.53068\tabularnewline
\hline 
$b_{3}$ & -0.00728332\,K$^{-1}$\tabularnewline
\hline 
$f_{1}$ & 4.923$\times10^{-5}$ hPa$^{-1}$\tabularnewline
\hline 
$f_{2}$ & -3.25$\times10^{-7}$hPa$^{-1}$K$^{-1}$\tabularnewline
\hline 
$f_{3}$ & 5.84$\times10^{-10}$hPa$^{-1}$K$^{-2}$\tabularnewline
\hline 
\end{tabular}\textbf{~~~~~}%
\begin{tabular}{|c|c|}
\hline 
\textbf{coefficient} & \textbf{value}\tabularnewline
\hline 
\hline 
$c_{0}$ & 6.091886 hPa\tabularnewline
\hline 
$c_{1}$ & 6.564725\tabularnewline
\hline 
$c_{2}$ & -6763.22\,K\tabularnewline
\hline 
$c_{3}$ & -4.210\tabularnewline
\hline 
$c_{4}$ & 0.000367\,K$^{-1}$\tabularnewline
\hline 
$c_{5}$ & 0.0415\,K$^{-1}$\tabularnewline
\hline 
$c_{6}$ & -0.1525967\tabularnewline
\hline 
$c_{7}$ & -1331.22\,K\tabularnewline
\hline 
$c_{8}$ & -9.44523\tabularnewline
\hline 
$c_{9}$ & 0.014025\,K$^{-1}$\tabularnewline
\hline 
\end{tabular}%
\end{minipage}}\\
\\
The vapor pressure in the instrument housing, \sindex[lis]{eh@$e_{h}$= water vapor pressure in an instrument housing}$e_{h}$,
is related to the sensed dew or frost point according to equation
(\ref{eq:MK1-1}) or (\ref{eq:proposedNewWater}), but further corrections
must also be made for the enhancement factor and to account for possible
difference between the pressure in the sensor housing\sindex[lis]{ph@$p_{h}$= pressure in a sensor housing}
$p_{h}$ and the ambient pressure $p_{a}$: \\
\begin{equation}
e_{a}=f(p_{a},T_{p})e_{h}\frac{p_{a}}{p_{h}}\label{eq:HousingPressureCorrection}
\end{equation}
 \\
\label{punch:4.6}Because processing to obtain the corrected dew
point DPxC\index{DPxC} from the ambient vapor pressure\index{pressure!water vapor}
$e_{a}$ would require difficult inversion of the above formulas,
interpolation is used instead. A table constructed from (\ref{eq:MK1-1})
and another constructed from (\ref{eq:proposedNewWater}), giving
water vapor pressure as a function of frost point or dew point temperature
in 1$^{\circ}$C increments from -100 to +50$\text{�}$C, is then
used with three-point Lagrange interpolation (via a function described
below as $F_{D}(e)$)\sindex[lis]{Fd@$F_{d}$= interpolation formula for dew point}
to find the dew point temperature from the vapor pressure.\footnote{prior to 2011 the conversion was made using the formula $\mathrm{DPxC=0.009109+DPx(1.134055+0.001038DPx)}$.
For instruments producing measurements of vapor density (RHO), the
previous Bulletin 9 section incorrectly gave the conversion formula
as $DPxC=273.0Z/(22.51-Z)$, a conversion that would apply to frost
point, not dew point. However, the code in use shows that the conversion
was instead $237.3Z/(17.27-Z)$, where Z in both cases is $Z=\ln((\mathrm{ATX}+273.15)\mathrm{RHO/1322.3)}$. }
\end{hangparagraphs}

~~
\begin{hangparagraphs}
~~~\\
Tests of these interpolation formulas against high-accuracy numerical
inversion of formulas (\ref{eq:MK1-1}) and (\ref{eq:proposedNewWater})
showed that the maximum error introduced by the interpolation formula
was about 0.004$^{\circ}$C and the standard error about 0.001$^{\circ}$C.
This inversion then provides a corrected dew point\index{dew point!corrected}
that incorporates the effects of the enhancement factor as well as
differences between the ambient pressure and that in the housing.
The algorithm is documented in the box below.\\
\\
For other instruments that measure vapor density, such as a Lyman-alpha\index{Lyman-alpha hygrometer}
or tunable diode laser hygrometers\index{hygrometer!tunable diode laser}
(including the Vertical Cavity Surface Emitting Laser (VCSEL) hygrometer),\index{hygrometer!VCSEL}
a similar conversion is made from vapor density to dew point, as documented
below:\\
\\
\\
\\
\doublebox{\begin{minipage}[t]{1\columnwidth - 2\fboxsep - 7.5\fboxrule - 1pt}%
\begin{center}
{[}See next page{]}
\par\end{center}%
\end{minipage}}\\
\fbox{\begin{minipage}[t]{0.95\columnwidth}%
$T_{p}$ = DPx\index{DPx} = mirror-temperature measurement\sindex[lis]{Tp@$T_{p}$=mirror temperature}
from instrument x {[}$^{\circ}$C{]}, or alternately\\
RHO = water vapor density\index{water vapor!density} measurement
{[}$\mathrm{g\,}\mathrm{m}^{-3}${]}; only one is used in any calculation\\
ATX\index{ATX} = reference ambient temperature\index{temperature!ambient}
{[}$^{\circ}C${]}\\
$T_{K}$=ATX+$T_{0}$~$^{\dagger}$\sindex[lis]{Tk@$T_{K}$= absolute temperature in kelvin}
= ambient temperature {[}K{]} \\
$p$ = PSXC\index{PSXC} = reference ambient pressure {[}hPa{]}\\
$p_{h}$ = CAVP\_x (e.g.) = pressure in instrument housing {[}hPa{]}\\
$e_{t}$ = intermediate vapor pressure used for calculation only\\
$e$ = EWx = water vapor pressure from source x {[}hPa{]}\\
$M_{w}$ = molecular weight of water$^{\dagger}$\\
$R_{0}$ = universal gas constant$^{\dagger}$\\
$f(p_{h},T_{p})$ = enhancement factor (cf.~(\ref{eq:EnhancementFactor}))\\
$F_{d}(e)$ = interpolation formula\sindex[lis]{Fd@$F_{d}(e)$=interpolation formula for dew point}
giving dew point temperature from water vapor pressure

\rule[0.5ex]{1\linewidth}{1pt}

for dew/frost point hygrometers, producing the measurement DPx:~~~~if
DPx < 0$\text{�}$C:

~~~~~~~~obtain $e_{t}$ from (\ref{eq:MK1-1}) using $T_{FP}$=DPx
+ $T_{0}$

~~~~else (i.e., DPx $\geq$ 0$\text{�}$C):

~~~~~~~~obtain $e_{t}$ from (\ref{eq:proposedNewWater})
using $T_{DP}=\mathrm{DPx}+T_{0}$

~~~~correct $e_{t}$ for enhancement factor and internal pressure,
to get ambient vapor pressure $e$:

~~~~~~~~$e=f(p_{h},T_{P})\,e_{t}\,(p/p_{h})$

~~~~obtain DPxC by finding the dew point corresponding to the
vapor pressure $e$:

~~~~~~~~DPxC = $F_{d}(e)$

\textemdash - \textemdash - \textemdash - \textemdash - \textemdash -
\textemdash - \textemdash - \textemdash - \textemdash - \textemdash -
\textemdash - \textemdash - \textemdash - \textemdash - \textemdash -

for other instruments producing measurements of vapor density (RHO
{[}g~m$^{-3}${]}:\footnote{prior to 2011 the following formula was used: 
\[
Z=\frac{\ln((\mathrm{ATX}+273.15)\,\mathrm{RHO}}{1322.3}
\]

\[
\mathrm{DPxC}=\frac{273.0\,Z}{(22.51-Z)}
\]
}

~~~~find the water vapor pressure in units of hPa:

~~~~~~~~$e=$ (\{RHO\}~$R_{0}\,T_{K}$\,/\,$M_{w}$)$\times10^{-5}$

~~~~find the equivalent dew point:

~~~~~~~~DPxC = $F_{d}(e)$%
\end{minipage}}\\
\begin{comment}
this is a description of the old Bulletin-9 section, saved here for
reference. It did not correspond to the code in use prior to the 2011
change, however; only the >0 form of $f$ was used, and there was
a small error in coefficients, as described in a 2010 Note that documented
the change made in 2011 and the reasons for it.%
\begin{minipage}[t]{1\columnwidth}%
DPX = measured dew point ($\geq0^{\circ}$) or frost point ($<0^{\circ})$
in $^{\circ}C$\\
$D_{K}$ = DPX + 273.15\,K = measured dew point in kelvin\\
$f$= enhancement factor (Appendix C)\\
\\
\rule[[0.5ex]]{1.0\linewidth}{1pt}if DPX$\geq0^{\circ}C:$\\
\[
A=23.832241-5.02808\,\log\left(D_{K}\right)-1.3816\times10^{-7}(10^{11.334-0.0303998(D_{K}})
\]
\[
f=1.0007+(3.46\times10^{-6}\mathrm{PSXC})
\]
if DPX $<0^{\circ}C$:
\[
A=3.56654\log_{10}(D_{K})-0.0032098(D_{K})-\frac{2484.956}{D_{K}}+2.0702294
\]
\[
f=1.0003+(4.18\times10^{-6}\mathrm{PSXC})
\]
vapor pressure:

\begin{equation}
\mathrm{EDPC}=f\,10^{A}\label{eq:8.11GoffG-1}
\end{equation}
%
\end{minipage}
\end{comment}

\textbf{Dew Point Determined from the VCSEL Hygrometer ($^{\circ}$C):
}\textbf{\uline{DP\_VXL}}\index{DP_VXL@DP\_VXL}\sindex[var]{DP_VXL@DP\_VXL}\\
\emph{The dew point temperature determined from the measured water
vapor density from the VCSEL hygrometer. }\index{hygrometer!VCSEL}The
calculation is as described at the bottom of the box immediately above
this paragraph. The water vapor density converted from a molecular
density {[}molecules~cm$^{-3}${]} to a mass density {[}g~m$^{-3}${]}
via\footnote{The conversion factor is given by this formula:\\
\[
C^{\prime}=\frac{10^{6}\mathrm{cm}^{3}}{\mathrm{m}^{3}}\times\frac{M_{W}^{\dagger}}{N_{A}^{\dagger}}
\]
where $N_{A}$ is the Avogadro constant, 6.022147$\times10^{26}$
molecules~kmol$^{-1}$\index{Avogadro constant}\sindex[lis]{NA@$N_{A}$ = Avogadro constant, molecules per kmol}.} \{CONCV\_VXL\}{*}2.9915$\times10^{-17}$ is used for \{RHO\}. DP\_VXL
is given by DPxC on the last line of that algorithm box. See CONCV\_VXL
below.

\textbf{Frost Point Temperature from the CR2 Cryogenic Hygrometer
($^{\circ}$C): }\textbf{\uline{FP\_CR2\sindex[var]{FP_CR2@FP\_CR2}\index{FP_CR2@FP\_CR2}}}\textbf{,}\\
\textbf{\uline{MIRRORT\_CR2}}\index{MIRRORT_CR2@MIRRORT\_CR2}\sindex[var]{MIRRORT_CR2@MIRRORT\_CR2}\\
\emph{The mirror temperature in the CR2 cryogenic hygrometer, }which
is normally the frost point inside the measuring chamber of the instrument\emph{.
}\label{punch:4-7}The measurement is often suspect when the value
is above about -15$^{\circ}$C; the measurement is intended for use
below this value. The CR2 is a cabin-mounted instrument, so the measured
pressure (P\_CR2\sindex[var]{P_CR2@P\_CR2}) in the instrument must
be used with the ambient pressure (PSXC) to convert the measurement
to ambient humidity measures like DP\_CR2 and EW\_CR2. 

\textbf{Corrected Dew Point Temperature from the CR2 Cryogenic Hygrometer
($^{\circ}$C): }\textbf{\uline{DP\_CR2C}}\index{DP_CR2C@DP\_CR2C}\sindex[var]{DP_CR2C@DP\_CR2C}\\
\emph{The dew point temperature corresponding to equilibrium at the
ambient humidity, }as determined by the CR2 hygrometer.\index{hygrometer!CR2}
The measurement of the mirror temperature inside the CR2, FP\_CR2,
is converted to a vapor pressure assuming equilibrium water vapor
pressure relative to a plane ice surface at that temperature, and
the resulting vapor pressure is converted to an ambient value via
the assumption that the ratio of vapor pressure internal to the instrument
to ambient vapor pressure is the same as the corresponding total pressure
ratio. The resulting ambient vapor pressure (EW\_CR2) is then converted
to an equivalent ambient dew point. The steps are the same as those
in the algorithm box above, with these substitutions: FP\_CR2 is used
for DPx and P\_CR2 for $p_{h}$.

\textbf{Uncorrected Water Vapor Number Density from the VCSEL Hygrometer
(molecules cm$^{-3}$): }\textbf{\uline{RAWCONC\_VXL}}\index{RAWCONC_VXL@RAWCONC\_VXL}\sindex[var]{RAWCONC_VXL@RAWCONC\_VXL}
\emph{}\\
\emph{The uncorrected water vapor number density reported by the VCSEL
hygrometer.} \index{water vapor!density}This is determined by comparing
the measured absorption peak height against a reference spectrum generated
using the HITRAN spectral parameters, the ambient temperature and
the ambient pressure.\footnote{For details see Zondlo, M. A., M. E. Paige, S. M. Massick, and J.
A. Silver, 2010: Vertical cavity laser hygrometer for the National
Science Foundation Gulfstream-V aircraft. \emph{J. Geophys. Res.,
}\textbf{115,} D20309, doi:10.1029/2010JD014445.}

\textbf{Corrected Water Vapor Concentration from the VCSEL Hygrometer
(molecules cm$^{-3}$): }\textbf{\uline{CONCV\_VXL}}\uline{:}\index{CONCV_VXL@CONCV\_VXL}\sindex[var]{CONCV_VXL@CONCV\_VXL}\\
\emph{The corrected water vapor number density produced by the VCSEL
hygrometer,} after minor corrections for ambient temperature, pressure,
laser intensity and water vapor concentration.\index{concentration!water vapor}\label{punch:4-8}
For more information on calibration and data processing for this instrument,
see the \href{https://www.eol.ucar.edu/instruments/vertical-cavity-surface-emitting-laser-vcsel-hygrometer}{instrument web page}
and additional documentation there.

\textbf{Voltage Output from the UV Hygrometer (V): }\textbf{\uline{XSIGV\_UVH}}\textbf{\index{hygrometer!UV}\index{XSIGV_UVH@XSIGV\_UVH}}\sindex[var]{XSIG_UVH@XSIG\_UVH}\\
\emph{The voltage from a modern (as of 2012) version of the Lyman-alpha
hygrometer,} which provides a signal that represents water vapor density.
The instrument also provides measurements of pressure and temperature
inside the sensing cavity; they are, respectively, XCELLPRES\_UVH
and XCELLTEMP\_UVH. See the discussion of EW\_UVH below for the data-processing
algorithm that uses this variable.

\textbf{Water Vapor Number Density from the UV Hygrometer (molecules
cm$^{-3}$): }\textbf{\uline{CONCH\_UVH}}\index{CONCH_UVH@CONCH\_UVH}\sindex[var]{CONCH_UVH@CONCH\_UVH}\\
\emph{Water vapor number density (or concentration of molecules) measured
by the UV Hygrometer.} This is the direct measurement from the instrument.
Its calculation relies on a bench calibration that fits the water
vapor number density to the Beers-Lambert absorption law and corrects
for output offsets and the effect of UV absorption by atmospheric
constituents other than water vapor. See also the discussion of EW\_UVH
in the paragraph that immediately follows.

\textbf{Water Vapor Pressure (hPa): }\textbf{\uline{EWx}}\textbf{\sindex[var]{EWx}}\index{EWx}\textbf{,
}\textbf{\uline{EWX}}\textbf{\sindex[var]{EWX}}\index{EWX}\textbf{,}
\textbf{\uline{EW\_UVH,\index{EW_UVH@EW\_UVH}}}\sindex[var]{EW_UVH@EW\_UVH}\textbf{
}\textbf{\uline{EDPC}}\sindex[var]{EDPC}\index{EDPC}\textbf{
}(obsolete)\textbf{ }\\
\emph{The ambient vapor pressure of water, }also used in the calculation
of several derived variables. It is often obtained from an instrument
measuring dew point or water vapor density. In the case where it is
derived from a measurement of dew point (DPx\index{DPXC}), a correction
is applied for the enhancement factor\index{enhancement factor} that
influences dew point or frost point measurements.\footnote{prior to 2011, this variable was calculated using the Goff-Gratch
formula. See the discussion of DPXC for more information on previous
calculations.} The formula for obtaining the ambient water vapor pressure as a function
of dew point is given in the discussion of DPxC above, Eqs.~(\ref{eq:proposedNewWater})
and (\ref{eq:EnhancementFactor}), where the calculation of the variables
EWx and EWX are also discussed. EWX (or previously EDPC) is the preferred
variable that is selected from among the possibilities \{EWx\} for
subsequent calculation of derived variables.\\
\\
For the case where water vapor pressure is determined by the VCSEL
hygrometer, EW\_VXL is determined from CONCV\_VXL: EW\_VXL=$C$$k$\{CONCV\_VXL\}\{ATX+273.15)
where $k$ is the Boltzmann constant and $C=10^{-4}$(cm/m)$^{3}$(hPa/Pa)
converts units to hPa.\\
\\
In the case where the water vapor pressure is determined from the
UV Hygrometer data, this variable is calculated using one of two methods:
(a) using the ideal gas law to convert the water vapor number density
from the UV Hygrometer to water vapor pressure, using XCELLTEMP\_UVH
and XCELLPRES\_UVH, the measured temperature and pressure in the absorption
cell, via the equation\\
\[
\mathrm{EW\_UVH=C\,\{CONC\_UVH\}\,\frac{k\,(\mathrm{\{XCELLTEMP\_UVH\}+273.15)\,\mathrm{\{PSX\}}}}{\mathrm{{\{XCELLPRES\_UVH\}}}}}\,\,\,\,;
\]
or (b) through use of a polynomial fit of the form\\
\[
\mathrm{EWX}=c_{0}+c_{1}{\mathrm{\{XSIGV\_UVH\}}}+c_{2}{\mathrm{\{XSIGV\_UVH\}}}^{2}
\]
\\
where EWX is a reference water vapor pressure provided by another
instrument. This preserves the fast-response characteristics of the
UV hygrometer while linking the absolute values to a baseline provided
by a more stable instrument. This can be done on a flight-by-flight
basis and largely eliminates drift.\footnote{For more details see Beaton, S. P. and M. Spowart, 2012: UV Absorption
Hygrometer for Fast-Response Airborne Water Vapor Measurements. \emph{J.
Atmos. Oceanic Technol., }\textbf{\emph{29. }}DOI: 10.1175/JTECH-D-11-00141.1} See the project reports to determine which method was used for a
particular project.

\textbf{Relative Humidity (per cent or Pa/hPa): }\textbf{\uline{RHUM}}\sindex[var]{RHUM}\index{RHUM}\\
\index{humidity!relative}\index{relative humidity|see {humidity, relative}}\emph{The
ratio of the water vapor pressure to the water vapor pressure in equilibrium
over a plane }liquid\emph{-water surface, }scaled to express the
result in units of per cent or Pa/hPa:\\
 %
\fbox{\begin{minipage}[t]{0.95\columnwidth}%
EWX\index{EWX} = atmospheric water vapor pressure (hPa)\\
ATX\index{ATX} = ambient air temperature ($^{\circ}C$)\\
$T_{0}=$273.15 K\\
$e_{s.w}(\mathrm{ATX+T_{0}})$ = equilibrium water vapor pressure\index{pressure!water vapor!equilibrium}
at \emph{dewpoint} ATX (hPa)\\
~~~~~~~~~~(see eq. \ref{eq:proposedNewWater} for the formula
used.)\\
\\
\\
\rule[0.5ex]{1\columnwidth}{1pt}

\begin{equation}
\mathrm{RHUM}=100\%\,\times\,\frac{\mathrm{\{EWX\}}}{e_{s,w}(\mathrm{\{ATX\}+T_{0}})}\label{eq:8.16RHUM-1}
\end{equation}
 %
\end{minipage}}\\
To follow normal conventions, the change in equilibrium vapor pressure
that arises from the enhancement factor\index{enhancement factor}
is not included in the calculated relative humidity, even though the
true relative humidity should include the enhancement factor as specified
in (\ref{eq:EnhancementFactor}) in the denominator of (\ref{eq:8.16RHUM-1}).
\\

\textbf{Relative Humidity with respect to Ice (per cent or Pa/hPa):
}\textbf{\uline{RHUMI}}\sindex[var]{RHUMI}\index{RHUMI}\\
\index{humidity!relative to ice}\index{relative humidity wrt ice|see {humidity, relative to ice}}\emph{The
ratio of the water vapor pressure to the water vapor pressure in equilibrium
over a plane }ice\emph{ surface, }scaled to express the result in
units of per cent or Pa/hPa:\\
 %
\fbox{\begin{minipage}[t]{0.95\columnwidth}%
EWX\index{EWX} = atmospheric water vapor pressure (hPa)\\
ATX\index{ATX} = ambient air temperature ($^{\circ}C$)\\
$T_{0}=$273.15 K\\
$e_{s,i}(\mathrm{ATX+T_{0}})$ = equilibrium water vapor pressure\index{pressure!water vapor!equilibrium}
at \emph{frostpoint} ATX (hPa)\\
~~~~~~~~~~(see eq. \ref{eq:MK1-1} for the formula used.)\\
\\
\\
\rule[0.5ex]{1\columnwidth}{1pt}

\begin{equation}
\mathrm{RHUMI}=100\%\,\times\,\frac{\mathrm{\{EWX\}}}{e_{s,i}(\mathrm{\{ATX\}+T_{0}})}\label{eq:RHUMI}
\end{equation}
 %
\end{minipage}}\\
To follow normal conventions, the change in equilibrium vapor pressure
that arises from the enhancement factor\index{enhancement factor}
is not included in the calculated relative humidity, even though the
true relative humidity should include the enhancement factor as specified
in (\ref{eq:EnhancementFactor}) in the denominator of (\ref{eq:RHUMI}).
\\

\textbf{Absolute Humidity, Water Vapor Density (g/m$^{3}$): }\textbf{\uline{RHOx}}\sindex[var]{RHOx}\index{RHOx}\\
\emph{\index{humidity!absolute}The water vapor density\index{density!water vapor}\index{water vapor!density}
computed from various measurements of humidity as indicated by the
'x' suffix, }and conventionally expressed in units of g\,kg$^{-1}$
or per mille. The calculation proceeds in different ways for different
sensors. For sensors that measure a \index{chilled-mirror}\index{hygrometer!chilled-mirror}chilled-mirror
temperature, the calculation is based on the equation of state for
a perfect gas and uses the water vapor pressure determined by the
instrument, as in the following box.\label{punch:4-9}\\
\\
\fbox{\begin{minipage}[t]{0.95\textwidth}%
ATX\index{ATX} = ambient temperature ($^{\circ}C$)\\
EWX\index{EWX} = water vapor pressure, hPa\\
$C_{mb2Pa}$\sindex[con]{Cmb2@$C_{mb2Pa}$= conversion factor, hPa to Pa}
= 100 Pa\,hPa$^{-1}$ (conversion factor to MKS units) \\
$C_{kg2g}=$\sindex[con]{Ckg2@conversion factor, kg to g}$10^{3}$\,g\,kg$^{-1}$
\sindex[lis]{Cx2y@$C_{x2y}$=conversion factor from x to y}(conversion
factor to give final units of g\,m$^{-3}$)\\
$T_{0}$ = 273.15\,K

\rule[0.5ex]{1\columnwidth}{1pt}
\begin{eqnarray}
\mathrm{RHOx} & = & C_{kg2g}\frac{C_{mb2Pa}\mathrm{\{EWX\}}}{R_{w}\mathrm{(\{ATX\}+\mbox{\ensuremath{T_{0}}})}}\label{eq:8.17RHO-1}
\end{eqnarray}
%
\end{minipage}}\\
For instruments measuring the vapor pressure density (including the
Lyman-alpha\index{hygrometer!Lyman-alpha} probes and the newer version
called the UV hygrometer\index{hygrometer!UV}), the basic measurement
from the instrument is the water vapor density, \textbf{\uline{RHOUV}}\sindex[var]{RHOUV}\index{RHOUV}
or\textbf{ }\textbf{\uline{RHOLA}}\textbf{\sindex[var]{RHOLA}\index{RHOLA}},
determined by applying calibration coefficients to the measured signals
(XUVI\sindex[var]{XUVI} or VLA\sindex[var]{VLA}). In addition, a
slow update to a dew-point measurement is used to compensate for drift
in the calibration. The algorithm for the UV Hygrometer is as described
in the following box; the processing used for early projects with
the Lyman-alpha instruments is similar but more involved and won't
be documented here because the instruments are obsolete. See \href{http://www.eol.ucar.edu/raf/Bulletins/bulletin9.html}{RAF Bulletin 9}
for the processing previously used for archived measurements from
the Lyman-alpha hygrometers.\\

\textbf{Specific Humidity (g/kg): }\textbf{\uline{SPHUM}}\sindex[var]{SPHUM}\index{SPHUM}\\
\index{humidity!specific}\emph{The mass of water vapor per unit mass
of (moist) air, conventionally measured in units of g/kg or per mille.
}\\
\\
\fbox{\begin{minipage}[t]{0.95\textwidth}%
PSXC\index{PSXC} = ambient pressure. hPa \\
EWX\index{EWX} = ambient water vapor pressure, hPa\\
$C_{kg2g}=$$10^{3}$\,g\,kg$^{-1}$ (conversion factor to give
final units of g\,kg$^{-1}$)\\
$M_{w}=$molecular weight of water$^{\dagger}$\\
$M_{d}=$molecular weight of dry air$^{\dagger}$\\
\\
\rule[0.5ex]{1\columnwidth}{1pt}
\begin{eqnarray}
\mathrm{SPHUM} & = & C_{kg2g}\frac{M_{w}}{M_{d}}(\mathrm{\frac{\{EWX\}}{\mathrm{\{PSXC\}-(1-\frac{M_{w}}{M_{d}})\{\mathrm{EWX}\}}}})\label{eq:8.18SPHUM-1}
\end{eqnarray}
%
\end{minipage}}\\
\\

\textbf{Mixing Ratio (g/kg): }\textbf{\uline{\label{Mixing-Ratio-(g/kg):}MR}}\sindex[var]{MR}\textbf{\index{MR},
}\textbf{\uline{MRCR}}\textbf{\sindex[var]{MRCR}\index{MRCR},
}\textbf{\uline{MRLA}}\textbf{\sindex[var]{MRLA}\index{MRLA},
}\textbf{\uline{MRLA1}}\textbf{\index{MRLA1}, }\textbf{\uline{MRLH}}\sindex[var]{MRLH}\index{MRLH}\textbf{,
}\textbf{\uline{MRVXL}}\index{MRVXL}\sindex[var]{MRVXL}\\
\index{water vapor!mixing ratio}\emph{The ratio of the mass of water
to the mass of dry air in the same volume of air, }conventionally
expressed in units of g/kg or per mille. Mixing ratios may be calculated
for the various instruments measuring humidity on the aircraft, and
the variable names reflect the source: MR from the dewpoint \index{hygrometer!dew point}hygrometers,
MRCR from the cryogenic hygrometer\index{hygrometer!cryogenic}, MRLA
from the Lyman-alpha\index{hygrometer!Lyman-alpha} sensor, MRLA1
if there is a second Lyman-alpha sensor, MRLH from a tunable-diode
laser hygrometer\index{hygrometer!tunable diode laser}, and MRVXL
is from the VCSEL\index{hygrometer!VCSEL} hygrometer (also a laser
hygrometer). The example in the box below is for the case of the dewpoint
hygrometers; others are analogous.\\
\\
\fbox{\begin{minipage}[t]{0.95\textwidth}%
EWX\index{EWX} = water vapor pressure, hPa\\
PSXC\index{PSXC} = ambient total pressure, hPa\\
$C_{2kg2g}=$$10^{3}$\,g\,kg$^{-1}$ (conversion factor to give
final units of g\,kg$^{-1}$)\\
$M_{w}=$molecular weight of water$^{\dagger}$\\
$M_{d}=$molecular weight of dry air$^{\dagger}$\\
\\
\rule[0.5ex]{1\columnwidth}{1pt}
\begin{equation}
\mathrm{MR=C_{kg2g}\frac{M_{w}}{M_{d}}\frac{\mathrm{\{EWX\}}}{(\mathrm{\{PSXC\}-\{EWX\})}}}\label{eq:8.19MR-1}
\end{equation}
%
\end{minipage}}\\

\end{hangparagraphs}


\subsection{Derived Thermodynamic Variables\index{derived variables}}
\begin{hangparagraphs}
\textbf{Potential Temperature (K): }\textbf{\uline{THETA}}\sindex[var]{THETA}\index{THETA}\\
\index{temperature!potential}\index{potential temperature|see {temperature, potential}}\emph{The
absolute temperature reached if a dry parcel at the measured pressure
and temperature were to be compressed or expanded adiabatically to
a pressure of 1000 hPa}. It does not take into account the difference
in specific heats caused by the presence of water vapor, and water
vapor can change the exponent in the formula below enough to produce
errors of 1\,K or more.\\
\fbox{\begin{minipage}[t]{0.95\columnwidth}%
ATX\index{ATX} = ambient temperature, $^{\circ}$C\\
PSXC\index{PSXC} = ambient pressure (hPa)\\
$p_{0}$\sindex[lis]{p0@$p_{0}$= reference pressure equal to 1000 hPa}
= reference pressure = 1000 hPa\\
$R_{d}$ = gas constant\index{gas constant!dry air} for dry air$^{\dagger}$\\
$c_{pd}$ = specific heat\index{specific heat!constant pressure}
at constant pressure for dry air$^{\dagger}$\\
\\
\rule[0.5ex]{1\columnwidth}{1pt}
\begin{equation}
\mathrm{THETA}=\left(\mathrm{\{ATX\}}+T_{0}\right)\left(\frac{p_{0}}{\mathrm{\{PSXC\}}}\right)^{R_{d}/c_{pd}}\label{eq:8.13THETA-1}
\end{equation}
 %
\end{minipage}}\\
\\

\textbf{Pseudo-Adiabatic Equivalent Potential Temperature (K): }\textbf{\uline{THETAP}}\sindex[var]{THETAP}\sindex[var]{THETAE}\index{THETAE}\index{THETAE}\\
\emph{The absolute temperature reached if a parcel of air were to
be expanded pseudo-adiabatically (i.e., with immediate removal of
all condensate) to a level where no water vapor remains, after which
the dry parcel would be compressed to 1000 hPa. }Beginning in 2011,
pseudo-adiabatic equivalent potential temperature\index{temperature!pseudo-adiabatic equivalent potential}\sindex[lis]{thetaP@$\Theta_{P}$=temperature, pseudo-adiabatic equivalent potential}
is calculated using the method developed by Davies-Jones\index{Davies-Jones}
(2009).\footnote{Davies-Jones, R., 2009: On formulas for equivalent potential temperature.
\emph{Mon. Wea. Review, }\textbf{137, }3137\textendash 3148.} This is discussed in the memo available at \href{https://drive.google.com/open?id=0B1kIUH45ca5ATVl1MjV6Q0E0YXc}{this URL}.
The following summarizes that study. The Davies-Jones formula is\\
\begin{equation}
\Theta_{P}=\Theta_{DL}\exp\{\frac{r(L_{0}^{*}-L_{1}^{*}(T_{L}-T_{0})+K_{2}r)}{c_{pd}T_{L}}\}\label{eq:DaviesJonesThetaP-1}
\end{equation}
and\\
\begin{equation}
\Theta_{DL}=T_{K}(\frac{p_{0}}{p_{d}})^{0.2854}\,(\frac{T_{k}}{T_{L}})^{0.28\times10^{-3}r}\label{eq:ThetaDL-1}
\end{equation}
\\
where $T_{K}$ is the absolute temperature (in kelvin) at the measurement
level, $p_{d}$\sindex[lis]{pd@$p_{d}$= partial pressure of dry air}
is the partial pressure of dry air at that level, $p_{0}$ is the
reference pressure (conventionally 1000 hPa), $r$\sindex[lis]{r@$r$=water-vapor mixing ratio, dimensionless}
is the (dimensionless) water vapor mixing ratio, $c_{pd}$ the specific
heat of dry air, $T_{L}$\sindex[lis]{temperaturelifted@$T_{L}$= temperature, lifted condensation level}
the temperature at the lifted condensation level\index{lifted condensation level}
(in kelvin), and $T_{0}=273.15\,$K. The coefficients in this formula
are: $L_{0}^{*}=2.56313\times10^{6}\mathrm{{J\,kg^{-1}}}$, $L_{1}^{*}=1754$
J\,kg$^{-1}\,\mathrm{{K}^{-1}}$, and $K_{2}=1.137\times10^{6}$\,J\,kg$^{-1}$.
The asterisks on $L_{0}^{*}$ and $L_{1}^{*}$ indicate that these
coefficients depart from the best estimate of the coefficients that
give the latent heat of vaporization\index{latent heat of vaporization}
of water, but they have been adjusted to optimize the fit to values
obtained by exact integration. Note that, unlike the formula discussed
below that was used prior to 2011, the mixing ratio must be used in
dimensionless form (i.e., kg/kg), \emph{not} with units of g/kg. The
following empirical formula, developed by Bolton\index{Bolton} (1980),\footnote{Bolton, D., 1980: The computation of equivalent potential temperature.
\emph{Mon. Wea. Rev.,} \textbf{108,} 1046\textendash 1053. } is used to determine $T_{L}$:\\
\begin{equation}
T_{L}=\frac{\beta_{1}}{3.5\ln(T_{K}/\beta_{3})-\ln(\mathrm{e/\beta_{4}})+\beta_{5}}+\beta_{2}\label{eq:TLCL-1}
\end{equation}
where $e$ is the water vapor pressure, $\beta_{1}=2840\,K$, $\beta_{2}=55\,$K,
$\beta_{3}=1$\,K, $\beta_{4}=1$\,hPa, $\beta_{5}=-4.805$. (Coefficients
$\beta_{3}$ and $\beta_{4}$ have been introduced into (\ref{eq:TLCL-1})
only to ensure that arguments to logarithms are dimensionless and
to specify the units that must be used to achieve that.)\\
\\
Prior to 2011, the variable called the equivalent potential temperature\footnote{The AMS glossary defines equivalent potential temperature as applying
to the adiabatic process, not the pseudo-adiabatic process; the name
of this variable has therefore been changed.}\index{temperature!equivalent potential} and named THETAE in the
output data files was that obtained using the method of Bolton (1980),
which used the same formula to obtain the temperature at the lifted
condensation level ($T_{L}$) and then used that temperature to find
the value of potential temperature\index{temperature!potential} of
dry air that would result if the parcel were lifted from that point
until all water vapor condensed and was removed from the air parcel.
 The formulas used were as follows:\\
\\
\fbox{\begin{minipage}[t]{0.95\textwidth}%
$T_{L}$= temperature at the lifted condensation level, K\\
ATX\index{ATX} = ambient temperature ($^{\circ}C$)\\
EDPC\index{EDPC} = water vapor pressure (hPa) \textendash{} now superceded
by EWX\\
MR\index{MR} = mixing ratio (g/kg)\\
THETA\index{THETA} = potential temperature (K)\\
\\
\rule[0.5ex]{1\columnwidth}{1pt}
\[
T_{L}=\frac{2840.}{3.5\ln(\mathrm{\{ATX\}+T_{0}})-\ln(\mathrm{\{EDPC\}})-4.805}+55.
\]
\begin{equation}
\mathrm{THETAE}=\mathrm{\{THETA\}\left(\frac{3.376}{T_{L}}-0.00254\right)(\{MR\})(1+0.00081(\{MR\}))}\label{eq:8.13THETAE-1}
\end{equation}
%
\end{minipage}}\textbf{}\\
Differences vs the new formula are usually minor but can be of order
0.5\,K.\textbf{}\\

\textbf{Virtual Temperature ($\text{�}$C): }\textbf{\uline{\label{TVIR}TVIR}}\sindex[var]{TVIR}\index{TVIR}\\
\index{temperature!virtual}\emph{The temperature of dry air having
the same pressure and density as the air being sampled.} The virtual
temperature thus adjusts for the buoyancy added by water vapor. \\
\fbox{\begin{minipage}[t]{0.95\columnwidth}%
ATX\index{ATX} = ambient temperature, $^{\circ}C$\\
$r$ = mixing ratio, dimensionless {[}kg/kg{]} = \{MR\}/(1000 g/kg)\\
$T_{0}=273.15$\,K\\
\\
\\
\rule[0.5ex]{1\columnwidth}{1pt}

\begin{equation}
\mathrm{TVIR}=(\mathrm{\{ATX\}}+T_{0})\left(\frac{1+\frac{M_{d}}{M_{w}}r}{1+r}\right)-T_{0}\label{eq:814TVIR-1}
\end{equation}
 %
\end{minipage}}\\
\\

\textbf{Virtual Potential Temperature (K): }\textbf{\uline{THETAV}}\sindex[var]{THETAV}\index{THETAV}\\
\index{temperature!virtual potential}\emph{A potential temperature
analogous to the conventional potential temperature except that it
is based on virtual temperature instead of ambient temperature.} Dry-adiabatic
expansion or compression to the reference level (1000 hPa) is assumed.
As for THETA, use of dry-air values for the gas constant and specific
heat at constant pressure can lead to significant errors in humid
conditions. For further information, see \href{https://drive.google.com/open?id=0B1kIUH45ca5AZXZZbWJSbEctdmM}{this note}.\\
\fbox{\begin{minipage}[t]{0.95\columnwidth}%
TVIR\index{TVIR} = virtual temperature, $^{\circ}C$\\
PSXC\index{PSXC} = ambient pressure, hPa\\
$R_{d}=$gas constant for dry air$^{\dagger}$\\
$c_{pd}=$specific heat at constant pressure for dry air$^{\dagger}$\\
$T_{0}=273.15$\,K\\
$p_{0}$ = reference pressure, conventionally 1000 hPa

\rule[0.5ex]{1\columnwidth}{1pt}
\begin{equation}
\mathrm{THETAV}=\left(\mathrm{\{TVIR\}}+T_{0}\right)\left(\frac{p_{o}}{\mathrm{\{PSXC\}}}\right)^{R_{d}/c_{pd}}\label{eq:8.15THETAV-1}
\end{equation}
 %
\end{minipage}}\href{file:https://drive.google.com/open?id=0B1kIUH45ca5AZXZZbWJSbEctdmM}{this note}\\

\textbf{Wet-Equivalent Potential Temperature (K): }\textbf{\uline{THETAQ}}\index{THETAQ}\sindex[var]{THETAQ}\\
\emph{The absolute temperature reached if a parcel of air were to
be expanded adiabatically (i.e., retaining the condensed water in
the liquid phase and accounting for the specific heat of that condensate)
to a level where no water vapor remains, after which the condensate
would be removed and the resulting dry parcel compressed to 1000 hPa.
}This variable was not included in data archives prior to 2012. Emanuel
(1994) gives the following formula (his Eq. 4.5.11):\sindex[lis]{temperatureweteq@$\Theta_{q}=$temperature, wet-equivalent potential}\emph{
}\\
\begin{equation}
\Theta_{q}=T(\frac{p_{0}}{p_{d}})^{\frac{R_{d}}{c_{pt}}}\exp\left\{ \frac{L_{v}r}{c_{pt}T}\right\} \left(\frac{e}{e_{s,w}(T)}\right)^{-rR_{w}/c_{pt}}\label{eq:ThetaQEquation}
\end{equation}
where $\Theta_{q}$ is the wet-equivalent potential temperature, $L_{v}$
the latent heat of vaporization\sindex[lis]{Lv@$L_{v}$=latent heat of vaporization of water}\index{latent heat of vaporization},
$r$ the (dimensionless) water-vapor mixing ratio\sindex[lis]{r@$r$=water-vapor mixing ratio, dimensionless},
$c_{pt}=c_{pd}+r_{t}c_{w}$ where $r_{t}$ is the total-water mixing
ratio including vapor and condensate, $c_{w}$ is the specific heat
of liquid water, and other symbols are as used previously. See \href{https://drive.google.com/open?id=0B1kIUH45ca5ATVl1MjV6Q0E0YXc}{this memo}
for additional discussion of this variable, for values to use for
the latent heat and specific heat, and in particular for analysis
indicating that $\Theta_{q}$ evaluated with this formula can be expected
to vary from the true adiabatic value by a few tenths kelvin (in a
worst case, by about 1 K) because of variation in (and uncertainty
in) the specific heat of supercooled water at low temperature. The
details of the calculation are described in the following box. Note
that this algorithm only uses the liquid water content as measured
by a King probe, PLWCC; other similar calculations could be based
on other measures of liquid water such as that from a cloud-droplet
spectrometer.\\
\fbox{\begin{minipage}[t]{1\columnwidth - 2\fboxsep - 2\fboxrule}%
$e=$\{EDPC\}{*}100 = water vapor pressure (Pa)\\
ATX\index{ATX} = ambient temperature ($^{\circ}C$)\\
$r=$\{MR\}/1000.\index{MR} = mixing ratio (dimensionless)\\
$p_{d}=$(\{PSXC\}$-$\{EDPC\}){*}100 = ambient dry-air pressure (Pa)
\\
$p_{0}=$reference pressure for potential temperature, 10$^{5}$Pa
\\
$\chi=$\{PLWCC\}/1000.=cloud liquid water content\sindex[lis]{chi@$\chi$=liquid water content}
(kg\,m$^{-3}$)\\
$R_{d}=$gas constant for dry air$^{\dagger}$\\
$\rho_{d}=$density of dry air = $\frac{p_{d}}{R_{d}(\{ATX\}+T_{0})}$\\
$c_{pd}=$specific heat of dry air$^{\dagger}$\\
$c_{w}=$specific heat of liquid water$^{\dagger}$\\
$L_{V}=L_{0}+L_{1}\mathrm{\{ATX\}}$ where $L_{0}=2.501\times10^{6}\mathrm{J}\,\mathrm{kg^{-1}}$
and $L_{1}=-2370\,\mathrm{J\,\mathrm{kg^{-1}\,\mathrm{K^{-1}}}}$\\
\\
\rule[0.5ex]{1\columnwidth}{1pt}\\
\[
r_{t}=r+(\chi/\rho_{d})
\]
\[
c_{pt}=c_{pd}+r_{t}c_{w}
\]

If outside cloud or below 100\% relative humidity, define

\[
F_{1}=\left(\frac{e}{e_{s,w}(T)}\right)^{-\frac{rR_{w}}{c_{pt}}}\,\,\,\,\,\,,
\]

otherwise set $F_{1}=1$. 

\[
T_{1}=\mathrm{(\{ATX\}}+T_{0})\left\{ \frac{p_{0}}{(\mathrm{\{PSXC\}}-\mathrm{\{EDPC\})}}\right\} ^{\frac{R_{d}}{c_{pt}}}
\]
\[
\mathrm{THETAQ}=T_{1}F_{1}\exp\left\{ \frac{L_{v}r}{c_{pt}(\{\mathrm{ATX\}}+T_{0})}\right\} 
\]
%
\end{minipage}}

\end{hangparagraphs}


